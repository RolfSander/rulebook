\chapter{Road Racing}

\section{Definition}
A Road Race is defined as a unicycle race that runs predominantly on public roads, public bike paths and/or in other public areas. The itinerary may or may not be closed for public use during the event. 

\section{General Rules}
Water/food stations are recommended. Personal music systems are not allowed for any races on public roads where there may be motorized traffic. Riders can be divided by age and/or unicycle type, such as 24” and 29" track unicycles, Standard (any size wheel and cranks), and Unlimited (see definitions). 24” and smaller wheels are not allowed for races longer than 20km without express permission of the racing director. 
Traditional road race distances have been 10k, Full Marathon (42.195k) and 100k.

\subsection{Traditional Races}
Traditional road race distances have been 10k with Unlimited and Standard 24” classes; Full Marathon (42.195k) with Unlimited and Standard 29” classes; and 100k with solo racers, small teams (2-4 racers), and large teams (5-12 races), all Unlimited class.