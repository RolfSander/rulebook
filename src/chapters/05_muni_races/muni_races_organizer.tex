\chapter{Event Organizer Rules}

\section{Venue}

\textbf{Uphill?}

\oldrule{4.10}
For the downhill race, courses must be primarily downhill but may include flat or uphill sections.
Recommended course length is 2.5 km, or 1 km at a minimum, depending on available terrain, trails and schedule time.

\oldrule{4.11}
A Cross Country race should be at least 10 km or longer, depending on available terrain, trails and schedule time.
If only shorter trails are available, riders can be required to complete two or more laps of the course.

\oldrule{4.4}
Courses must be clearly marked, so that riders can easily see where to go.
Very dangerous sections should be secured (for example by removing sharp stones/branches from areas where riders are likely to fall/run into due to the physics of the course).

\section{Officials}

\begin{framed}
Does the IUF require particular officials for these events?
\end{framed}

\section{Communication}

\begin{framed}
Some thoughts on what might need to be communicated:
\begin{itemize}
\item false start method
\item age groups
\item starting method
\item timing method?
\item course map?
\item results
\end{itemize}
\end{framed}

\oldrule{4.7}
The host must publish two lists of results for each discipline after the competition: Age group based ranking and overall ranking (separating
male/female).

\oldrule{4.2}
If the hosts wish to include events other than the first three (Up, DH, XC), they must remember to provide detailed rules for these events at the same time the events are announced.

\section{Age Groups}
\oldrule{4.6}
Age groups must be offered as male and female age group.
There must not be any age group specific restrictions on equipment.
The following age groups are the maximum allowable for muni competitions:

\begin{tabular}{ l l}
Under 15 & Youth \\
15-16 & Juniors \\
17-18 & Rookies \\
19-29 & Elites \\
30-49 & Masters \\
50+ & Veterans \\
\end{tabular}

\section{Practice}

\oldrule{4.4}
For all muni races, every rider must get the chance of at least one test run to get familiar with the track before the actual race.
If possible, the track should be open for training during all days of the event prior to the race.
For multiday events the muni competitions should take place during the second half of the event in order to give riders more time to practice on the course.

\section{Race Configuration}

\oldrule{4.10}
For a downhill course length less than 2 km, two separate runs should be held.
In this case the ranking of the riders is based on the fastest of the two runs.
Riders should race one at a time, released at regular time intervals.
If the schedule has a small time window for the race, riders should be run in heat sizes that allow passing on the course, and do not bottleneck at the beginning.

\section{Starting Configuration}

\oldrule{4.5}
There are three different types of starting modes, that can be used in muni races.
\begin{enumerate}
\item \textbf{Mass starts:} All riders start at the same time.
Mass starts must not be used when the race duration is expected to be shorter than 30 minutes.
The track must provide sufficient space for passing in the first section, so that the field of starters is aligned before the track narrows down.
Space for passing must be given along the track.
Mass starts with more than 40 riders have to be split to avoid accidents.
\item \textbf{Heat starts:} Groups of riders start at intervals that can vary from 30 seconds to a few minutes.
The maximum number of riders per heat is determined by the average width of the first 100m of the track.
There can be one rider for each meter in width.
\item \textbf{Individual starts:} Individual riders start at intervals that can vary from 30 seconds to a few minutes.
\end{enumerate}

\section{Starting Order}

\oldrule{4.5}
The fastest riders should always start first, regardless of the starting mode.
The order can be determined by seeding runs.


\section{Starter}

\begin{framed}
Are there any starter requirements?
\end{framed}

\section{False Starts}

\begin{framed}
Are there any false start rules?
\end{framed}

\section{Finishes}

\begin{framed}
Are there any finishing?
\end{framed}

\section{Disputes}

\begin{framed}
Some words about how disputes are handled should go here.
\end{framed}

