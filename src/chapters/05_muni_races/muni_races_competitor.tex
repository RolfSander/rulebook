\chapter{Competitor Rules}

\section{Safety}
\oldrule{5.3}
For all muni events, riders must wear shoes, knee pads, gloves/wrist-guards and helmets (see definitions in chapter \ref{chap:general_definitions}). Additional equipment such as shin, elbow or ankle protection are optional.

Water and food are the responsibility of the rider. Hosts may offer food and water stations at their discretion.

\section{Unicycles}

\oldrule{5.1}
There are no restrictions on wheel size, crank arm length, brakes, or gearing.

\section{Rider Identification}

Riders must wear their race number clearly visible on their chest so that it is visible during the race and as the rider crosses the finish line.
Additionally, the rider may be required to wear a chip for electronic timing.

\section{Protests}

Protests must be filed on an official form within two hours of the posting of event results.
Every effort will be made for all protests to be handled within 30 minutes from the time they are received.

\section{Event Flow}

\subsection{Uphill Race \label{sec:muni_uphill}}

\oldrule{5.9}
An Uphill muni race challenges a riders ability to climb.
Courses may be short and steep or longer, endurance-related challenges. 

\begin{comment-2016}
We decided to remove the following variant, because it doesn't fit the
scope of this chapter.  If such a race were to be run,
the rules would have to be thought out and documented.

Generally it is a timed event, but on an extremely difficult course, riders can be measured as to how far they ride before dismounting.
The race can be offered as a no-dismounts challenge, which either measures who gets the farthest, or disqualifies anyone who doesn't complete the distance without a dismount.
Multiple tries can be allowed, or the race can be a simple timed event.
\end{comment-2016}

\subsection{Downhill Race \label{sec:muni_downhill}}

\oldrule{5.10}
A Downhill muni race is a test of speed and ability to handle terrain while riding downhill.

\subsection{Cross Country (XC) Race\label{sec:muni_xc}}

\oldrule{5.11}
The Cross Country race is an off-road distance race that challenges a rider's fitness and ability to ride fast on rough terrain.

\subsection{Starting}

Riders start with the fronts of their tires (forwardmost part of wheel) behind the nearest edge of the starting line.

\begin{comment-2016}
\subsection{False Starts}

Currently there are no rules for false starts.
\end{comment-2016}

\subsection{Passing}

\oldrule{5.4, 2.8.1} %comment-2016 Really? We don't need this much space
No physical contact between riders is allowed.
Riders must maintain a minimum of one (24$"$) wheel diameter (618mm as judged by eye) between each other when passing, and at all other times.
This is measured from wheel to wheel, so that one rider passing another may come quite close, as long as their wheels remain at least 618mm apart.

\subsubsection{Passing: Downhill}

\oldrule{from Scott at UNIOEC}
If a faster rider comes from behind, the rider in front does not need to yield to the rider behind, as long has he/she is mounted.
The faster rider should try to pass when safe. A mounted rider always has priority over an unmounted rider.

\subsection{Dismounts}
\oldrule{5.8}
Dismounts are allowed in all muni races unless otherwise noted.
In mass-start events, dismounted riders must yield to mounted riders behind them as quickly as possible after a dismount, and until re-mounted.
Riders may not impede the progress of mounted riders when trying to mount.
If necessary they must move to a different location so mounted riders can pass.
If riders choose not to ride difficult sections of the course, they must not pass any mounted riders while walking or running through them.
In time trial-type events, see below for variations based on the other event details.
Violations of these non-riding rules may result in disqualification or a time penalty, to be determined and announced before the race start.

\subsubsection{Dismounts: Uphill}
\oldrule{5.9.1}
If the Uphill race is run as a time trial, riders are intended to ride the entire distance.
In the event of a dismount, the rider must remount the unicycle:
\begin{enumerate}[(a)]
\item At the point where the dismount occurred if the unicycle falls back down the course toward the start.
\item Where the unicycle and/or rider come to a stop after dismounting.
Excessive running/walking/stumbling after a dismount may be grounds for a
penalty at the discretion of the Referee.
\item Riders may also choose to back up (toward the start line) from one of those spots to remount, if they prefer the terrain there.
\end{enumerate}

\subsubsection{Dismounts: Downhill}%scott This needs to be cleaned up. See google doc.

\oldrule{5.10.1}
Dismounted riders must not impede the progress of, or pass mounted riders.
They must remain aware of riders coming from behind, and not block them with their
unicycles or bodies.

Running and fast walking are not allowed, except momentarily to slow down after an unplanned dismount.
After a dismount, riders have to come to a complete halt before mounting the unicycle again.
Riders may generally walk slowly if necessary.
The following penalties apply if riders disregard this:
\begin{itemize}
\item Riders get an immediate time penalty of five seconds when they intentionally run or walk fast, not recovering from a fall.
A judge must clearly indicate when the time penalty starts and when the rider may continue, for example by blowing a whistle and counting down from five.
\item Riders get disqualified immediately when they do not stop and wait five seconds after the judge's indication.
The disqualification should be signaled to the rider immediately by a judge, for example by blowing a whistle twice.
\item Judges must be trained and tested to correctly enforce these rules.
Riders must be informed about the type of signaling prior to the race.
\end{itemize}

\oldrule{from Scott}
If you fall off your unicycle, you must stop running as quickly as possible and you
must come to a complete halt before remounting. If you fall in front of your unicycle,
you may run back up the hill to retrieve it, but you must come to a complete halt
before remounting.
If you do not stop quickly and continue to run, a course marshall will blow a whistle
once and count down from 5. If a whistle is blown at you due to running, you must
stop and wait those five seconds.
If you do not stop after the whistle is blown, a second whistle is blown and you are
immediately disqualified.
If you choose to dismount for a difficult section, you must walk slowly through the
section until stopping to remount. If you run, the course marshall will blow a whistle
and count down from 5, during which the rider must stop and wait. Ignoring the
whistle results in a second whistle and an immediate disqualification.

\subsubsection{Dismounts: Cross Country}

\oldrule{5.11.1}
If the event is held as a time trial, dismounted rider restrictions must be announced before the start of the race.
Depending on course length and difficulty, dismounted riders may be required to walk, or walk only limited distance, or have no restrictions at all.

\subsection{Finishes}

\oldrule{5.8}
Riders must also ride completely across the finish line\textit{, as described in section \ref{sec:track-field_finishes}}. 

\oldrule{2.6}
Riders must cross the line mounted and in control of the unicycle.
``Control'' is defined by the rearmost part of the wheel crossing completely over the finish line with the rider having both feet on the pedals.
In the event of a dismount at the finish line the rider must back up, remount and ride across the finish line again.
