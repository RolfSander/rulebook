\chapter{Judges and Officials Rules}

\section{Chief Judge}

\subsection{Training Officials}
\oldrule{1.20}
As the rules state, competitions cannot be started until all key artistic officials have been trained and understand their tasks.
For Artistic events, the Chief Judge is in charge.
For certain artistic events, a minimum level of judging experience is required.
See section \ref{sec:freestyle_judging-panel}.

\subsection{Starting Groups}
\oldrule{6.49}
If there are obvious critical groups (e.g. all the best riders in one group), the chief judge is allowed to modify the groups.

\section{Judges}
\oldrule{6.51}
The Chief Judge composes the Judging Table for each starting group.
All judges can be active competitors or non-competitors.
Preferred are people with judging experience and competitors.
Non-Competitors can apply for being a judge by contacting the Chief Judge in advance.
The Chief Judge sets the application deadline.
It is recommended that every starting group is judged by two other starting groups.
The judging table consists of at least 5 Judges.

\section{Judging Table}

\section{Judging}
\oldrule{6.52}
Every routine is judged by the judging table.
One judging table stays for one starting group.
Judges can judge alone or in pairs.
Judging in pairs is the preferred system.
All judges must either judge alone or in pairs so that each judge's vote has equal weight.
(Pair judges are referred to as one judge below.)
The judge should rank the riders of the current starting group in order.
They should do this by comparing the difficulty of the shown skills.
The same skill when completed with higher quality (for example elegant, smooth, or clean) is considered more difficult.
Assigning the same rank to multiple riders is allowed.

Only executed skills are taken into account.
An executed skill is defined as when the rider reaches the point of being in control.

Examples:
\begin{itemize}
\item The landing of a unispin is part of the skill.
The rider can only reach the point of being in control after landing.
If the rider is hopping four times after the unispin without control and then falls off the unicycle, the skill does not count.
\item In coasting, the rider is in control after getting far enough.
Getting back to pedals is a separate skill.
\end{itemize}

Negative aspects like dismounts are ignored.
Every judge should use blank sheets of paper to take notes.

The highest and the lowest placing points per rider are discarded.
All the remaining placing points get summed up for each rider.
The 3 riders with the fewest points win and advance to the next round.

