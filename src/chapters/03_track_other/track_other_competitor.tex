\chapter{Competitor Rules}

\section{Safety}

Riders must wear shoes, knee pads and gloves (definitions in chapter \ref{chap:general_definitions}), unless otherwise noted, below.

Riders on wheels larger than 24 Class (or with gearing) must also wear helmets.

The Downhill Coast, Downhill Glide, and 50m Fast Backward races require helmets.

\section{Unicycles}

Only standard unicycles may be used.
Riders may use different unicycles for different racing events, as long as all comply with the rules for events in which they are entered.

For events divided by wheel size, there is a maximum allowable tire diameter and minimum crank arm length for each category:

\begin{longtable}{|p{3cm}|p{5cm}|p{2cm}|p{2cm}|}
\hline
\textbf{Unicycle Class} & \textbf{Max Diameter} & \textbf{Min Crank Length} & \textbf{Transmission}\\
\hline
16 Class & 418mm & 89mm & standard \\
\hline
20 Class & 518mm & 100mm & standard \\
\hline
24 Class & 618mm & 125mm & standard \\
\hline
29 Class & 778mm & No limit & standard \\
\hline
Unlimited Class & No limit & No limit & unlimited \\
\end{longtable}

For any tire in question, its outside diameter must be accurately measured.

Crank arm length is measured from the center of the wheel axle to the center of the pedal axle.
Longer sizes may be used.

In all track racing events on standard unicycles, shoes must not be fixed to the pedals in any way (no click-in pedals, toe clips, tape, magnets or similar).

\section{Rider Identification}

Riders must wear their race number clearly visible on their chest so that it is visible during the race and as the rider crosses the finish line.
Additionally, the rider may be required to wear a chip for electronic timing.

\section{Protests}

Protests must be filed on an official form within 30 minutes of the posting of event results.
For a large event (like Unicon) this period should be extended to 60 minutes, if possible.
The time may also be extended for riders who have to be in other races during the protest period.
All protests will be handled within 30 minutes from the time they are received.
Mistakes in paperwork, inaccuracies in placing, and interference from other riders or other sources are all grounds for protests.
All Referee decisions are final, and cannot be protested.

\section{Wheel Size Categories}

Wheel sizes for track racing are 20 Class, 24 Class and 29 Class.
Additional groups for 16 Class or other wheels can be added.
When not otherwise specified, 24 Class is the maximum wheel size above age 10.
For age groups with a maximum age of 10 or younger, the maximum wheel size is 20 Class (or smaller, if smaller sizes are also used).
The youngest age group for 24 Class wheels should have a minimum age of 0, so riders 10 and younger have the option of racing on 24 Class with those groups (e.g.\ 0-13 or 14-16).
All riders in age groups with a maximum age of 10 or younger will race a 10m Wheel Walk, and 10m Ultimate Wheel, if used (instead of 30m).

\section{Event Flow}

\begin{comment2016}
  A lot of work needs to be done, if these event rules are going to be formalized.
  For now, they can borrow from Track, and an event organizer needs to specify the rules they want to use.
\end{comment2016}

In general, the rules of Track apply, such as false starts, lane use, and dismounts.

\subsection{Relay (Track)}
Usually 4 x 100m or 4 x 400m like in athletics.

The takeover zones are 20 meters long and must be marked on the track.
Riders may remount if necessary, and must pick up the baton if it is dropped.
The handover of the baton must be within the takeover zone.
This means that before the baton crosses the start mark of the takeover zone \emph{only} the incoming rider is in touch with the baton and at the end of the takeover zone \emph{only} the outgoing rider is in touch with the baton.
Riders may not throw the baton to make a pass and may not touch the ground with any part of their body while making a pass.
If the baton is not handed over within the marked takeover zone, the team will be disqualified.
Leaving of the lane within the takeover zone or when remounting does not result in disqualification as long as the riders do not obstruct, impede or interfere with another rider's progress.
There is no defined preparation area for the next riders as long as they stay within their lanes.

Mixed male/female teams may be used, and reasonable age groups may be used depending on the number of expected competitors of the event.
Each relay team may have any mix of ages, the age of the oldest rider determines the age group.

\subsection{Coasting Events}
An event to determine which rider coasts the furthest distance.
Riders' coasting distances are measured from a `starting line' with a 5 meter minimum, which will be marked by a `qualifying line.'
If the rider does not cross the qualifying line it will count as a failed attempt.
The farthest distance from the line wins.
The distance is measured to the rearmost part of the rider that touches the ground when dismounting, or to the rear of the tire where the rider stops coasting.
Remounting is not allowed.
Riders must not touch any part of their tires, wheels or pedals while coasting.
Riders get two attempts.
If a rider crosses the coasting line (front of the tire) not in coasting position, he or she is disqualified in that attempt.
The riding surface should be as smooth and clean as possible, and it may be straight or curved.
Ample time must be allowed for all competitors to make some practice runs on the course before the official start.
The type of event(s) to be used should be announced well in advance of the competition.
Crank arm rules do not apply in any coasting or gliding events.

\subsubsection{Road Coasting}

This event is best held on a roadway with a very slight downward slope.
Riders are allowed an unlimited distance to speed up and start coasting before the starting line.

\subsubsection{Track Coasting \label{subsubsec:track-field_alternate-optional-fun-events_coasting_track-coasting}}
30 meter speed-up distance.
This event is held only on a track, or a very level, smooth surface.
Wind must be at a minimum for records to be set and broken.
This event can be compared with other races at different tracks worldwide.

\subsubsection{Downhill Coasting}
This is a speed coasting event.
Riders start from a standstill, or speed up to the `starting line'.
Riders are timed over a measured distance to the finish line.
Dismounts before the finish line disqualify the rider in that attempt.
The slope must be very gradual for this event to be safe, and helmets are mandatory.

\subsubsection{Indoor Coasting}
30 meter starting distance.
This event is held indoors in a gym, or on a very level, smooth surface.
Rider will coast in a circle on the outer edge of the gym, separated by cones.
Both directions are allowed for the start (clockwise or counterclockwise), and rider will have a maximum of 30m before beginning to coast.
Indoor coasting is the recommended coasting competition at a Unicon.

\subsection{Gliding Events}

Gliding is like coasting, but with one or both feet dragging on top of the tire to provide balance from the braking action.
These events are similar to the coasting events above, with riders gliding for time or distance from a given point.
The rules are the same as for the coasting events (above) with the addition that the riding surface must be dry.
Coasting is allowed.

\subsubsection{Slope Glide Or Track Glide}
A slope glide can be done on a small hill.
Riders start on the hill, gliding down to level ground and continuing as far as they can before stopping.
This event can have a limited starting distance, or no starting distance at all, with riders gliding from a dead stop.
If it is a Track Glide, it is held on a track with the same rules as Track Coasting (see section \ref{subsubsec:track-field_alternate-optional-fun-events_coasting_track-coasting}).

\subsubsection{Downhill Glide}
A downhill race for speed.
Riders start from a standstill, or speed up to the `starting line.'
Riders are timed over a measured distance to the finish line.
Dismounts before the finish line disqualify the rider in that attempt.
Helmets are mandatory.

\subsection{Slow Forward}

In Slow Forward, the rider riders in a continuous forward motion as slowly as possible without stopping, going backward, hopping or twisting more than 45 degrees to either side on a 10 m x 15 cm board.
There are no crank arm length or wheel size restrictions for this event.

Riders must wear shoes.
No other safety gear is required.

\subsubsection{Timing}
The position of the unicycle during a Slow Race is measured from the bottom of the unicycle wheel.
In a Slow Race, the rider starts behind the starting line.
On command by the starter, the rider has 10 seconds to start forward motion and let go off the starting post.
The timer starts recording time when the bottom of the wheel crosses the starting line.
The time stops when the bottom of the wheel crosses the finish line, or touches the ground after the end of the board that marks the finish line.


\subsubsection{Penalty Rules}
The judges give penalties to riders who seem to make ``micro-errors'' (for example twisting about 46 or 48 degrees or vibrations of the wheel) or if they are in doubt if an error was made.
Each penalty deducts one second from the ridden time.
Riders are still disqualified if their wheel comes off the board or other obvious errors are made, for example dismounting or twisting 90 degrees.

\subsubsection{Rules For International and Large Competitions}
These rules are required to be used at Unicon.

\textbf{Qualification round:}
\begin{itemize}
\item Riders must complete a time equal or greater than 45 seconds to move on to the finals.
\item Riders get two attempts to complete this result.
\item Previous results are valid: If a rider has already completed a result of 45 seconds or greater at another competition, they can start automatically in the finals and they don't have to take part in the qualification round, provided that the result can be found in an official result list.
\item The boards can be marked with tape on the floor.
\item No age groups will be ranked.
\item Results will not be valid for records (world, continental, national and regional records).
\end{itemize}

\textbf{Final round:}
\begin{itemize}
\item There will only be one team of judges, in order to have a fair competition.
\item All riders who are qualified for the final round start here.
\item Riders get two attempts.
\item Only results from the finals will be valid for records ( world, continental, national and regional records).
\item The champion is the rider who performs the best result in the final round.
\end{itemize}

\subsubsection{Options for Smaller Competitions}
At regional or national championships, the host can decide to offer age groups ranking and awards, and adjust the qualification time to a lower time as needed.
If the hosts decides to offer age groups, the results from the qualification round count for age group results.
However, the final round is still required.
The results from the final round will also be included in the ranking for age group results.
Previous results from other events are not valid to be included in the age group results.
If the host decides to offer age groups, the board size of 10 m x 30 cm can be used for the 0-10 age group.

\subsection{Slow Backward}
This is the same as the Slow Forward race, with the following differences:
\begin{itemize}
\item Riders ride backward.
\item It is an error to ride forward.
\item Riders ride on a 10 m x 30 cm board.
\item If the host of a national or regional championship decides to offer age groups, the board size of 10 m x 60 cm can be used for the 0-10 age group.
\item Riders move on to the finals if they have completed a time equal or greater than 40 seconds, previous results are valid.
\end{itemize}

\subsection{Slow Giraffe Race}

This is the same as slow forward, but on giraffes.
Helping hands can be used as starting posts.
No limits on size or gear ratio, but unicycles must have their pedal axle above the wheel axle, with a chain, belt, or other form of drive system.

\subsection{Stillstand}
Stillstand is a competition in which the rider attempts to balance as long as possible.
The rider cannot hop or turn the tire more than 45 degrees, and must remain on a 25 cm long, 10 cm wide, and 3 cm tall block of wood.
The competition should take place indoors on a level surface
The only required safety gear is shoes.

Each participant has 2 attempts that can be done at any time during the time window set by the host.
The host can decide to add to each of the 2 attempts a window up to 20 seconds, in which the competitor can start the number of tries needed.

The starting post is placed anywhere the participant prefers.
Time starts running when the competitor lets go of the starting post.
After time starts running, the starting post will be taken away.
Time stops at the moment when the participant rides off the board, dismounts, starts hopping or turns the tire more than 45 degrees.

There are no finals for the Stillstand competition.
The overall results will be determined by the best results per gender.

\subsection{700c Racing}

\begin{comment2016}
We believe that these rules are already included in track racing, and that this section can be removed.
\end{comment2016}


Races of any length and type can also be conducted in a 700c wheel category.
\begin{itemize}
\item Maximum bead seat diameter (BSD): 622 mm.
\item If these races are intended to exclude 24 Class wheels, sizes must be greater than 618 mm.
\item No restrictions on crank length.
\item Beyond these, 700c unicycles must comply with all other requirements for racing unicycles.
\item The host may choose age groups.
\end{itemize}

\subsection{Unlimited Track Racing}

An unlimited race is one in which there are no unicycle size restrictions.
Any size wheels, any length crank arms, giraffes or any types of unicycles (see definition in chapter \ref{chap:general_definitions}) are allowed.
All other Track racing rules apply.

\subsection{Juggling Unicycle Race}

The traditional distance is 50 meters.
Riders use the 5 meter line from the One-Foot Race, and must be juggling when they cross this line.
Three or more non-bouncing objects must be used.
If an object is dropped (hits the ground) or the juggling pattern is otherwise stopped, the rider is disqualified.
Two balls stopping in one hand during a 3-ball cascade is defined as stopping.
Riders who start by juggling four or more objects may drop one, as long as their pattern continues, unbroken, into three.
The juggling pattern must be `in control' when the rider crosses the finish line.
`Control' is determined by the Referee.

\subsection{Ultimate Wheel Race}

An ultimate wheel is a unicycle with no frame or seat.
The traditional distance is 10m for 0-10 riders, and 30m for 11-UP riders.
Maximum wheel size is 618 mm (24 Class) for all ages, with 125 mm minimum crank arm length or 250 mm between pedal holes.
The host may allow other limitations, or none, if these details are announced well in advance.

\subsection{50m Fast Backward}

Riders must face and pedal backward.
The Starter lines up the rear of the tire above the start line.
Helmets are mandatory.
Timing is stopped when the rear of the tire crosses the finish line.

\subsection{Medley}

This is a race involving riding several different ways of riding.

\textbf{Example:} Forward 25m, seat in front 25m, one foot 25m, hopping 10m, with 5m transition areas.
Rules are set by the host.
Remounting is allowed.

