\chapter{Competitor Rules}

\section{Safety}

\oldrule{8.3}
All riders must wear a helmet, shin guards, and shoes as defined in chapter \ref{chap:general_definitions}.
Gloves and knee protection are recommended.

\section{Unicycles}

\oldrule{8.14}
Any unicycle may be used. 
There is no restriction on changing unicycles during the competition.

\section{Rider Identification}

The rider number must be visible on the rider or unicycle.

\section{Protests}

\begin{comment-2016}
The below rule doesn't really make sense.  When results are posted is too late to protest something that happened on a particular line during the competition.  If a protest is against something that happened on a line, it seems like it should be addressed while the competition is going on.  I would think a more likely protest would be that your final score does not match the number of lines you know you did, so your card must be re-checked.
\end{comment-2016}

\oldrule{8.12}
A protest can be lodged by anyone against an Line Judge's ruling. 
Protests typically arise when a bystander (another rider, or a spectator) observes a rider make an infraction that is not recorded by the Line Judge, or when an Line Judge gives the wrong penalty.

Protests must be lodged with the event director within fifteen minutes of the official results being posted. 
Protests must be in writing, and must note the rider, and section number and a description of the protest.

\section{Event Flow}

\subsection{Rider Responsibility}

\oldrule{8.11}
The rider is responsible for knowing where a section starts and ends, and which route he or she is supposed to take.

\oldrule{8.6}
If there is a lineup for a section, the rider must go to the end of the line after each attempt.

If two or more riders are on overlapping sections at one time, the rider who started first has the right-of-way.


\subsection{Score Card}
\oldrule{8.11}
The rider is responsible for his or her scorecard. 
If it becomes damaged, the rider can ask the Event Director for a new one. 
If it becomes lost, the rider will be issued a new card but their score will return to zero.

\subsection{Scoring Points}
\oldrule{8.7}
Each section is worth one point, and the objective is to score points by successfully riding (``cleaning'') as many sections as possible within the specified time period.

\subsection{Definition Of ``Cleaning''}
\oldrule{8.7.1}
Cleaning a section is defined as follows:

\begin{enumerate}
\item \textbf{Riding into a section.} This is defined as the moment a rider's tire crosses over the start line.
\item \textbf{Riding through the section without ``dabbing''.} Dabbing is defined as follows:
	\begin{enumerate}[a.]
	\item Allowing any part of the rider's body to touch the ground or obstacle. 
	If loose clothing brushes against the ground or obstacle but does not influence the rider's balance, then this is acceptable (does not constitute a dab).
	\item Allowing any part of the cycle except the tire, rim, spokes, crank arms, pedals,or bearing caps to touch the ground or obstacle.
	\item Riding or hopping outside the boundaries of the defined section.
	The unicycle must be within the boundaries of the section at all times, even if the rider is in the air (for example, a rider cannot hop over a section boundary that turns a corner, even if they land back inside the section).
	\item Breaking the flagging tape or other markers that are delineating a section boundary. 
	Touching or stretching the tape does not constitute a dab, as long as the unicycle remains inside the section boundary.
	\item Riding a section in any way that is not consistent with the instructions outlined for that problem.
	\end{enumerate}
\item \textbf{Exiting the section.} A rider exits a section when their wheel fully cross over the finish line, or are within a defined finish area (such as a taped circle on top of a boulder). 
There is no requirement to exit in control.
If a rider falls across the defined finish line but manages to exit without dabbing, they have cleaned the section. %comment-2016 really?
\end{enumerate}

\oldrule{8.7.2}
When hooking a pedal on an obstacle, it is acceptable for a rider's heel and/or toe to initially contact the ground, as long as most of the rider's foot is still on the pedal. 
However, after a rider is established in position, weighting the heel or toe on the ground constitutes a dab.

It is acceptable for a rider's body to be entirely on one side of the centerline of the unicycle.

\subsection{Multiple Attempts}
\oldrule{8.7.2}
Riders may attempt any problem multiple times until they succeed or decide to abandon the section.
However, it is not possible to earn additional points by cleaning a section more than once, and no points are awarded if the rider does not clean the entire section.

\subsection{Time Limit}

\oldrule{8.4}
All riders must stop riding at the end of the time limit. 
If a rider is mid-way through an attempt when the time limit is reached, they are allowed to finish that attempt.

\oldrule{8.11}
The rider must gauge their time. 
No allowance will be made for riders who spend too much time at one obstacle and cannot complete the course before the end of the competition time period.

\subsection{Prohibited Activities}
\oldrule{8.11}
No rider may attempt any obstacle prior to the start of the competition. 

Intentional modification of a section by riders or spectators is prohibited. 
Note that kicking objects to test stability does not constitute intentional modification if an object moves. 
If a section is unintentionally modified or broken by a rider, they should inform the Event Director or Course Setter who will return the obstacle to its original form if possible. %comment-2016 is course setter defined elsewhere or only used here?
