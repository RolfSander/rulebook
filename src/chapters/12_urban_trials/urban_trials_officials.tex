\chapter{Judges and Officials Rules}

\section{Trials Officials}

\begin{framed}
We think that the officials for each event need to be specified in event chapter, not in chapter 1. For this event, we have taken the approach that the rules will specify the scope of responsibility for the official.
\end{framed}

\subsection{Trials Director}

\textbf{The Trials Director is the head organizer and administrator of trials
events.  With the Convention Host, the Trials Director determines the course,
obtains permis, interfaces with the community, and determines the system
used to run the event.  The Trials Director is responsible for the logistics and
equipment for all trials events.  
Trials Director is in charge of keeping events running on schedule, and
answers all questions not pertaining to rules and judging.  The Trials
Director is the highest authority on everything to do with the trials
events, except for decisions on rules and results.}

\subsection{Chief Judge}

\textbf{The Chief Judge is the head Trials official, whose primary job is to make sure
the competitors follow the rules.  The Chief Judge makes all final decisions
regarding rule infractions. The Chief Judge is responsible for resolving
protests.}

\subsection{Observer}

\begin{framed}
Observer should probably be changed to Line Judge.
\end{framed}

\oldrule{8.8}
The observers are responsible for judging whether a rider has successfully cleaned a section.

\section{Safety}

\oldrule{8.3}
If an Observer or the Trials Director feels the safety is compromised by a rider attempting a section that is beyond their ability, they may prohibit the rider from attempting that obstacle.
In cases where a fall from an obstacle could be particularly dangerous, the Trials Director may also permit attempts only by highly skilled riders who believe they will qualify for the Finals.

\section{Scoring Methods}

\subsection{Method 1}
\oldrule{8.9.1}
Method 1 is mandatory for all major competitions and is the recommended method for all other competitions.

Each rider is issued a scorecard (see example) at the beginning of the competition, and must give their card to an Observer prior to attempting a section.
If the competition is self-judged, the rider attempting the section gives their card to another rider who must observe them attempt the section.
If they clean the section, the observer indicates that they have completed the section by initialing or punching the box corresponding to that section. 
At the end of the competition, riders hand in their cards to the Trials Director or to a designated person for tallying of scores.

\textbf{Example scorecard:}

\begin{tabular}{|l|l|l|}
\hline 
\textbf{Rider Name:} & \textbf{Category:} &  \\ 
\hline 
Section Number  & Section Number  & Section Number \\ 
\hline 
1 & 6 & 11 \\ 
\hline 
2 & 7 & 12 \\ 
\hline 
3 & 8 & 13 \\ 
\hline 
4 & 9 & 14 \\ 
\hline 
5 & 10 & 15 \\ 
\hline 
\end{tabular}

\subsection{Method 2} 
\oldrule{8.9.3}
This method is intended for small events, and is not appropriate for larger events. 
Major events such as Unicon or national meets must not use this system of scoring.

In this method, one or two observers keep track of scores for numbered sections on a computer or paper spreadsheet such as this:

\begin{tabular}{|c|c|c|c|c|c|c|c|c|c|c|c|c|c|c|c|}
\hline 
 & \textbf{Section:} & & &  & &  &  &  &  &  & & & &  &   \\ 
\hline 
\textbf{Rider} & \textbf{Category} & 1 & 2 & 3 & 4 & 5 & 6 & 7 & 8 & 9 & 10 & 11 & 12 & 13 & 14 \\ 
\hline 
Jane Smith & Expert &  &  &  &  &  &  &  &  &  &  &  &  &  & \\ 
\hline 
John Anderson & Sport &  &  &  &  &  &  &  & &  &  &  &  &  &  \\ 
\hline 
 &  &  &  &  &  &  &  &  &  &  &  &  &  &  & \\ 
\hline 
 &  &  &  &  &  &  &  &  &  &  &  &  &  &  & \\ 
\hline 
\end{tabular} 

\section{Protests and Disputes}

\begin{framed}
This section makes no sense.
\end{framed}

\oldrule{8.12}
The jury will base its ruling on the input from the relevant parties, including the rider, the Observer, and the person who lodged the protest.
In the evaluation of protests, the benefit of the doubt must go to the Observer. 
The Jury is not obliged to overrule the Observer based on testimony from witnesses. 
Only if all parties present at the incident agree on the facts, and the Observer accepts that he or she made an error in assigning penalties, can an Observer's decision be overturned.

\oldrule{8.12}
For small-scale events, the Trials Director can act as the sole jury member. 
For larger events there should be a Jury consisting of at least three members, and they should be appointed in advance of the event. 
The Jury should be composed of the Trials Director, the head Observer or Event Commissar if applicable, and a riders' representative. 
If there is no head Observer, the Trials Director can appoint any person with experience in trials. 
Care should be taken to avoid conflict of interest and, in the event that a protest involves someone close to a Jury member, that person should be replaced for evaluation of the protest in question.

