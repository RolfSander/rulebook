\chapter{Event Organizer Rules}

\section{Venue}

\oldrule{5.13}
It will be a multi-lap event featuring a bit of cross country trail, grassy fields and natural and man-made obstacles where dismounting will be necessary.
A course should have no fewer than two and no more than six obstacle or barrier sections where riders normally dismount and run with the unicycle.
The starting and finishing stretches shall be free of obstacles within 10 meters.

It is suggested that the length of the course (used by both classes) not be much shorter than 1 km in length and no longer than 2.5 km in length. 
Organizers should keep in mind that most of the course should be visible from several vantage points.

\section{Officials}

\begin{framed}
We only specified officials that are mentioned in the rules.
\end{framed}

\textbf{The host must designate the following officials for each cyclocross race:
\begin{itemize}
\item Cyclocross Director
\item Referee
\end{itemize}}

\section{Communication}

\begin{framed}
Some thoughts on what might need to be communicated:
\begin{itemize}
\item age groups
\item results
\end{itemize}
}
\end{framed}

\textbf{The cyclocross event is exempt from the rule about early publication of course details. This is because the cyclocross course is typically set up immediately before the race.}

\oldrule{1.4.2}
\textit{Details of all non-track racing events, or other events with unique courses or details must be published as soon as they are known. 
This is to provide competitors with the information they need to train, and to help them prepare the appropriate unicycles.
These are major needs for attendees from far away. 
Necessary details depend on the event, but include things like course length, elevation and elevation change, steepness, level of terrain difficulty, amount of turns, riding surfaces, course width, etc.
Maps should be provided when possible. 
While sometimes courses cannot be planned until weeks or days before the convention, as soon as they are known the details must be posted to the convention web site and/or all places where convention information is posted. 
It is acceptable to publish tentative courses while waiting for permits to be approved, etc.}

\section{Age Groups}

\begin{framed}
As with muni, is ``maximum'' really what is intended.  Also, this is straight from the muni chapter, and may not apply to cyclocross in its entirety.
\end{framed}

\oldrule{5.6}
Age groups must be offered as male and female age group.
There must not be any age group specific restrictions on equipment.
The following age groups are the maximum allowable for muni competitions:

\begin{tabular}{|l|l|}
\hline
Under 15 & Youth \\
\hline
15-16 & Juniors \\
\hline
17-18 & Rookies \\
\hline
19-29 & Elites \\
\hline
30-49 & Masters \\
\hline
50+ & Veterans \\
\hline
\end{tabular}

\section{Course Availability for Practice}

\textbf{The cyclocross course must be available for practice at least one hour immediately prior to a cyclocross event. This will ensure that course setters do not have an advantage over other riders, and may compete.}

\oldrule{1.4.3}
\textit{If the course is open for practice to all riders for at least 7 days leading up to the event, then there are no restrictions on who can compete. 
If the course is not open for practice until the day of the event, then anyone who has pre-ridden the course is not allowed to compete. 
Organizers must therefore ensure that course marking and set-up are done by non-competing staff/volunteers.|

\section{Race Configuration}

\oldrule{5.14}
It is advised that Cyclocross be run as two separate races, (Unlimited and Standard) as the nature of a multi-lap event on a short course will lead to passing and lapping.

\oldrule{5.16}
It is suggested that the Unlimited race be close to 45 minutes in length and the Standard race be close to 30 minutes in length.
Using the time from the top rider's first two laps, the referee will determine how many laps could be completed in the desired time limit (i.e. 45 minutes).
From this point on, the number of remaining laps (for the leaders) will be displayed and this will be used to determine when finish of the race occurs. A bell will be rung with one lap to go.

\oldrule{5.18}
Lapped riders in the race will all finish on the same lap as the leader and will be placed according to the number of laps they are down and then their position at the finish.
