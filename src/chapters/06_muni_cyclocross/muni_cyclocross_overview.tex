\chapter{Overview \label{chap:cyclocross}}

\section{Definition}

\textbf{Cyclocross is a multi-lap event featuring cross country trail, grassy
fields, pavement, and natural and man-made obstacles where dismounting will
be necessary. A typical lap length is between 1 km and 2.5 km. All riders
race for approximately the same length of time, with faster riders
completing more laps than slower riders. }

\oldrule{5.12}
\textit{Cyclocross racing is one of the fastest growing segments of bicycle racing in the United States.
The course typically includes grassy fields, pavement, and possibly dirt trails.
The courses are multi-lap and very spectator friendly.
The typical bike of choice is a drop bar bike with skinny knobby tires.
Cyclocross differs from the usual bike races in that there are mandatory dismounts sections, called run-ups, where racers must carry their bikes over barriers, up stair-sets, or other obstacles that are too big to ride over.
The riders run through these sections and remount on the other side.
The speed in which a rider can transition from riding to running and back to riding is very important.}

\section{Rider Summary}

\textbf{This section is intended as an overview of the rules, but does not
substitute for the actual rules.
\begin{itemize}
\item You must wear shoes, knee pads, gloves, and helmet.
\item Cyclocross has wheel size and gearing requirements that you need to 
be aware of.
\item Be aware of the rules regarding passing, dismounts, illegal riding, and protests.
\end{itemize}}