\chapter{Competitor Rules}

\begin{framed}
I removed the sentence that Cyclocross follows the muni rules. Any relevant rules have been copied to this section.
\end{framed}

\oldrule{5b}
\textit{All rules in the general muni chapter apply unless changed below.}

\section{Safety}
\oldrule{5.3}
\textit{For all muni events, r}Riders must wear shoes, kneepads, gloves/wrist-guards and a helmet (see definition in chapter \ref{chap:general_definitions}).

\oldrule{1.22}
\textit{
Riders must have kneepads, gloves and shoes that meet the definitions below, and helmets for certain events.}

\textit{The IUF allows no exceptions to this for muni events. Additional equipment such as shin, elbow or ankle protection are optional.}

\section{Unicycles}

\oldrule{5.14}
For Cyclocross, Unlimited is defined as any standard unicycle (see definition in chapter \ref{chap:general_definitions}) with a rim with a bead seat diameter (BSD) of 622 mm (700c) or larger.
Unlimited also includes a unicycle with a rim smaller than a 622 mm BSD {\em only} if it has a functioning gearing system which will yield a virtual wheel size greater than a 622 mm BSD.
The Standard category only includes standard unicycles without gearing and with a rim less than 622 mm BSD.
There are no restrictions on crank length.
For example, the following wheel sizes generally fit the above definitions: \\
Unlimited: 28/29/36, geared 24/26/28/29/36 inch \\
Standard: 24/26 inch

\oldrule{1.22}
\textit{Riders must use unicycles that conform to the definitions and dimensions for racing unicycles.}

\section{Rider Identification}

\textbf{Riders must wear their race number clearly visible on their chest so that it is visible during the race and as the rider crosses the finish line.  Additionally, the rider may be required to wear a chip for electronic timing.}

\section{Protests}

\textbf{Protests must be filed on an official form within two hours of the posting of event results. Every effort will be made for all protests to be handled within 30 minutes from the time they are received.}

\oldrule{1.11}
\textit{An official protest/correction form must be available to riders at all times. All
protests against any results must be submitted in writing on the proper form within
two hours after the results are posted, unless there is a shorter time specified for certain
events (for example: track racing). The form must be filled in completely. This time
may be extended for riders who have to be in other races/events during that time period.
Every effort will be made for all protests to be handled within 30 minutes from the time
they are received. Mistakes in paperwork and interference from other riders or other
sources are all grounds for protests. Protests handed in after awards have been delivered
will not be considered if the results have been posted for at least three hours before the
awards. If awards are delivered before results are posted, it is recommended to announce
the schedule of posting and the deadline for protests at the awarding ceremonies. All
Chief Judge or Referee decisions are final, and cannot be protested.}

\section{Event Flow}

\subsection{Starting}

\oldrule{5.15}
There will be a Le Mans style start.
Unicycles will be lined up in a designated area away from the riders near the lap/finish line.
Riders will line up behind an additional line and then be required to run to retrieve their unicycle when the race starts.
They will then need to mount their unicycle to ride.
Riders must be mounted within 10 meters after crossing the lap/finish line.

\subsection{Passing}

\oldrule{5.18}
In the case of a rider being lapped, the passing rider has the right-of-way.
The approaching rider needs to alert the slower rider of their intentions to pass.
Special care at international events should be taken due to language differences.

\subsection{Dismounts}

\oldrule{5.17}
Upon dismounting there are no restrictions about passing riders.
Dismounted riders may run with their unicycle.
Courtesy is expected to avoid accidents, but the running unicyclist does not have to yield to riding unicyclist.

\subsection{Illegal Riding}

\oldrule{5.19}
Riders cannot cut the course around the obstacles.
They may ride through the obstacle section if possible or dismount and run with their unicycle.
By definition, the majority of riders should not be able to ride or hop the obstacle section.
Riding or hopping through the obstacle section should not damage or break the obstacle.

\oldrule{5.20}
Unicyclists must attempt to ride at least 50\% of the course on each lap.
This is to avoid someone running the entire race carrying or pushing a unicycle without riding it. 
A racer in violation will be warned by a racing official.
Failure to heed the warning will result in a disqualification.

\subsection{Finishes}

\oldrule{5.8}
Riders must also ride completely across the finish line\textit{, as described in section \ref{sec:track-field_finishes}}. 

\oldrule{2.6}
Riders must cross the line mounted and in control of the unicycle.
``Control'' is defined by the rearmost part of the wheel crossing completely over the finish line with the rider having both feet on the pedals.
In the event of a dismount at the finish line the rider must back up, remount and ride across the finish line again.
