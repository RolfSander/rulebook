\chapter{Competitor Rules}

\textbf{I removed the sentence that Cyclocross follows the muni rules. Any relevant rules should be copied to this section, but I didn't see many that applied.}

\section{Safety}
Riders must wear shoes, kneepads, gloves and a helmet (see definitions in chapter \ref{chap:general_definitions}).

\textbf{Personal music systems are disallowed.}

\section{Unicycles}

For Cyclocross, Unlimited is defined as any standard unicycle (see definition in chapter \ref{chap:general_definitions}) with a rim with a bead seat diameter (BSD) of 622mm (700c) or larger.
Unlimited also includes a unicycle with a rim smaller than a 622mm BSD {\em only} if it has a functioning gearing system which will yield a virtual wheel size greater than a 622mm BSD.
The Standard category only includes standard unicycles without gearing and with a rim less than 622mm BSD.
There are no restrictions on crank length.
For example, the following wheel sizes generally fit the above definitions: \\
Unlimited: 28/29/36, geared 24/26/28/29/36 inch \\
Standard: 24/26 inch

\section{Rider Identification}

\textbf{Something probably needs to be said about rider number or the wearing of a chip.  This seems to be missing from the existing rules.}

\section{Event Flow}

\subsection{Riders Must Be Ready}
Riders must be ready when called for their races.
Riders not at the start line when their race begins may lose their chance to participate.
The Starter will decide when to stop waiting, remembering to consider language barriers, and the fact that some riders may be slow because they are helping run the convention.

\subsection{Starting}

There will be a Le Mans style start.
Unicycles will be lined up in a designated area away from the riders near the lap/finish line.
Riders will line up behind an additional line and then be required to run to retrieve their unicycle when the race starts.
They will then need to mount their unicycle to ride.
Riders must be mounted within 10 meters after crossing the lap/finish line.

\subsection{False Starts}

\textbf{Are there any rules for false starts?}

\subsection{Passing}

In the case of a rider being lapped, the passing rider has the right-of-way.
The approaching rider needs to alert the slower rider of their intentions to pass.
Special care at international events should be taken due to language differences.
Lapped riders in the race will all finish on the same lap as the leader and will be placed according to the number of laps they are down and then their position at the finish.

\subsection{Dismounts}

Upon dismounting there are no restrictions about passing riders.
Dismounted riders may run with their unicycle.
Courtesy is expected to avoid accidents, but the running unicyclist does not have to yield to riding unicyclist.

\subsection{Illegal Riding}

Riders cannot cut the course around the obstacles.
They may ride through the obstacle section if possible or dismount and run with their unicycle.
By definition, the majority of riders should not be able to ride or hop the obstacle section.
Riding or hopping through the obstacle section should not damage or break the obstacle.

Unicyclists must attempt to ride at least 50\% of the course on each lap.
This is to avoid someone running the entire race carrying or pushing a unicycle without riding it. 
A racer in violation will be warned by a racing official.
Failure to heed the warning will result in a disqualification.

\subsection{Finishes}

Riders must ride completely across the finish line in control, as described in section \ref{sec:track-field_finishes}.
