\chapter{Competitor Rules}

\section{Safety}
Riders must wear shoes, knee pads, gloves/wrist-guards and a helmet (see definitions in chapter \ref{chap:general_definitions}).

\section{Unicycles and Categories}

Separating categories depends on the course.
If the course is not favoring any type of unicycle, no separate categories should be made.
If the course is favoring big wheels and geared unicycles, the recommended categories are:
\begin{itemize}
\item \textbf{Limited}: Ungeared unicycles up to and including 29 Class wheels. No restrictions on cranks arm lengths or pedal types.

\item \textbf{Unlimited}: Ungeared unicycles greater than 29 Class and geared unicycles. No restrictions on cranks arm lengths or pedal types.
\end{itemize}

\section{Rider Identification}

Riders must wear their race number clearly visible on their chest so that it is visible during the race and as the rider crosses the finish line.
Additionally, the rider may be required to wear a chip for electronic timing.

\section{Protests}

Protests must be filed on an official form within two hours of the posting of event results.
Every effort will be made for all protests to be handled within 30 minutes from the time they are received.

\section{Event Flow}

\subsection{Starting}

There will be a Le Mans style start.
Unicycles will be lined up in a designated area away from the riders near the lap/finish line.
Riders will line up behind an additional line and then be required to run to retrieve their unicycle when the race starts.
They will then need to mount their unicycle to ride.
Riders must be mounted within 10 meters after crossing the lap/finish line.

\subsection{Passing}

In the case of a rider being lapped, the passing rider has the right-of-way.
The approaching rider needs to alert the slower rider of their intentions to pass.
Special care at international events should be taken due to language differences.

\subsection{Dismounts}

Upon dismounting there are no restrictions about passing riders.
Dismounted riders may run with their unicycle.
Courtesy is expected to avoid accidents, but the running unicyclist does not have to yield to riding unicyclist.

\subsection{Illegal Riding}

Riders cannot cut the course around the obstacles.
They may ride through the obstacle section if possible or dismount and run with their unicycle.
By definition, the majority of riders should not be able to ride or hop the obstacle section.
Riding or hopping through the obstacle section should not damage or break the obstacle.

Unicyclists must attempt to ride at least 50\% of the course on each lap.
This is to avoid someone running the entire race carrying or pushing a unicycle without riding it.
A racer in violation will be warned by a racing official.
Failure to heed the warning will result in a disqualification.

\subsection{Finishes}

Riders must ride mounted completely across the finish line.
They may not run across the finish line.

Riders must cross the line mounted and in control of the unicycle.
``Control'' is defined by the rearmost part of the wheel crossing completely over the finish line with the rider having both feet on the pedals.
In the event of a dismount at the finish line the rider must back up, remount and ride across the finish line again.
