\chapter{Competitor Rules}

\section{Safety}

Riders must wear shoes, no additional safety equipment is needed.

\section{Unicycles}

Standard unicycles only (see definitions in chapter \ref{chap:general_definitions}), though any number can be used.
Unicycles with metal pedals and marking tires are allowed, so these competitions are generally intended for outdoors.

\section{Rider Identification}

No rider identification is required.

\section{Protests}

Protests must be filed on an official form within 15 minutes of the posting of event results.
Protest is only possible against mistakes in calculation or other mistakes not connected to a judge's subjective score.
The Chief Judge must resolve all protests within 30 minutes of receipt of the written form.

\begin{comment-2016}
There probably needs to be some text about how protests apply to the preliminary round/battle format of the competition.  The following rule also applies to protests, but may not be accurate.
\end{comment-2016}

\section{Results}
Final results will be continuously announced and/or posted for public view.
Results Sheets will be posted after each age category of an event.
The protest period begins at this point.

\section{Music}

In Flatland, a DJ plays music for the competition.
Competitors may optionally bring their own music but is not judged.
The DJ has the right to not play the request song.
Competitors who bring music must provide it in a form that is supported by the DJ.

\section{Costume and Props}
Clothing has no influence on the score.
Riders are encouraged to dress in the uniform of their national teams or clubs, or in clothing that represents their teams, groups or countries.
No props allowed, other than what is included in the performing area.

\section{Event Flow}

\subsection{Riders Must Be Ready}

Riders who are not ready at their scheduled competition time may or may not be allowed to perform after the last competitor in their age group.

\subsection{Competition Format}
Riders perform a one minute preliminary run and the top riders continue on to tournament-style Battle finals.

\subsection{Preliminary Round}

The preliminary round will last one minute.
No tricks after time is called will be counted.
If a rider is in a combo when their time ends, they may end the trick they are performing but are not allowed to go into another trick.
After the time has ended, the rider has 2 attempts to perform a last trick.

\subsection{Battle-style Overview}
In a Flatland battle, two riders compete head-to-head, taking turns performing lines of tricks.
The winner of each battle is determined immediately following the battle by the judges.
The winner continues to the next battle and the loser is eliminated.

At the conclusion of each battle, the rider(s) will have 3 attempts to perform a last trick.

\subsection{Number Of Competitors Entering Battles}
The final battles will consist of up to the 16 highest-scoring riders.
To decide on the number, the judges will vote.
A simple majority is needed to decide whether 4, 8, or 16 riders will advance.
However, a number other than 4, 8, or 16 may be chosen if the judges unanimously agree that a different number would be more conducive to the goal of producing the most exciting battles for riders and spectators.
In this case, byes would be used for this group to fit the next largest bracket (for example, 11 riders would use the 16 rider bracket, and the top 5 riders would get a bye for the first round of battles).

\subsubsection{Battle Assignments}
Battles will proceed according to the following brackets, depending on whether 4, 8, or 16 riders advance. Due to time constraints the losers bracket may be disregarded at the host's discretion. At Unicon, the full bracket must be used in the Jr. Expert and Expert competitions.

http://www.printyourbrackets.com/pdfbrackets/4teamDouble.pdf \\
http://www.printyourbrackets.com/pdfbrackets/8teamDouble.pdf \\
http://www.printyourbrackets.com/pdfbrackets/16teamdouble.pdf

\subsection{Battle Finals}
Each battle will last two minutes, except for the final 4 battles.
These semifinal and final battles will last three minutes, unless another duration between 2 and 4 minutes is agreed upon by both riders.
The rider with the better ranking from the preliminary round must choose if he or she wishes to start the battle or go second (and may ask the other rider for a preference).
There will be two countdown timers, one for each rider, and each of these will be set to one minute.
Each timer will be started and stopped when each rider starts and stops.
Riders should aim to complete each turn in about 15 seconds.
No tricks after time is called will be counted.
If a rider is in a combo when their time ends, they may end the trick they are performing but are not allowed to go into another trick.
After one rider's time runs out, the other rider will ride for the rest of their time and then both will proceed to Last Trick.
The rider who started the battle will also go first for the last trick.
The riders must alternate between attempts until they complete the trick or use up all attempts.
