\chapter{2013 Flatland \& Street \label{chap:flat-street} Overview}

\section{Difference Between These Events}
In Flatland, riders perform in a flat area with no obstacles or props.
There is no judging of music and costume, and the emphasis is on originality and creativity.
Street is about using the environment, such as ramps, rails, stairs and platforms, to do tricks in the style of skateboarding and.
Riders are judged on the skill and creativity of moves and combinations they do.

\textbf{Note:} These rules apply to both Flatland and Street unless otherwise noted.

\section{Deadline For Signing Up}
These events have a deadline for entry, which must be specified in the registration form.
If not specified in the registration form, the deadline is one month before the official convention start date.
A maximum of ten Individuals for each event will be allowed to be added after this time to account for difficulties in travel planning or other valid reasons that are communicated about in advance.
These will be added in the order of their request to the Chief Judge and Convention Director via email or fax.
Participants who attempt to sign up less than 36 hours prior to the beginning of the specified competition will not be allowed to enter.

\section{Safety Gear}
\textbf{For Flatland} riders must wear shoes, no additional safety equipment is needed.
The Chief Judge will remove from the competition any riders with dangerously loose shoelaces. 

\textbf{For Street} riders must wear shoes, shinpads and helmet.
The Chief Judge will remove from the competition any riders not wearing the minimum required safety gear and or has dangerously loose shoelaces.

\section{Unicycles}
Standard unicycles only (see definition in chapter \ref{chap:general_definitions}), though any number can be used.
Unicycles with metal pedals and marking tires are allowed, so these competitions are generally intended for outdoors.

\section{Size Of Performing Areas}
Hosts must publicize the dimensions of the available performing area as far in advance of the competition as possible, and organizers of international championships at least three months prior to the event. See section \ref{sec:flat-street_flatland-performing-area} (for Flatland) and section \ref{sec:flat-street_street-performing-area} (for Street) for details on performing areas.

\section{Riders Must Be Ready}
Riders who are not ready at their scheduled performance time may or may not be allowed to perform after the last competitor in their age group.
The Chief Judge will remember to consider language barriers, and that riders may be engaged in convention work to slow them down.
A rider may not perform before a different set of judges than those that judged the rest of their age group.

\section{Interruption Of Judging}
An interruption of judging can result from material damage, injury or sudden illness of a competitor, or interference with a competitor by a person or object.
If this happens, the Chief Judge determines the amount of time left and whether any damage may be the fault of the competitor.
Re-admittance into competition must happen within the regulatory competition time.
If a routine is continued and the competitor was not at fault for the interruption, all devaluations coming forth from the interruption will be withdrawn.

\section{Music, Costume and Props}
In Flatland, competitors may optionally bring their own music but is not judged.
For Street Comp, background music will be played.

\subsection{Media Types}
The host is required to have the capability of playing recordable CDs.
Other media types may also be supported, at the host's discretion.
The Artistic Director is responsible for announcing what media types will be supported, and making sure the necessary equipment is provided.

\subsection{Music Preparation}
Competitors who bring music must provide it in a form that is supported, and has been announced by the Artistic Director.
All music must be clearly labeled with the competitor name, age group and the track number.
Whenever possible, competition music should be the first track on the CD.
The DJ (music operator) is not responsible for any errors resulting from unsupported types or mislabeled tracks.

\subsection{Music Volume}
Volume level is controlled by the DJ, at instructions from the Chief Judge.
The base volume should be loud enough to sound clear, and be heard by all.
Some music may start with especially loud or quiet sections, and the DJ should be advised of these so volume levels do not get compensated in the wrong direction.
Some competitors may request that their music be played at lower levels.
These requests can be made directly to the DJ.
Requests for higher volumes must be approved by the Chief Judge, who has the option of passing this responsibility to the DJ.

\subsection{Costume and Props}
Clothing has no influence on the score.
Riders are encouraged to dress in the uniform of their national teams or clubs, or in clothing that represents their teams, groups or countries.
No props allowed, other than what is included in the performing area.

\section{Announcing Of Results}
Final results will be continuously announced and/or posted for public view.
Results Sheets will be posted after each age category of an event.
The protest period begins at this point.

\section{Protests}
Must be filed in writing, within 15 minutes from the posting of event results.
Protest against judges' scores is not permissible.
Protest is only possible against calculation mistakes or other mistakes not connected to the scoring.
The Chief Judge must resolve all protests within 30 minutes from receipt of the written form.

\section{Judging Panel}
For Flatland, there must always be an odd number of judges.
For Street, there are three judges  per section for the preliminary rounds, and five judges for the finals.

\subsection{Selecting Judges}
A person should not judge an event if he or she is:
\begin{enumerate}
\item A parent, child or sibling of a rider competing in the event.
\item An individual or team coach, manager, trainer, colleague who is member of the same club specified in the registration form, colleague's family etc. of a rider competing in the event.
\item More than one judge from the same family judging the same event at the same time.
If the judging pool is too limited by the above criteria, restrictions can be eliminated starting from the bottom of the list and working upward as necessary only until enough judges are available.
If there are some candidates who have the same level of restrictions and judging score, their agreement about publishing the results need to be considered.
The eliminations must be agreed upon by the Chief Judge and Artistic Director, or next-highest ranking artistic official if the Chief Judge and Artistic Director are the same person.
\end{enumerate}

\subsection{Judging Panel May Not Change}
The individual members of the judging panel must remain the same for entire age groups; for example one judge may not be replaced by another except between age groups.
In the event of a medical or other emergency, this rule can be waived by the Chief Judge.

\subsection{Rating Judge Performance}
Judges are rated by comparing their scores to those of other judges at previous competitions.
Characteristics of Judging Weaknesses:
\begin{itemize}
\item \textbf{Excessive Ties:}
A judge should be able to differentiate between competitors.
Though tying is most definitely acceptable, excessive use of tying defeats the purpose of judging.
\item \textbf{Group Bias:}
If a judge places members of a certain group or nation significantly different from the other judges.
This includes a judge placing members significantly higher or significantly lower (a judge may be harsher on his or her own group members) than the other judges.
\item\textbf{Inconsistent Placing:}
If a judge places a large number of riders significantly different from the average of the other judges.
\end{itemize}

\subsection{Judges Workshop}
The hosts of the convention must provide for a judge's workshop at least 24 hours prior to the start of the first competition.
A minimum of 3 hours must be set aside, in a classroom or similar environment.
If possible, it is strongly recommended to have more than one workshop to accommodate schedules.
Variations on this can be approved by the Chief Judge.
Workshop schedule(s) must be announced to all judges at least three weeks prior to the start of the competition.

Judges should have read the rules prior to the start of the workshop.
The workshop will include a practice judging session.
Each judge will be required to sign a statement indicating they have read the rules, attended the workshop, agree to follow the rules, and will accept being removed from the list of available judges if their judging accuracy scores show Judging Weaknesses.

\section{World Champions}
The male/female winner of the Expert category at Unicon is the Male/Female \textbf{World Champion}.
The male/female winner of the Junior Expert category at Unicon is the Male/Female \textbf{Junior World Champion}.
In the absence of any of these categories, no title will be awarded.
No title is awarded for the Advanced category.

\chapter{2013 Flatland}

\section{Flatland Overview \label{sec:flat-street_flatland-overview}}
A Flatland skill is any unicycle skill performed on a flat surface.
Flatland encourages riders to demonstrate a high level of technical difficulty and variety, as well as combinations and transitions between skills.

\section{Flatland Competition Performing Area \label{sec:flat-street_flatland-performing-area}}
A 11 x 14 meter area is required.
Judges will be situated along one of the 14m sides.
The audience may be as close to the boundaries as possible provided that they do not impede or obstruct the judges.

\section{Flatland Competition Riding Surface}
Paved, outdoor areas are ideal for Flatland, so the riding surface must be of a similar texture and riding quality.
The optimum surface is concrete or asphalt, outside.
While being indoors offers shelter and climate control, most indoor surfaces are not suitable so this should be avoided.
Keep in mind that Flatland unicycles have tires that may mark, and metal pedals. Indoor concrete is probably not suitable.

\section{Course Preparation}
Event organisers must specify the venue for the Flatland competition by the beginning of the convention/competition.

\section{Categories}
Male and female competitions should be offered in each of the following categories: Junior Expert (0-14), Expert (15+), and Advanced.
If there are less than 3 Junior Expert competitors they may choose whether to compete in Expert or Advanced.
If there are less than three females or less than three males overall, the male and female categories are merged.

\section{Competition Format}
Riders perform a one minute preliminary run and the top riders continue on to tournament-style Battle finals.

\section{Preliminary Round}
The preliminary round will last one minute.
No tricks after time is called will be counted.
If a rider is in a combo when their time ends, they may end the trick they are performing but are not allowed to go into another trick.

\section{Battle-style Overview}
In a Flatland battle, riders compete head-to-head in groups of two, taking turns performing lines of tricks.
The winner of each battle is determined immediately following the battle by the judges.
The winner continues to the next battle and the loser is eliminated.

\section{Number Of Competitors Entering Battles}
The final battles will consist of up to the 16 highest-scoring riders.
To decide on the number, the judges will vote.
A simple majority is needed to decide whether 4, 8, or 16 riders will advance.
However, a number other than 4, 8, or 16 may be chosen if the judges unanimously agree that a different number would be more conducive to the goal of producing the most exciting battles for riders and spectators.
In this case, byes would be used for this group to fit the next largest bracket (for example, 11 riders would use the 16 rider bracket, and the top 5 riders would get a bye for the first round of battles).

\subsection{Battle Assignments:}
Battles will proceed according to the following brackets, depending on whether 4, 8, or 16 riders advance. Due to time constraints the losers bracket may be disregarded at the host's discretion. At Unicon, the full bracket must be used in the Jr. Expert and Expert competitions.

http://www.printyourbrackets.com/pdfbrackets/4teamDouble.pdf \\
http://www.printyourbrackets.com/pdfbrackets/8teamDouble.pdf \\
http://www.printyourbrackets.com/pdfbrackets/16teamdouble.pdf

\section{Battle Finals}
Each battle will last two minutes, except for the final 4 battles.
These semifinal and final battles will last three minutes, unless another duration between 2 and 4 minutes is agreed upon by both riders.
There will be two countdown timers, one for each rider, and each of these will be set to one minute.
Each timer will be started and stopped when each rider starts and stops.
Riders should aim to complete each turn in about 15 seconds.
No tricks after time is called will be counted.
If a rider is in a combo when their time ends, they may end the trick they are performing but are not allowed to go into another trick.
After one rider's time runs out, the other rider will ride for the rest of their time and then both will proceed to Last Trick.

\section{Flatland Judging and Scoring \label{sec:flat-street_flatland-judging-scoring}}

\subsection{Number of Judges}
There must always be an odd number of judges to prevent ties. 

\subsection{Preliminaries}
Difficulty, consistency, variety, and last trick contribute to the total score.
Scoring: A total of 99 points is possible.
Higher numbers are better scores.
The judges will add up all scores for each competitor and rank then accordingly.
Rankings from individual judges are averaged to determine overall ranking.
The points are allocated as following: 

\textbf{Variety:}

(Score of 1-30 is given:)

High scores are awarded to competitors who perform a wide range of tricks and combos.
Lots of repeated tricks or similar tricks will receive low scores.

\textbf{Consistency/Flow:}

(Score of 1-30 is given:)

Fewer falls relative to number of landed skills results in higher score.
Higher points are rewarded to skills completed smoothly with minimal corrective hops or drastic movements to regain balance.

\textbf{Difficulty:}

(Score of 1-30 is given:)

High scores are given for technical, difficult tricks and combos, if they are completed successfully.
If a rider completes part of a combo and then falls, they are awarded points for everything they did up until the fall. 

\textbf{Last Trick:}

(Score of 0-9 is given:)

The last trick demonstrates how strong the rider is, physically and mentally, in the end.
The rider will have 3 attempts to perform a final trick.
Partial points may be given for a trick that is almost landed.
Only the last attempt will be scored, other failed attempts do not subtract from the score.
The rider is not obligated to try the same trick in every attempt.

\subsection{Battle \label{subsec:flat-street_flatland-judging-scoring_battle}}
Battles are judged using the same criteria as the preliminary round.
Judges must determine a winner individually, then the chief judge holds a vote to decide on the winner of that battle.
Judges are not required to write down scores for each category during battles.
If a rider repeatedly rides longer than their allowed time, distracting the audience and other rider, the judges may choose to eliminate that rider.

At the conclusion of each battle, the rider(s) will have 3 attempts to perform a last trick as described above.

\subsubsection{Finals/Semi-Finals:}
The two competitors who make it to the last battle compete for 1st and 2nd place in the Finals.
The two competitors who lose in the second round of battles will continue to the Semi-Finals where they will battle for 3rd and 4th place.

\subsection{Flatland Scoring}
In the preliminary round, raw scores from the judges are added to determine the placing of the riders.
The highest and lowest scores are removed.
If there are two riders with equal points in places 1–16, the rider with most points in ``last trick'' get an additional fraction of a point to break the tie.
The additional fraction of a point cannot result in that rider receiving a higher score than any previously higher-scoring rider.
If the riders' ``last trick'' scores are equal, they must show a last line and the judges must vote for the best, like later in battles.
Once place 1–8 is figured out, the battles can be configured like described in section \ref{subsec:flat-street_flatland-judging-scoring_battle}.

For battles, judges must decide on a single rider to vote on, they cannot tie the riders.
If a judge feels both riders performed equally based on their judging criteria, they must look at the ``last trick.''
The rider with the best score for ``last trick'' will be the winner.
