\chapter{Event Organizer Rules}

\section{Venue}

\subsection{Minimum Area \label{sec:flat-street_flatland-performing-area}}
A 11 x 14 meter area is required.
Judges will be situated along one of the longer sides.
The audience may be as close to the boundaries as desired provided that they do not impede or obstruct the judges.

\subsection{Riding Surface}
Paved, outdoor areas are ideal for Flatland.
The riding surface must be of a similar texture and riding quality throughout the competition area.
Note that Flatland unicycle have black tires and sometimes that may mark or damage indoor surfaces.
Although being indoors offer shelter and climate control, most indoor surfaces are not suitable.
Indoor concrete is not suitable as it is most often polished, hence too slippery.
Unicycles with metal pedals and marking tire are allowed.
The Event Host must organize the competition where marking tires and metal pedals are allowed.

\subsection{Postponement due to Weather}

In the case of rain or bad weather and an uncovered Flatland area, the organizers should postpone the event.
The competition should be canceled if it is considered dangerous for the riders.
If the event is postponed or moved to an indoor location, allowances must be the same (metal pedals, marking tires, etc.)
The event host should try to place events that may be influenced by weather conditions in the first days of the event, giving a larger period of time to reschedule it.

\subsection{Music}
In Flatland, a DJ plays music for the competition.

\section{Officials}

Flatland must have the following officials:
\begin{itemize}
\item Flatland Director
\item Chief Judge
\end{itemize}

The host must designate the Flatland Director well in advance of the event.
For an international events, it is recommended that the Flatland Director is chosen at least one year in advance so that they may be consulted on scheduling.
The Flatland Director must select the Chief Judge.
The Chief Judge may be the same person as the Flatland director.

\section{Communication}

Hosts must publicize details of the available performing area as far in advance of the competition as possible, and organizers of international championships at least three months prior to the event.
For other events, the organizers must specify the venue for the Flatland competition by the beginning of the convention/competition at the latest.

\section{Categories}
Male and female competitions should be offered in each of the following categories: Junior Expert (0-14), and Expert (15+).
The Advanced category is optional however it is not allowed at Unicon.
If there are less than 3 Junior Expert competitors, they may choose whether to compete in Expert or Advanced.
If there are less than three females or less than three males overall, the male and female categories may be merged.

\section{Practice}

There are no specific requirements for the competition area to be available for practice.
