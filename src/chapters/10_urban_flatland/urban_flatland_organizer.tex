\chapter{Event Organizer Rules}

\section{Venue}

\subsection{Minimum Area \label{sec:flat-street_flatland-performing-area}}
A 11 x 14 meter area is required.
Judges will be situated along one of the 14 meter sides.
The audience may be as close to the boundaries as possible provided that they do not impede or obstruct the judges.

\subsection{Riding Surface}
Paved, outdoor areas are ideal for Flatland, so the riding surface must be of a similar texture and riding quality.
The optimum surface is concrete or asphalt, outside.
While being indoors offers shelter and climate control, most indoor surfaces are not suitable so this should be avoided.
Keep in mind that Flatland unicycles have tires that may mark, and metal pedals. Indoor concrete is probably not suitable.

\subsection{Postponement due to Weather}

In the Flatland competition the organizers should postpone the events in the case of rain or bad weather.
These competitions should be canceled if considered dangerous for the riders.
If postponed or moved to an indoor location the organizers must try to keep the allowances the same as outdoors competitions (metal pedals allowed for example).
If originally on the competition schedule, these canceled competitions should be rescheduled during the convention duration.
The event host should try to place events that may be influenced by weather conditions in the first days of the event, giving a larger period of time to reschedule it.

\subsection{Music}
In Flatland, a DJ plays music for the competition.

\section{Officials}

The host must designate the following officials for flatland:
\begin{itemize}
\item Flatland Director
\item Chief Judge
\end{itemize}

\section{Communication}

Hosts must publicize the dimensions of the available performing area as far in advance of the competition as possible, and organizers of international championships at least three months prior to the event.
For other other events, the organizers must specify the venue for the Flatland competition by the beginning of the convention/competition at the latest.

\section{Categories}
Male and female competitions should be offered in each of the following categories: Junior Expert (0-14), and Expert (15+).
The Advanced category is optional however it is not allowed at Unicon.
If there are less than 3 Junior Expert competitors they may choose whether to compete in Expert or Advanced.
If there are less than three females or less than three males overall, the male and female categories may be merged.

\section{Practice}

There are no specific requirements for the competition area to be available for practice.

\section{Judges Workshop}
The hosts of the convention must provide for a judge's workshop at least 24 hours prior to the start of the first competition.
A minimum of 3 hours must be set aside, in a classroom or similar environment.
If possible, it is strongly recommended to have more than one workshop to accommodate schedules.
Variations on this can be approved by the Chief Judge.
Workshop schedule(s) must be announced to all judges at least three weeks prior to the start of the competition.
