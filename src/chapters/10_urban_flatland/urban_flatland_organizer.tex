\chapter{Event Organizer Rules}

\section{Venue}

\oldrule{1.16.2}
In the Trials and Street Comp, the organizers should postpone the events and exchange all the affected parts of the course for dry ones (replacing pallets for example).
These competitions should be canceled if considered dangerous for the riders.
If postponed or moved to an indoor location the organizers must try to keep the allowances the same as outdoors competitions (metal pedals allowed for example).
If originally on the competition schedule, these canceled competitions should be rescheduled during the convention duration.
The event host should try to place events that may be influenced by weather conditions in the first days of the event, giving a larger period of time to reschedule it.

\oldrule{7.5}
Hosts must publicize the dimensions of the available performing area as far in advance of the competition as possible, and organizers of international championships at least three months prior to the event.

\subsection{Flatland Competition Performing Area \label{sec:flat-street_flatland-performing-area}}
\oldrule{7.14}
A 11 x 14 meter area is required.
Judges will be situated along one of the 14 meter sides.
The audience may be as close to the boundaries as possible provided that they do not impede or obstruct the judges.

\subsection{Flatland Competition Riding Surface}
\oldrule{7.15}
Paved, outdoor areas are ideal for Flatland, so the riding surface must be of a similar texture and riding quality.
The optimum surface is concrete or asphalt, outside.
While being indoors offers shelter and climate control, most indoor surfaces are not suitable so this should be avoided.
Keep in mind that Flatland unicycles have tires that may mark, and metal pedals. Indoor concrete is probably not suitable.

\section{Officials}

\begin{framed}
This section should list officials for street.  That would seem to include:
\begin{itemize}
\item Chief Judge
\item Judges
\item DJ
\end{itemize}
\end{framed}

\section{Communication}

\oldrule{7.8.1}
The Artistic Director is responsible for announcing what media types will be supported, and making sure the necessary equipment is provided.

\subsection{Course Preparation}

\oldrule{7.16}
Event organizers must specify the venue for the Flatland competition by the beginning of the convention/competition.

\section{Categories}
\oldrule{7.17}
Male and female competitions should be offered in each of the following categories: Junior Expert (0-14), Expert (15+), and Advanced.
If there are less than 3 Junior Expert competitors they may choose whether to compete in Expert or Advanced.
If there are less than three females or less than three males overall, the male and female categories are merged.

\section{Practice}

\begin{framed}
Does the venue need to be available for practice before the competition? Are there rules regarding this?
\end{framed}

\section{Judges Workshop}
\oldrule{7.11.4}
The hosts of the convention must provide for a judge's workshop at least 24 hours prior to the start of the first competition.
A minimum of 3 hours must be set aside, in a classroom or similar environment.
If possible, it is strongly recommended to have more than one workshop to accommodate schedules.
Variations on this can be approved by the Chief Judge.
Workshop schedule(s) must be announced to all judges at least three weeks prior to the start of the competition.

\section{Registration Deadline}

\oldrule{7.2}
These events have a deadline for entry, which must be specified in the registration form.
If not specified in the registration form, the deadline is one month before the official convention start date.
A maximum of ten Individuals for each event will be allowed to be added after this time to account for difficulties in travel planning or other valid reasons that are communicated about in advance.
These will be added in the order of their request to the Chief Judge and Convention Director via email or fax.
Participants who attempt to sign up less than 36 hours prior to the beginning of the specified competition will not be allowed to enter.

\section{Music Volume}
\oldrule{7.8.3}
Volume level is controlled by the DJ, at instructions from the Chief Judge.
The base volume should be loud enough to sound clear, and be heard by all.
Some music may start with especially loud or quiet sections, and the DJ should be advised of these so volume levels do not get compensated in the wrong direction.
Some competitors may request that their music be played at lower levels.
These requests can be made directly to the DJ.
Requests for higher volumes must be approved by the Chief Judge, who has the option of passing this responsibility to the DJ.

\section{Announcing Of Results}
\oldrule{7.9}
Final results will be continuously announced and/or posted for public view.
Results Sheets will be posted after each age category of an event.
The protest period begins at this point.
