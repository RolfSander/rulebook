\chapter{Judges and Officials Rules}

\section{Flatland Director}

The Flatland Director is the head organizer and administrator of the flatland competition.
With the Convention Host, the Flatland Director determines the system used to run the event.
The Flatland Director is responsible for the logistics and equipment for all flatland events.
With the Chief Judge, the Flatland Director is in charge of keeping events running on schedule, and answers all questions not pertaining to rules and judging.
The Flatland Director is the highest authority on everything to do with the flatland competition, except for decisions on rules and results.

\section{Chief Judge}

The Chief Judge is the head flatland official, whose primary job is to make sure the rules are followed.
The Chief Judge oversees the competition, deals with protests, and answers all rules and judging questions.
The Chief Judge is responsible for seeing that all judges are trained and ready.
The Chief Judge is also responsible for the accuracy of all judging point tabulations and calculations.


\oldrule{7.6}%scott is this the right place for this?
The Chief Judge will remember to consider language barriers, and that riders may be engaged in convention work to slow them down.
A rider may not perform before a different set of judges than those that judged the rest of their age group.

\oldrule{7.7}%scott this too
An interruption of judging can result from material damage, injury or sudden illness of a competitor, or interference with a competitor by a person or object.
If this happens, the Chief Judge determines the amount of time left and whether any damage may be the fault of the competitor.
Re-admittance into competition must happen within the regulatory competition time.
If a routine is continued and the competitor was not at fault for the interruption, all devaluations coming forth from the interruption will be withdrawn.

\section{Judges}

\subsection{Judging Panel}

\oldrule{7.23.1}
There must always be an odd number of judges to prevent ties. 

\subsection{Selecting Judges}
\oldrule{7.11.1}
A person should not judge an event if he or she is:
\begin{enumerate}
\item A parent, child or sibling of a rider competing in the event.
\item An individual or team coach, manager, trainer, colleague who is member of the same club specified in the registration form, colleague's family etc. of a rider competing in the event.
\item More than one judge from the same family judging the same event at the same time.
\end{enumerate}

If the judging pool is too limited by the above criteria, restrictions can be eliminated starting from the bottom of the list and working upward as necessary only until enough judges are available.
If there are some candidates who have the same level of restrictions and judging score, their agreement about publishing the results need to be considered.
The eliminations must be agreed upon by the Chief Judge and Flatland Director, or next-highest ranking street official if the Chief Judge and Flatland Director are the same person.

\subsection{Judging Panel May Not Change}

\oldrule{7.11.2}
The individual members of the judging panel must remain the same for an entire category; for example one judge may not be replaced by another except between categories.
In the event of a medical or other emergency, this rule can be waived by the Chief Judge.

\subsubsection{Rating Judge Performance}
\oldrule{7.11.3}
Judges are rated by comparing their scores to those of other judges at previous competitions.
Characteristics of Judging Weaknesses:
\begin{itemize}
\item \textbf{Excessive Ties:}
A judge should be able to differentiate between competitors.
Though tying is most definitely acceptable, excessive use of tying defeats the purpose of judging.
\item \textbf{Group Bias:}
If a judge places members of a certain group or nation significantly different from the other judges.
This includes a judge placing members significantly higher or significantly lower (a judge may be harsher on his or her own group members) than the other judges.
\item\textbf{Inconsistent Placing:}
If a judge places a large number of riders significantly different from the average of the other judges.
\end{itemize}

\subsection{Training}
\oldrule{7.11.4}
Judges should have read the rules prior to the start of the workshop.
The workshop will include a practice judging session.
Each judge will be required to sign a statement indicating they have read the rules, attended the workshop, agree to follow the rules, and will accept being removed from the list of available judges if their judging accuracy scores show Judging Weaknesses.

\section{Flatland Judging and Scoring \label{sec:flat-street_flatland-judging-scoring}}

\subsection{Judging Criteria}

\oldrule{7.23.3}
Preliminary rounds and battles are judged using the following criteria:

\oldrule{7.23.2}
Difficulty, consistency, variety, and last trick contribute to the total score.
Scoring: A total of 99 points is possible.
Higher numbers are better scores.
The judges will add up all scores for each competitor and rank then accordingly.
Rankings from individual judges are averaged to determine overall ranking.
The points are allocated as following: 

\textbf{Variety:}

(Score of 1-30 is given:)

High scores are awarded to competitors who perform a wide range of tricks and combos.
Lots of repeated tricks or similar tricks will receive low scores.

\textbf{Consistency} \& Flow:

(Score of 1-30 is given:)

Fewer falls relative to number of landed skills results in higher score.
Higher points are rewarded to skills completed smoothly with minimal corrective hops or drastic movements to regain balance.

\textbf{Difficulty:}

(Score of 1-30 is given:)

High scores are given for technical, difficult tricks and combos, if they are completed successfully.
If a rider completes part of a combo and then falls, they are awarded points for everything they did up until the fall. 

\textbf{Last Trick:}

(Score of 0-9 is given:)

The last trick demonstrates how strong the rider is, physically and mentally, in the end.
The rider will have 3 attempts to perform a final trick.
Partial points may be given for a trick that is almost landed.
Only the last attempt will be scored, other failed attempts do not subtract from the score.
The rider is not obligated to try the same trick in every attempt.

\subsection{Battle Advancement \label{subsec:flat-street_flatland-judging-scoring_battle}}
\oldrule{7.23.3}
Judges must determine a winner individually, then the chief judge holds a vote to decide on the winner of that battle.

\subsection{Sportsmanship}
\oldrule{7.23.3}
If a rider distracts or delays other riders, judges, commentator, or shows unsportsmanlike conduct, the judges may choose to warn or eliminate that rider.
The chief judge may decide to name a battle manager to take care of these decisions.
The battle manager may be any one of the judges or the time keeper.

\subsection{Finals/Semi-Finals}
\oldrule{7.23.3.1}
The winner and loser of the final battle round take first and second place in the competition. The losers of the semi-final battle round compete in a final battle for third place.

\subsection{Preliminary Round Scoring}
\oldrule{7.23.4}
In the preliminary round, raw scores from the judges are added to determine the placing of the riders.
For each rider, the highest and lowest scores are removed. %comment-2016 still?
If there are two riders with equal points in places 1-16, the rider with most points in ``last trick'' get an additional fraction of a point to break the tie.
The additional fraction of a point cannot result in that rider receiving a higher score than any previously higher-scoring rider.
If the riders' ``last trick'' scores are equal, they must show a last line and the judges must vote for the best, like later in battles.

\subsection{Battle Scoring}
\oldrule{7.23.4}
For battles, judges must decide on a single rider to vote on, they cannot tie the riders.
If a judge feels both riders performed equally based on their judging criteria, they must look at the ``last trick.''
The rider with the best score for ``last trick'' will be the winner.

\oldrule{7.23.3}
Judges are not required to write down scores for each category during battles.
