\chapter{Overview}

\section{Preface}

Unicycle hockey is a variant of hockey which is played on unicycles with a tennis ball.
It is usually played in a gym.
These rules cannot cover every situation.
Teams have to agree on a specific amount of elbowroom before playing.
The different backgrounds of the players and the conditions of the location have to be considered.
Fairness of everyone involved is vital.

\section{Rider Summary}

This section is intended as an overview of the rules, but does not substitute for the actual rules.
\begin{itemize}
\item A player may only take part in a game when riding the unicycle.
 After falling off he or she has to mount at the same spot, but if necessary move out of the way of play first.
\item A player must not rest on the goal or the wall.
\item The game is non-contact in order not to endanger others.
  Only in the vicinity of the ball, the opponent's stick may be touched by the own stick.
  However, this contact may not be hard.
\item At the beginning and after each goal all players have to go to their own half.
  Then the game starts as soon as a player of the team in possession or the ball crosses the center line.
\item The player may touch the ball once with the flat hand (but not to score a goal directly).
\item The upper end of the stick must always be covered with one hand to avoid injuries of other players.
\item The blade of the stick must always be below the hips of all players in the vicinity.
\item A player who holds his stick in a way that someone else rides over it is committing a foul, regardless of intention.
\item A goal is disallowed if the ball was in one's own half when shot and wasn't touched by anyone afterwards (long shot).
\item The free shot is indirect, i.e.\ after the shot another player has to touch the ball.
\end{itemize}
