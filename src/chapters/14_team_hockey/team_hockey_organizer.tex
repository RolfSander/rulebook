\chapter{Event Organizer Rules}

\section{Venue}

Hockey should be play in a gym that is large enough to house the playing field.
\section{Officials}

The host must designate the following officials for each hockey tournament:
\begin{itemize}
\item Hockey Director
\item Board of Referees
\end{itemize}

\begin{comment2016}

\section{Communication}

\section{Age Groups}

\section{Practice}

\end{comment2016}

\section{Playing Field}

\subsection{Dimensions}


\setlength{\unitlength}{0.1mm}
\begin{picture}(1600,1000)

% field:
\thicklines
\put(850,475){\oval(1300,750)}

% 6.50m line:
\put(400,100){\line(0,1){30}}
\put(400,160){\line(0,1){30}}
\put(400,220){\line(0,1){30}}
\put(400,280){\line(0,1){30}}
\put(400,340){\line(0,1){30}}
\put(400,400){\line(0,1){30}}
\put(400,460){\line(0,1){30}}
\put(400,520){\line(0,1){30}}
\put(400,580){\line(0,1){30}}
\put(400,640){\line(0,1){30}}
\put(400,700){\line(0,1){30}}
\put(400,760){\line(0,1){30}}
\put(400,820){\line(0,1){30}}

% center line:
\put(850,100){\line(0,1){750}}

% 6.50m line:
\put(1300,100){\line(0,1){30}}
\put(1300,160){\line(0,1){30}}
\put(1300,220){\line(0,1){30}}
\put(1300,280){\line(0,1){30}}
\put(1300,340){\line(0,1){30}}
\put(1300,400){\line(0,1){30}}
\put(1300,460){\line(0,1){30}}
\put(1300,520){\line(0,1){30}}
\put(1300,580){\line(0,1){30}}
\put(1300,640){\line(0,1){30}}
\put(1300,700){\line(0,1){30}}
\put(1300,760){\line(0,1){30}}
\put(1300,820){\line(0,1){30}}

% breadth:
\put(100,400){\vector(0,-1){300}}
\put(100,550){\vector(0,1){300}}
\put(80,100){\line(1,0){40}}
\put(120,850){\line(-1,0){40}}

% length:
\put(650,950){\vector(-1,0){450}}
\put(1050,950){\vector(1,0){450}}
\put(1500,930){\line(0,1){40}}
\put(200,970){\line(0,-1){40}}

% 6.50m marks:
\put(400,475){\circle*{15}}
\put(1300,475){\circle*{15}}

% center mark:
\put(850,475){\circle*{15}}

% corner marks:
\put(300,200){\circle*{15}}
\put(300,750){\circle*{15}}
\put(1400,200){\circle*{15}}
\put(1400,750){\circle*{15}}

% goal line:
\put(300,400){\line(0,1){30}}
\put(300,460){\line(0,1){30}}
\put(300,520){\line(0,1){30}}
\put(250,400){\line(0,1){150}}
\put(300,400){\line(-1,0){50}}
\put(300,550){\line(-1,0){50}}

% goal line:
\put(1400,400){\line(0,1){30}}
\put(1400,460){\line(0,1){30}}
\put(1400,520){\line(0,1){30}}
\put(1450,400){\line(0,1){150}}
\put(1400,400){\line(1,0){50}}
\put(1400,550){\line(1,0){50}}

% distances:
\put(1300,650){\vector(1,0){100}}
\put(1400,650){\vector(-1,0){100}}
\put(1400,630){\line(0,1){40}}

% distances:
\put(1400,300){\vector(1,0){100}}
\put(1500,300){\vector(-1,0){100}}
\put(1400,280){\line(0,1){40}}

% text:
\put(900,650){\makebox(0,0)[l]{ center line }}
\put(900,650){\vector(-1,-1){50}}
\put(460,675){\makebox(0,0)[l]{ corner mark }}
\put(460,675){\vector(-2,1){150}}
\put(300,620){\makebox(0,0)[c]{ goal area }}
\put(460,530){\makebox(0,0)[l]{ 6.50~m mark }}
\put(460,530){\vector(-1,-1){50}}
\put(900,530){\makebox(0,0)[l]{ center mark }}
\put(900,530){\vector(-1,-1){50}}
\put(460,230){\shortstack{corners\\rounded or\\beveled}}
\put(450,230){\vector(-2,-1){200}}
\put(700,50){\makebox(0,0)[l]{ side line }}
\put(1250,400){\makebox(0,0)[r]{ goal line }}
\put(1250,400){\vector(2,1){150}}
\put(1250,300){\makebox(0,0)[r]{ 6.50~m line }}
\put(1250,300){\vector(1,-1){50}}
\put(1530,550){\shortstack[l]{ground\\line}}
\put(850,950){\makebox(0,0){length: 35 -- 45~m}}
\put(100,475){\makebox(0,0){\shortstack{breadth:\\20 -- 25~m}}}
\put(1300,700){\makebox(0,0)[l]{ 6.50~m}}
\put(1500,350){\makebox(0,0)[r]{2.50~m }}
\end{picture}

The field has a length of 35 to 45 meters and a breadth of 20 to 25 meters.
It is surrounded by barriers.
The corners are rounded or beveled.

\subsection{Goals}
The posts are 2.50 m in from the ends of the playing field (ground lines), ensuring that the players can go behind them.
The inside dimensions of goal openings are 1.20 m high and 1.80 m wide.
The goals must be made in such a way that the ball cannot enter through the rear or sides.
The goals must not have sharp, pointed or protruding parts.

\subsection{Markings}
The center line divides the field into two equal halves, and the center mark is in the middle of the center line.
There are marks in front of each goal at a distance of 6.5 m.
The goal lines connect the posts on the ground.
The corner marks are on the extension of the goal lines, 1.0 m in from each side line.
The 6.5 m lines are parallel to the goal lines and run through the 6.5 m marks.
The goal areas are between the 6.5 m lines and the ends of the field.

\section{Ball}
A ``dead'' tennis ball that reaches 30 percent to 50 percent of its original height after bouncing onto concrete is used.
Alternatively, a street hockey ball can be used.
The choice is made by the hosting organization if the opposing teams do not agree on which ball to use.
The chosen type of ball must be announced well in advance of the competition, and must be obtainable in all participating countries.

