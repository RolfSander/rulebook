\chapter{Event Organizer Rules}

\section{Venue}

\section{Officials}

The host must designate the following officials for Jump Events:
\begin{itemize}
\item Jump Director
\item Chief Judge
\item Judge
\end{itemize}

\begin{comment-2016}
\section{Communication}
\end{comment-2016}


\begin{comment-2016}%scott does this need to be solved before release?
\section{Age Groups}

Should this inherit the track age groups?
\end{comment-2016}

\section{Practice}

\oldrule{3.2}
The organizer should provide a place and equipment similar to those used for the official competition so the riders can practice before making their official attempts.
The equipment should be available during the whole length of the event, and even before if the organizer decides so.

For bigger events such as Unicon, national or continental events, the organizer must provide said equipment.

\section{Setup: High Jump}
\oldrule{3.5.1}
Around the High Jump apparatus a circle with a radius of 3 meters must be marked.
This circle is start and finish line.
The bar must be held loosely in the jumping apparatus so it can fall or break away if the rider does not complete the desired height.
Magnetic systems are not allowed.
The bar shall have a minimum diameter of 2cm and a maximum bending of 2 cm.
(The bar may sag no more than 20 mm at its lowest point.)
The bar must be sufficiently long such that minimum distance between the two apparatuses holding the bar is 2 meters.

\section{Setup: Platform High Jump}
\oldrule{3.6.1}
The structure consists of two parts: a platform and a landing surface.
The top surface of the platform must be at least 120 x 120 cm in size, the maximum and recommended size is 120 cm wide by 160 cm long.
The sides of the platform must be nearly perpendicular with the ground to ensure its presence does not hamper riders.
The landing surface consists of a flat piece of wood of the same dimensions as the platform (120 x 120 cm to 120 x 160 cm) that is firmly affixed to the top of the platform.
The method of attachment must be chosen so that it does not interfere with riders during jump attempts.
The structure should be sturdily built and shimmed so that there is minimal motion when jumps are attempted.
The front of the platform must be covered by a wooden plate that extends a minimum of 60cm from the top of the platform down.
This cover must be used when riders are jumping 60cm or higher.

A 3 meter safety ring must be marked out around all sides of the platform where no persons may enter during a jump attempt.
Organizers may choose to mark out a 4 m ring where only select persons may enter during a jump attempt (e.g. photographers or judges).
A 15 m runway must also be cleared of persons for riders that will roll into the jump.

\section{Setup: Long Jump}
\oldrule{3.7.1}
The riding area consists of a start line, a jump marker, a landing marker and a finish line beyond the jump marker.

The finishing line should be at least 4 meters from the landing marker but no more than 8 meters away.
We suggest that judges set up the finishing line 8 meters from the jump marker and move it further away if need during longer jumps.
Riders must ride or hop across the finish line for the attempt to count.
Successfully crossing the finish line is judged the same as in racing (see section \ref{sec:track-field_finishes}).
The start line must be a minimum of 25 meters in front of the jump marker to allow the riders to accelerate.
There must be an area behind the finishing line which is a minimum of 7 meters long and 2 meters wide as safety zone.
Riders may use all or part of the 25 meters between start line and jump marker.
Riders are also allowed to start from beside to be able to do accelerated side jumps.
Markers for takeoff and landing (jump marker and landing marker) must consist of a material which cannot be deformed in order to have the same conditions for all riders.
The markers must be at least 1.20 meter in width (across the runway), no more than 10 mm in height (above the runway), and no less than 5 centimeters in depth (front to back).
A Long Jump competition needs a minimum area of 40x2.5 meters.

\section{Setup: Long Jump on Platform}
\oldrule{3.8.1}
The riding area consists of a starting run-up of pallets which is between 7.5 and 8.5 meters long, 1 to 1.5 meters wide and 40 to 45 cm (3 pallets) high.
A landing platform of the same height and width as the run-up is required.
The landing platform should be approximately 4.5 meters long and fixed together so that they do not move when landed on.
It is recommended to cover the pallets with plywood or a similar material.
Behind the landing platform a half circle with 3 meter in diameter has to be marked.
If Euro pallets are used, the setup would be a run-up of 15 pallets stacked 5 long and 3 high and a landing platform of 9 pallets stacked 3 long and 3 high.
Thus a minimum of 25 Euro pallets would be needed.
A Long Jump on Platform competition needs a minimum area of 20x2 meters.
