\chapter{Overview \label{chap:road_racing}}

\begin{framed}
New text, not from the 2014 rulebook is in \textbf{boldface}. Text that we suggest be deleted is in \textit{italic}. Since the text has been reorganized, the original 2014 sections are marked with a boxed reference, such as \oldrule{3.4}. Meta-text, intended for the rulebook editors is boxed.
\end{framed}

\section{Definition}

\textbf{Road races are longer distance races held on paved roads or paved bike
paths. These rules specifically apply to 100k, Marathon, and 10k races, but
may also be applied to other road races, such as a Time Trial or a
Criterium.}

\oldrule{3.1}
\textit{These are races held usually on roadways or bike paths, generally for longer distances than our events on the track.
All riders may race together and be separated by age group afterward}

\oldrule{3.5}
\textit{Traditional road race distances have been:
\begin{itemize}
\item 10k with Unlimited and Standard 24$"$ classes, and
\item Full Marathon (42.195k) with Unlimited and possibly Standard 29$"$ classes.
\end{itemize}
However, any distances or wheelsize classes can be used for Road Races.}

\section{Rider Summary}

\textbf{This section is intended as an overview of the rules, but does not
substitute for the actual rules.
\begin{itemize}
\item You must wear shoes, knee pads, gloves, and helmet.
\item Personal music systems are not allowed for any races on public roads where there may be motorized traffic.
\item Water and food stations are at the discretion of the host.
\item Road racing events have wheel size, crank length, and gearing 
requirements that you need to be aware of.
\item You must be on time for the start of your race.
\item Be aware of the rules regarding passing, dismounts, and illegal riding.
\end{itemize}}