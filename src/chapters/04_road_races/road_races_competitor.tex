\chapter{Competitor Rules}

\section{Safety}
\oldrule{4.3}
Riders must wear shoes, gloves and a helmet (see definition in chapter \ref{chap:general_definitions}).
Knee pads and elbow pads are suggested optional safety gear.

\oldrule{4.2}
Personal music systems are not allowed for any races on public roads where there may be motorized traffic.

\oldrule{1.22}
\textit{Riders must have kneepads, gloves and shoes that meet the definitions below, and helmets for certain events.}

\oldrule{4.2}
\textit{Water/food stations are recommended.}

\textbf{Water and food are the responsibility of the rider. Hosts may offer food and water stations at their discretion.}

\section{Unicycles}

\begin{framed}
Scott needs to address how to include unlimited, here. The definitions in section 1D need to be updated, accordingly.

\end{framed}

\oldrule{4.6}
Only standard unicycles may be used.
Riders may use different unicycles for different racing events, as long as all comply with the rules for events in which they are entered.

\oldrule{4.2}
24$"$ and smaller wheels are not allowed for races longer than 20km without express permission of the racing director.

\oldrule{1.22}
\textit{Riders must use unicycles that conform to the definitions and dimensions for racing unicycles.}

\oldrule{4.6.1}
\textit{\begin{itemize}
\item For a ``Standard 700c Unicycle'' the rim may not have a bead seat diameter (BSD) larger than 622 mm (700c) and there is no minimum crank arm limit.
No gearing may be used.
\item For a ``Standard 24$"$ Unicycle'' the outside diameter of the tire may not be larger than 618 mm and crank arms may be no shorter than 125 mm.
No gearing may be used.
\item For a ``Standard 20$"$ Unicycle'' the outside diameter of the tire may not be larger than 518 mm and crank arms may be no shorter than 100 mm.
No gearing may be used.
\end{itemize}}

\textbf{The following chart defines the unicycle size limitations.
\begin{longtable}{|p{3cm}|p{5cm}|p{2cm}|p{2cm}|}
\hline
\textbf{Unicycle} & \textbf{Maximum wheel size} & \textbf{Minimum crank arm length} & \textbf{Gearing allowed?}\\
\hline
Standard 700c & rim bead seat diamter (BSD) 
no larger than 622mm & no limit & no \\
\hline
Standard 24$''$ & tire no larger than 618mm & 125mm & no \\
\hline
Standard 20$''$ & tire no larger than 518mm & 100mm & no \\
\hline
Unlimited & no limit & no limit & yes \\
\hline
\end{longtable}}

\section{Rider Identification}

\textbf{Riders must wear their race number clearly visible on their chest so that it is visible during the race and as the rider crosses the finish line.  Additionally, the rider may be required to wear a chip for electronic timing.}

\section{Protests}

\textbf{Protests must be filed on an official form within two hours of the posting of event results. Every effort will be made for all protests to be handled within 30 minutes from the time they are received.}

\oldrule{1.11}
\textit{An official protest/correction form must be available to riders at all times. All
protests against any results must be submitted in writing on the proper form within
two hours after the results are posted, unless there is a shorter time specified for certain
events (for example: track racing). The form must be filled in completely. This time
may be extended for riders who have to be in other races/events during that time period.
Every effort will be made for all protests to be handled within 30 minutes from the time
they are received. Mistakes in paperwork and interference from other riders or other
sources are all grounds for protests. Protests handed in after awards have been delivered
will not be considered if the results have been posted for at least three hours before the
awards. If awards are delivered before results are posted, it is recommended to announce
the schedule of posting and the deadline for protests at the awarding ceremonies. All
Chief Judge or Referee decisions are final, and cannot be protested.}

\section{Event Flow}

\subsection{Riders Must Be Ready}

\oldrule{4.9.1}
Riders must be ready when called for their races.
Riders not at the start line when their race begins may lose their chance to participate.
The Starter will decide when to stop waiting, remembering to consider language barriers, and the fact that some riders may be slow because they are helping run the convention.

\subsection{Starting}

\oldrule{4.9}
Riders start mounted, holding onto a starting post or other support.
The Starter will give a four-count start, for example, ``One, two, three, BANG!'' Alternatively, an electronic starter may be used.

\oldrule{4.9}
Riders start with the fronts of their tires (forwardmost part of wheel) behind the nearest edge of the starting line.
Rolling starts are not permitted in any road race.
However, riders may start from behind the starting line if they wish, provided all other starting rules are followed.
Riders may lean before the gun fires, but their wheels may not move forward before the gun fires.
Rolling back is allowed, but not forward.
Riders may place starting posts in the location most comfortable for them, as long as it doesn't interfere with other riders.

A rider's starting time is taken as when their heat begins (when the gun goes off) regardless of when they actually cross the starting line.

\subsection{False Starts}

\oldrule{4.10}
A false start occurs if a rider's wheel moves forward before the start signal, or if one or more riders are forced to dismount due to interference from another rider or other source.

\subsection{Passing}

\oldrule{4.11.1}
An overtaking rider must pass on the outside, unless there is enough room to safely pass on the inside.
Riders passing on the inside are responsible for any fouls that may take place as a result.
No physical contact between riders is allowed.
The slower rider must maintain a reasonably straight course, and not interfere with the faster rider.

\subsection{Dismounts}

\textit{In Road Racing} 

\oldrule{4.11.2}
Dismounting and remounting is allowed. 
If a rider is forced to dismount due to a fall by the rider immediately in front, it is considered part of the race, and both riders should remount and continue.

\subsection{Illegal Riding}

\textit{This includes}

\oldrule{4.11.3}
Illegal riding includes intentionally interfering in any way with another rider, deliberately crossing in front of another rider to prevent him or her from moving on, deliberately blocking another rider from passing, or distracting another rider with the intention of causing a dismount.

\subsection{Repair, Change, or Replace a (Broken) Unicycle}
\oldrule{4.12}
In Road Races, riders may make modifications to their unicycles, but must be self-sufficient in this.
The rider must carry all necessary parts and tools needed for the modification(s), and do all the work without any assistance.
For example, a rider may change cranks but must carry the new cranks and all tools from the start of the race.

Assistance is allowed in the event of a breakdown or damage to the unicycle.
Outside tools and hands-on help may assist the rider to continue, including replacing the unicycle if necessary.
The \textit{Chief} Referee must confirm that the situation was unplanned and was indeed ``accidental''.
If the \textit{Chief} Referee determines otherwise and the rider used outside assistance for changes to the unicycle, the rider will be disqualified.

\begin{framed}
In the above paragraph, we suggest removing ``Chief'', because there is no Chief Referee'' defined elsewhere.
\end{framed}

The rider may continue the course on foot (walking, not running) with the broken unicycle.
If the rider exits from the course, they must reenter the course \emph{at or before} the point where they exited from the course.
When the rider is off course, he may run or use any other form of transportation.

\begin{framed}
Let's not use italics for emphasis, if possible.  Its uses seem arbitrary.
\end{framed}

Any modifications made to the unicycle must still adhere to the requirements of the category that the rider is entered in.
For example, if a rider broke a crank in the Standard 24" 10k race, they are only allowed to install a new crank of 125 mm or longer.


\subsection{Finishes}

\oldrule{4.13}
Finish times are determined when the front of the tire first crosses the vertical plane of the nearest edge of the finish line. 

Riders are always timed by their wheels, not by outstretched bodies.
If riders do not cross the line in control, they are awarded a 5 second penalty to their time.
``Control'' is defined by the rearmost part of the wheel crossing completely over the vertical finish plane (as defined above) with the rider having both feet on the pedals.
(Note: a rider is not considered in control if the unicycle crosses the finish line independent of the rider.
The finish time is still measured by when the wheel crosses the vertical finish plane and the 5 second penalty is applied.)

In the case where a rider is finishing with a broken unicycle, the rider must bring at minimum the wheel to the finish line, and time is still taken when the wheel crosses the finish line.
The 5 second penalty is applied.
