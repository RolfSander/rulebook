\chapter{Judges and Officials Rules}

\section{Racing Officials}

\begin{framed}
Steve doesn't think we need to explicitly state that the Race Director and the Referee can be the same person.  These officials are roles, and it is always allowed for someone to have multiple roles.  If this needs to be explicitly mentioned, we should do it once in Chapter 1.
\end{framed}

\subsection{Race Director}

\oldrule{1.28}
\textit{\textbf{Race Director:} The Race Director is in charge of seeing that all equipment, forms, people, sound systems, and other requirements are taken care of before the convention starts.
Ideally, the Race Director is a member of the host organization, or has convenient access to the convention's locations.
The Race Director is responsible for the logistics, equipment, and scheduling for all racing events unless otherwise noted.}

\textbf{The Race Director is the head organizer and administrator of road race
events.  With the Convention Host, the Race Director determines the course,
obtains permis, interfaces with the community, and determines the system
used to run the event.  The Race Director is responsible for the logistics,
equipment for all road racing events.  With the Referee, the
Race Director is in charge of keeping events running on schedule, and
answers all questions not pertaining to rules and judging.  The Race
Director is the highest authority on everything to do with the road race
events, except for decisions on rules and results.}

\subsection{Referee}

\oldrule{1.28}
\textit{\textbf{Referee:} The Head racing official.
Makes all final decisions regarding race competitions.
Handles protests.
Makes sure racing areas and officials are trained and ready.
Works within the system set up by the Race Director for running the events.
The Referee should be someone very experienced in all aspects of unicycle racing, and must above all be objective and favor neither local, nor outside riders.
There can be separate Referees for different venues, or different categories of racing.}

\textbf{The Referee is the head racing official, whose primary job is to make sure
the competitors follow the rules.  The Referee makes all final decisions
regarding rule infractions. The Referee is responsible for resolving
protests.}

\subsection{Starter}

\oldrule{1.28}
The starter starts races; explains race rules; calls riders back in the event of false starts. The start also checks riders for correct unicycles and safety equipment.

\section{Officials Can Compete}

\textbf{The Referee may not compete.  The Race Director may compete, as long as the race course has been announced early enough that the Race Director does not have an advantage from knowledge of the course.}

\section{Consequences of Infractions}

\oldrule{4.3}
The Referee has final say on whether a rider's safety equipment is sufficient. 
The Starter will remove from the starting line-up any riders not properly equipped to race, including riders with dangerously loose shoelaces.

\oldrule{4.11.2}
\textit{[Regarding forced dismounts] The Referee can override this rule if intentional interference is observed.}

\begin{framed}
The above text is duplicated in the rule, below.
\end{framed}

\oldrule{4.11.3}
A rider who is forced to dismount due to interference by another rider may file a protest immediately at the end of the race.
Riders who intentionally interfere with other riders may receive from the Referee a warning, a loss of placement (given the next lower finishing place), disqualification from that race/event, or suspension from all races.
