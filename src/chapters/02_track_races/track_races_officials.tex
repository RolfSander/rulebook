\chapter{Judges and Officials Rules}

The Referee has final say on whether a rider’s safety equipment is sufficient.
The Starter will remove from the starting line-up any riders not properly equipped to race, including riders with dangerously loose shoelaces.

\section{Racing Officials}

\begin{comment2016}
All officials wording needs to be reviewed for consistency.
Some sections include wording about training, but others to no not.
\end{comment2016}

The Track Director is the head organizer and administrator of track events.
The Track Director is responsible for the logistics and equipment for all track events.
With the Referee, the Track Director is in charge of keeping events running on schedule, and answers all questions not pertaining to rules and judging.
The Track Director is the highest authority on everything to do with the track events, except for decisions on rules and results.

The Referee is the head track official, whose primary job is to make sure the competitors follow the rules.
The Referee makes all final decisions regarding rule infractions.
The Referee is responsible for resolving protests.
The Referee makes sure other track officials are trained and ready.

The Starter starts races; explains race rules; calls riders back in the event of false starts.
Also checks riders for correct unicycles and safety equipment.

The Finish Line Judge determines whether rides cross the finish line properly, according to the rules.

\section{Training Officials}
As the rules state, competitions cannot be started until all key track officials have been trained and understand their tasks.
For Racing, the Referee is in charge of making sure this happens.

\section{Starter Responsibilities\label{sec:track_starter}}

There should be about 3/4 second between each element in the count, with the same amount of time between each of them.
Starters should practice this before the races begin.
Timing of the count is very important for an accurate start.
This count can be in the local language, or a language agreed upon before competition starts.

Riders start mounted, holding onto a starting post or other support.
Unicycle riders need to be leaning forward before the starting gun fires, so the Starter will give a four-count start.
Example: ``One, two, three, BANG!''
This allows riders to predict the timing of the gun, for a fair start.

As an alternative a start-beep apparatus can be used.
In that case we have a six-count start.
Example: ``beep - beep -beep - beep - beep - buup!''
The timing between beeps is one second.
The first 5 beeps have all the same frequency.
The final tone (buup) has a slightly higher frequency, so that the racer can easily distinguish this tone from the rest.

If a heat has to be restarted, the Starter will immediately recall the riders, for example by firing a gun or blowing a whistle or other clear and predefined signal.
It is only the earliest false starting rider who gets assigned this false start and might get disqualified.

There are two options on how to deal with false starts:
\begin{itemize}
\item \textbf{One False Start Allowed Per Rider:}
In case of a false start, the heat is restarted.
Any rider(s) who caused their personal first false start may start again.
Any rider(s) causing their personal second false start are disqualified.
\item \textbf{One False Start Allowed Per Heat:} 
In case of a false start, the heat is restarted.
For the first false start of a particular heat, all riders may start again.
Thereafter, any rider(s) causing a false start are disqualified.
This option should not be used without an electronic false start monitoring system.
\end{itemize}

\section{Finish Line Judge Responsibilities}

\subsection{Judging Finish Line Dismounts}
One or more officials are required at the finish line to judge dismounts in all races where dismounting is allowed.
These officials must be appointed by the racing referee so they fully understand their crucial job.
The finish line judges are the voice of authority on whether riders must remount and cross the finish line again.
Any riders affected must be clearly and immediately signaled to return to a spot before the finish line, remount without overlapping the finish line, then ride across it again.
The path for backing up may involve going around any finish line timing or optical equipment to prevent data problems for other riders in the race.

\subsection{Timing Penalty For Finish Line Dismounts}
In electronically timed races, it's possible that no time will be recorded for the rider's successful finish.
Instead of recording an actual finish time, the rider's time will be recorded as 0.01 seconds faster than the next rider to cross the line after their remount and crossing.
If the rider in question is the last one on the track, the time recorded should be their actual time crossing the finish line after their remount.
