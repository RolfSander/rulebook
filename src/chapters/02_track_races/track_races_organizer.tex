\chapter{Event Organizer Rules}

\section{Venue}

\oldrule{1.15}
A track must be made available for conducting the track races.
The track must be marked in meters, and should be prepared in advance with start and finish lines for the various racing events that are unique to unicycle racing (such as 50, 30, 10 and 5 meter lines).
In addition to the track, a smooth area of sufficient size must be set aside to run the IUF Slalom.
A public address system must be provided to announce upcoming events and race winners.
Bullhorns are usually not adequate for the track environment.

\oldrule{1.16.1}
If the track is outdoors, plans must be made to deal with inclement weather.
Using an indoor track can eliminate this problem.
The track must be available for enough days to allow for inclement weather.

\section{Officials}

\textbf{The host must designate the following officials for track racing:
\begin{itemize}
\item Track Director
\item Referee
\item Starter
\item Finish Line Judges
\end{itemize}}

\section{Communication}

\begin{comment-2016}
Some thoughts on additional required communication:
\begin{itemize}
\item any changes to the standard rules
\item false start method
\item cut-off time
\item age groups
\item whether the event qualifies for a world record
\item whether helmets are required
\end{itemize}
\end{comment-2016}

\begin{comment-2016}
There should be a rule about when and how results are posted, so that the
protest period can begin.
\end{comment-2016}

\oldrule{2.3}
A Host is allowed to make helmets and/or knee pads mandatory for track races but it must be announced when registration is opened and must appear as an extra point to check for each discipline the competitor registers for.

\section{Age Groups \label{subsec:track-field_racing-categories_age-groups}}
\oldrule{2.1.2}
The following age groups are the minimum required by the IUF to be offered at the time of registration for any Track \& Field discipline: 0-10 (20$"$), 0-13, 14-18, 19-29, 30-UP.
Convention hosts are free to offer more age groups, and often do.
For example, a full range of offered age groups might look like 0-8 (20$"$), 9- 10 (20$"$), 0-12, 13-14, 15-16, 17-18, 19-29, 30-39, 40-49, 50-59, 60-UP.
All age groups must be offered as male and female age group.

\begin{comment-2016}
\section{Practice}

Does there need to be a rule to allow practice on the track?
\end{comment-2016}

\section{Minimum Racing Events \label{sec:track-field_minimum-racing-events}}
\oldrule{2.17}
The following races: 100m, 400m, 800m, One Foot, Wheel Walk, and IUF Slalom, are to be part of every Unicon.
Convention hosts are free to add more racing events.

\section{Track Combined Competition}%scott is this in the right place?
\oldrule{2.18}
The best finishers combined from the 6 racing events listed above will win this title.
Points are assigned for placement in each of the above races, based upon best times in the final heats or finishing age group times in the IUF Slalom.
1st place gets 8, 2nd place 5, 3rd place 3, 4th place 2, and 5th place 1.
Highest total points score is the World Champion; one each for male and female.
If there is a tie, the rider with the most first places wins.
If this still results in a tie, the title goes to the better finisher in the 100m race.
Points are not earned in age group heats.

\section{Race Configuration}

\oldrule{2.1.1}
Racing competition is held in two separate divisions: Male and Female.
No heat of any race shall be composed of both male and female riders without the approval of the Racing Referee.

\oldrule{2.10}
There will be no mixing of age groups, or sexes, in heats except with permission from the Racing Referee.

Track events must have both a preliminary and final round.

\section{Lane Assignments}

\oldrule{2.9}
At some conventions, lanes are preassigned at time of registration.
At other conventions, riders decide among themselves.
If riders disagree, the Clerk makes lane assignments.
In races where more than one heat is necessary per age group, every effort must be made to see that the fastest riders compete in the same heat.
If the track has undesirable lanes due to potholes or other problems, this should be considered when lanes are assigned.
A very bad or dangerous lane might not be used at all.
The Referee can override the Clerk's choice of lane assignments.
The general rule is that riders decide for themselves.

\section{Optional Race-End Cut-Off Time}
\oldrule{2.16}
It may be necessary to have a maximum time limit for long races, to keep events on schedule.
When this is planned in advance, it must be advertised as early as possible, so attending riders will know of the limit.
Additionally, at the discretion of the Racing Director, a race cut-off time may be set on the day of or during an event.
The purpose of this is to allow things to move on if all but a few slow racers are still on the course.
These cut-offs need not be announced in advance.
At the cut-off time, any racers who have not finished will be listed as incomplete (no time recorded, or same cut-off time recorded for all).
Optionally, if there is no more than one person on the course per age category and awards are at stake, they can be given the following place in the finishing order.
But if each participating age category has had finishers for all available awards (no awards at stake), there is no need to wait.
