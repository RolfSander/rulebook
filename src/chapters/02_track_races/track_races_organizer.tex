\chapter{Event Organizer Rules}

\section{Venue}

\oldrule{1.15}
A track must be made available for conducting the track races.
The track must be marked in meters, and should be prepared in advance with start and finish lines for the various racing events that are unique to unicycle racing (such as 50, 30, 10 and 5 meter lines).
In addition to the track, a smooth area of sufficient size must be set aside to run the IUF Slalom (and Slow Races, if held).
A public address system must be provided to announce upcoming events and race winners.
Bullhorns are usually not adequate for the track environment.

\oldrule{1.16.1}
If the track is outdoors, plans must be made to deal with inclement weather.
Using an indoor track can eliminate this problem.
The track must be available for enough days to allow for inclement weather.
\section{Officials}

\section{Communication}

\section{Age Groups \label{subsec:track-field_racing-categories_age-groups}}
\oldrule{2.1.2}
The following age groups are the minimum required by the IUF to be offered at the time of registration for any Track \& Field discipline: 0-10 (20$"$), 0-13, 14-18, 19-29, 30-UP.
Convention hosts are free to offer more age groups, and often do.
For example, a full range of offered age groups might look like 0-8 (20$"$), 9- 10 (20$"$), 0-12, 13-14, 15-16, 17-18, 19-29, 30-39, 40-49, 50-59, 60-UP.
All age groups must be offered as male and female age group.

\section{Practice}
\section{Race Configuration}

\subsection{Male/Female}
\oldrule{2.1.1}
Racing competition is held in two separate divisions: Male and Female.
No heat of any race shall be composed of both male and female riders without the approval of the Racing Referee.

\section{Race Configuration}

\oldrule{2.10}
There will be no mixing of age groups, or sexes, in heats except with permission from the Racing Referee.

\section{Starting Order}

\section{Lane Assignments}

\oldrule{2.9}
At some conventions, lanes are preassigned at time of registration.
At other conventions, riders decide among themselves.
If riders disagree, the Clerk makes lane assignments.
In races where more than one heat is necessary per age group, every effort must be made to see that the fastest riders compete in the same heat.
If the track has undesirable lanes due to potholes or other problems, this should be considered when lanes are assigned.
A very bad or dangerous lane might not be used at all.
The Referee can override the Clerk's choice of lane assignments.
The general rule is that riders decide for themselves.

\section{Starter}
\section{False Starts}

\oldrule{2.5}
There are two options on how to deal with false starts:
\begin{itemize}
\item \textbf{One False Start Allowed Per Rider:}
In case of a false start, the heat is restarted.
Any rider(s) who caused their personal first false start may start again.
Any rider(s) causing their personal second false start are disqualified.
\item \textbf{One False Start Allowed Per Heat:} 
In case of a false start, the heat is restarted.
For the first false start of a particular heat, all riders may start again.
Thereafter, any rider(s) causing a false start are disqualified.
This option should not be used without an electronic false start monitoring system.
\end{itemize}

\subsection{Finals}

\oldrule{2.1.4}
At Unicons, a `final' must be held for each of the following races: 100m, 400m, 800m, One Foot, Wheel Walk, and IUF Slalom. 
For any other Track \& Field discipline, a `final' may be held at the discretion of the organizer, after all age group competition for that discipline has been completed.

For disciplines that are run in heats, such as 100m races or relay races, this will take the form of a final heat. 
For disciplines that are not run in heats, such as IUF slalom or slow race, the final will take the form of successive attempts by the finalists.

The riders posting the best results regardless of age in the age group heats are entitled to compete in the final.
They can be called ``finalists''.
For each final, the number of finalists (finalist teams in case of relay) will be eight, unless for an event that uses lanes, the number of usable lanes is less than eight.
In that case the number of finalists equals the number of usable lanes.
Finals are composed regardless of age group, but male and female competitors are in separate finals.

Finals are subject to the same rules as age group competition, including false start rules and number of attempts.

The best result in a final determines the male or female Champion for that discipline (World Champion in the case of Unicon).

If a finalist disqualifies, gets a worse result, or doesn’t compete in the final, his/her result in age group competition will still stand.
The male and female winners of the finals will be considered the Champions for those disciplines, even if a different rider posted a better result in age group competition.
Speed records can be set in both age group competition and finals.

In disciplines for which no finals are held, finalist status will still be awarded on the basis of results in age group competition.
Accordingly, riders posting the best results in each discipline are the Champions for that discipline.

\section{Optional Race-End Cut-Off Time}
\oldrule{2.17}
It may be necessary to have a maximum time limit for long races, to keep events on schedule.
When this is planned in advance, it must be advertised as early as possible, so attending riders will know of the limit.
Additionally, at the discretion of the Racing Director, a race cut-off time may be set on the day of or during an event.
The purpose of this is to allow things to move on if all but a few slow racers are still on the course.
These cut-offs need not be announced in advance.
At the cut-off time, any racers who have not finished will be listed as incomplete (no time recorded, or same cut-off time recorded for all).
Optionally, if there is no more than one person on the course per age category and awards are at stake, they can be given the following place in the finishing order.
But if each participating age category has had finishers for all available awards (no awards at stake), there is no need to wait.

\section{Minimum Racing Events \label{sec:track-field_minimum-racing-events}}
\oldrule{2.18}
The following races: 100m, 400m, 800m, One Foot, Wheel Walk, and IUF Slalom, are to be part of every Unicon.
Convention hosts are free to add more racing events.

\section{Track Combined Competition}
\oldrule{2.19}
The best finishers combined from the 6 racing events listed above will win this title.
Points are assigned for placement in each of the above races, based upon best times in the final heats or finishing age group times in the IUF Slalom.
1st place gets 8, 2nd place 5, 3rd place 3, 4th place 2, and 5th place 1.
Highest total points score is the World Champion; one each for male and female.
If there is a tie, the rider with the most first places wins.
If this still results in a tie, the title goes to the better finisher in the 100m race.
Points are not earned in age group heats.

