\chapter{Judges and Officials Rules}

\section{Street Director}

The Street Director is the head organizer and administrator of the street competition.
With the Convention Host, the Street Director determines the system used to run the event.
The Street Director is responsible for the logistics and equipment for all street events.
With the Chief Judge, the Street Director is in charge of keeping events running on schedule, and answers all questions not pertaining to rules and judging.
The Street Director is the highest authority on everything to do with the street competition, except for decisions on rules and results.

\section{Chief Judge}

The Chief Judge is the head street official, whose primary job is to make sure the rules are followed.
The Chief Judge oversees the competition, deals with protests, and answers all rules and judging questions.
The Chief Judge is responsible for seeing that all judges are trained and ready.
The Chief Judge is also responsible for the accuracy of all judging point tabulations and calculations.

\oldrule{7.6}%comment-2016 is this the right place?
The Chief Judge will remember to consider language barriers, and that riders may be engaged in convention work to slow them down.
A rider may not perform before a different set of judges than those that judged the rest of their age group.

\oldrule{7.7}%comment-2016 is this the right place?
An interruption of judging can result from material damage, injury or sudden illness of a competitor, or interference with a competitor by a person or object.
If this happens, the Chief Judge determines the amount of time left and whether any damage may be the fault of the competitor.
Re-admittance into competition must happen within the regulatory competition time.
If a routine is continued and the competitor was not at fault for the interruption, all devaluations coming forth from the interruption will be withdrawn.

\section{Judges}

\subsection{Judging Panel}

\oldrule{7.11}
There are three judges  per section for the preliminary rounds, and five judges for the finals.

\subsection{Selecting Judges}
\oldrule{7.11.1}
A person should not judge an event if he or she is:
\begin{enumerate}
\item A parent, child or sibling of a rider competing in the event.
\item An individual or team coach, manager, trainer, colleague who is member of the same club specified in the registration form, colleague's family etc. of a rider competing in the event.
\item More than one judge from the same family judging the same event at the same time.
\end{enumerate}

If the judging pool is too limited by the above criteria, restrictions can be eliminated starting from the bottom of the list and working upward as necessary only until enough judges are available.
If there are some candidates who have the same level of restrictions and judging score, their agreement about publishing the results need to be considered.
The eliminations must be agreed upon by the Chief Judge and Street Director, or next-highest ranking street official if the Chief Judge and Street Director are the same person.

\subsection{Judging Panel May Not Change}

\oldrule{7.11.2}
The individual members of the judging panel must remain the same for an entire category; for example one judge may not be replaced by another except between categories.
In the event of a medical or other emergency, this rule can be waived by the Chief Judge.

\subsubsection{Rating Judge Performance}
\oldrule{7.11.3}
Judges are rated by comparing their scores to those of other judges at previous competitions.
Characteristics of Judging Weaknesses:
\begin{itemize}
\item \textbf{Excessive Ties:}
A judge should be able to differentiate between competitors.
Though tying is most definitely acceptable, excessive use of tying defeats the purpose of judging.
\item \textbf{Group Bias:}
If a judge places members of a certain group or nation significantly different from the other judges.
This includes a judge placing members significantly higher or significantly lower (a judge may be harsher on his or her own group members) than the other judges.
\item\textbf{Inconsistent Placing:}
If a judge places a large number of riders significantly different from the average of the other judges.
\end{itemize}

\subsection{Training}
\oldrule{7.11.4}
Judges should have read the rules prior to the start of the workshop.
The workshop will include a practice judging session.
Each judge will be required to sign a statement indicating they have read the rules, attended the workshop, agree to follow the rules, and will accept being removed from the list of available judges if their judging accuracy scores show Judging Weaknesses.

\subsection{Street Comp Judging}
\oldrule{7.31}
There will be three judges for each zone for the preliminaries, and five judges for each zone for the final.

The judges will have 5 minutes after each session to discuss the tricks.
They should use their preferred system for taking notes during competition (for example: they may take notes of tricks that were landed in that zone or assign point values).
It is recommended to ask riders about specific tricks (other riders should be present to justify the response), but neither the riders nor judges may discuss relative difficulty as it could influence the score.
After the judges have seen ALL of the riders at a single zone, they will rank the riders from best to worst.
Each judge is responsible for one set of rankings at their zone; they must judge all riders against each other even if they are in different groups.

For prelims, the riders will be given points according to their placement.
(So 1st place gets one point, 2nd place gets two points, etc.)
For finals, each place is awarded points as follows:

\begin{tabular}{|l|l|}
\hline
\textbf{1st}&  10 points\\
\hline
\textbf{2nd}& 7 points\\
\hline
\textbf{3rd}& 5 points\\
\hline
\textbf{4th}& 3 points\\
\hline
\textbf{5th}& 2 points\\
\hline
\textbf{6th}& 1 point\\ 
\hline
\textbf{7th and Beyond}& 0 points \\
\hline
\end{tabular}

The ranking should be influenced by the number of tricks done, and the difficulty of the tricks.
Consistency should not be considered, because it is inevitable that a consistent rider will land the most tricks.
However, note that the number of tricks
should also not always be the deciding factor on who wins.
Some one who performs 18 easy tricks should not be scored higher than someone who performs 3 outstanding tricks.
Once the judges assign places for every zone, the points will be added up and the results can be calculated.
After prelims, the riders with the lowest sum of placement points move on to finals.
In finals, the rider with the most points is the winner.
