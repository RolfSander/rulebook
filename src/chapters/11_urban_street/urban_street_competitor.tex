\chapter{Competitor Rules}

\section{Safety}

Riders must wear shoes and helmet.
Shin guards are recommended, but not mandatory.

\section{Unicycles}

Standard unicycles only (see definitions in chapter \ref{chap:general_definitions}), though any number can be used.
Unicycles with metal pedals and marking tires are allowed, so these competitions are generally intended for outdoors.

\section{Rider Identification}

\textbf{No rider identification is required.}

\section{Protests}

Protests must be filed on an official form within 15 minutes of the posting of event results.
Protest is only possible against mistakes in calculation or other mistakes not connected to a judge's subjective score.
The Chief Judge must resolve all protests within 30 minutes of receipt of the written form.

\begin{comment2016}
There probably needs to be some text about how protests apply to the preliminary and final format of the competition.
The following rule also applies to protests, but may not be accurate.
\end{comment2016}

\section{Results}
Final results will be continuously announced and/or posted for public view.
Results Sheets will be posted after each age category of an event.
The protest period begins at this point.

\section{Music}
For the Street comp, background music will be played.

\section{Costume and Props}
Clothing has no influence on the score.
Riders are encouraged to dress in the uniform of their national teams or clubs, or in clothing that represents their teams, groups or countries.
No props allowed, other than what is included in the performing area.

\section{Event Flow}

\subsection{Deadline For Entry}

This event has a deadline for entry, which must be specified in the registration form.
If not specified in the registration form, the deadline is one month before the official convention start date.

\subsection{Riders Must Be Ready}

Riders who are not ready at their scheduled competition time may or may not be allowed to perform after the last competitor in their age group.

\section{Preliminaries}
Riders will be put into groups of three or four (preferably 4, but in some cases, there may need to be up to 3 groups of 3 depending on the number of competitors).
Each group will be given a starting time, and they will proceed to their starting zone.
They will be given 6 goes per rider in each zone to perform as many tricks as possible.
(Depending on the possible time window the host can decide to reduce or expand the number of allowed goes but 4 is be the minimum number.)
The riders are assigned an order and they may only attempt a trick when it is their turn.
The order should be presented in writing as well as announced before the competition.
Riders may choose to skip their turn in the event of an injury or any other reason.
The group will then move on to the next zone.

\section{Finals}
The top 5 or 6 riders will be chosen to participate in the finals, which should be a few hours later, or the next day.
Finals should preferably not be before noon, because we want a lot of spectators, and we want to riders to have a chance to warm up and be ready to be at their best.
In the finals, the same 3 zones will be used, and all riders will go at the same time for 12 to 15 minutes in each zone.
(Depending on the possible time window the host can decide to reduce the time but 10 minutes is the minimum.)
The riders are assigned an order and they may only attempt a trick when it is their turn.
The order should be presented in writing as well as announced before the competition.
Riders may choose to skip their turn in the event of an injury or any other reason.
There will be 5 judges in the finals, and these can be made up from some of the judges of prelims, or even riders that did not make it into the finals.
