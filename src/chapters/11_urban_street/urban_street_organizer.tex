\chapter{Event Organizer Rules}

\section{Venue}

\oldrule{1.16.2}
In Street Comp, the organizers should postpone the events and exchange all the affected parts of the course for dry ones (replacing pallets for example).
These competitions should be canceled if considered dangerous for the riders.
If postponed or moved to an indoor location the organizers must try to keep the allowances the same as outdoors competitions (mmetal pedals allowed for example).
If originally on the competition schedule, these canceled competitions should be rescheduled during the convention duration.
The event host should try to place events that may be influenced by weather conditions in the first days of the event, giving a larger period of time to reschedule it.

\oldrule{7.5}
Hosts must publicize the dimensions of the available performing area as far in advance of the competition as possible, and organizers of international championships at least three months prior to the event. 

\subsection{Street Comp Performing Area \label{sec:flat-street_street-performing-area}}
\oldrule{7.25}
The Street course is to be composed of three ``zones''.
Each zone should have multiple obstacles, but each obstacle should encourage a specific type of skill.
The list below is an example of three typical things that can be used for the zones; however designers of the Street comp area should not limit themselves to the exact list.

\textbf{Zone 1:}
A ramp with a skate park rail in the middle, and a ledge on either side.
This zone will encourage technical grinds, without giving an advantage to a right of left footed grinder.

\textbf{Zone 2:}
Two different manny pads (a smooth platform of at least 3 m x 0.5 m and between 7 cm and 15 cm in height), one with two revs of length, and one with just one rev of length.
This will encourage the ability to perform technical flip tricks and other Street moves while having to set up quickly for the move down.

\textbf{Zone 3:}
A set of 5 stairs and a set of 7 stairs with a handrail in the middle of each (that are of a similar size to one that you would find in a city, not extremely steep).
This section would encourage the ability to perform bigger moves of all types.

Depending on the time limit of a host and physical limitations, elements described above can also be used in one bigger zone to give a rider more flexibility.
A host should be aware then that this needs 3 times as much time for the competition and the host should also be sure that judges are able to follow the rider in the entire big zone.
It is also possible to use a real street environment if that is possible.
This may result in having some different obstacles than specified above but it provides a 100\% real street atmosphere. 

Independent from the setup a host can go for, they should always take care to offer room for technical street, for grinds and for some big stuff.
He should also pay attention to offer enough room for the big wheelers to go for Big Street. 
The descriptions of the zones above should give a good idea about the requirements while offering a lot of room for being creative.

\subsection{Problems With Required Obstacles}
\oldrule{7.26}
The required obstacles must be built strong enough to endure many hours of heavy use.
They need to survive the competition without changing their shape or stability.
If one of the required obstacles is broken or made unusable during the competition, it must be repaired if one or more competitors say they need to use the damaged part.
If no competitors have a problem with the damage, no repair is necessary except for safety reasons, such as in the event of sharp exposed parts. 

\subsection{Minimum Age Groups}
\oldrule{7.28}
None.

\section{Officials}

\begin{framed}
This section should list officials for flatland.  That would seem to include:
\begin{itemize}
\item Chief Judge
\item Judges
\item DJ
\end{itemize}
\end{framed}

\section{Communication}

\oldrule{7.8.1}
The Artistic Director is responsible for announcing what media types will be supported, and making sure the necessary equipment is provided.

\section{Practice}

\oldrule{7.27}
Event organizers must arrange that the course for the Street competition is set up and available to be practiced on before competition.
With different time frames depending on the time frame and duration of the convention/competition.
At least 2 days prior to the day of the competition should apply to events that are over a weekend long (4 days and over of competition).
For events that last 1 - 3 days, practice time can be at the discretion of the competition organizers, for example, prior, but on the day of the competition, practice time must still be allowed.
If practicing on the competition grounds is not possible prior to the competition day, the organizers must build similar objects on another location for the riders to train on.

\section{Judges Workshop}
\oldrule{7.11.4}
The hosts of the convention must provide for a judge's workshop at least 24 hours prior to the start of the first competition.
A minimum of 3 hours must be set aside, in a classroom or similar environment.
If possible, it is strongly recommended to have more than one workshop to accommodate schedules.
Variations on this can be approved by the Chief Judge.
Workshop schedule(s) must be announced to all judges at least three weeks prior to the start of the competition.

\section{Registration Deadline}

\oldrule{7.2}
These events have a deadline for entry, which must be specified in the registration form.
If not specified in the registration form, the deadline is one month before the official convention start date.
A maximum of ten Individuals for each event will be allowed to be added after this time to account for difficulties in travel planning or other valid reasons that are communicated about in advance.
These will be added in the order of their request to the Chief Judge and Convention Director via email or fax.
Participants who attempt to sign up less than 36 hours prior to the beginning of the specified competition will not be allowed to enter.

\section{Music Volume}
\oldrule{7.8.3}
Volume level is controlled by the DJ, at instructions from the Chief Judge.
The base volume should be loud enough to sound clear, and be heard by all.
Some music may start with especially loud or quiet sections, and the DJ should be advised of these so volume levels do not get compensated in the wrong direction.
Some competitors may request that their music be played at lower levels.
These requests can be made directly to the DJ.
Requests for higher volumes must be approved by the Chief Judge, who has the option of passing this responsibility to the DJ.

\section{Announcing Of Results}
\oldrule{7.9}
Final results will be continuously announced and/or posted for public view.
Results Sheets will be posted after each age category of an event.
The protest period begins at this point.
