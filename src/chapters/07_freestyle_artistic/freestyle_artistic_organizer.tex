\chapter{Event Organizer Rules}

\section{Venue}

\oldrule{1.17}
Traditionally a gymnasium is used.
Artistic competitions are also possible in an auditorium, if the stage is large enough.
If this is done, a gym must also be available for practice, and possibly for group competition.
There must also be enough room for judges and spectators.
Seating must be provided for spectators, and a practice area must be provided for riders.
Ideally, this practice area would be in a separate gym.
The primary practice area cannot be outdoors, as wet or extreme weather would prevent riders from warming up and exchanging skills.
If necessary, the practice area can be behind a curtain in the competition gym, or behind the spectator seating.
Neither of these solutions is as desirable, due to the distraction that is unavoidably caused by riders using these areas.

The gym or riding surface must be marked with the boundaries of all riding areas for events in which they are required.
In some facilities black tires, metal pedals, untaped wooden hockey sticks, etc.
might not be allowed.
The host must make sure the participants are informed of this in advance.
All performing and practice areas must be in well-lit places that are protected from the weather, or have fallback locations in case the weather is bad.

It is very important that a good quality public address system be available for announcements, and to play competition music.
At least two music-playing devices must be provided (one as a backup or test machine).
These should be compatible with all the media types specified for the various events to be held there.

\subsection{Size Of Performing Areas}
\oldrule{6.17}
The minimum size for an Artistic Freestyle event must be 28 x 15 meters.
Hosts shall give additional space to riders.
Hosts must publicize the dimensions of the available performing area as far in advance of the competition as possible, and organizers of international championships at least three months prior to the event.

\section{Officials}

\begin{framed}
We only specified officials that are mentioned in the rules.
\end{framed}

\textbf{The host must designate the following officials for standard skill:
\begin{itemize}
\item Artistic Director
\item Chief Judge
\end{itemize}}

\section{Communication}

\begin{framed}
Some thoughts on additional required communication:
\begin{itemize}
\item age groups
\end{itemize}
\end{framed}

\subsection{Announcing Of Results}

\oldrule{6.8}
Final results will be continuously announced and/or posted for public view.
Results Sheets will be posted after each age category of an event.
The protest period begins at this point.

\section{Judges Workshop}
\oldrule{6.10.10}

The hosts of the convention must provide for a judge's workshop at least 24 hours prior to the start of the Artistic Freestyle competition.
A minimum of 3 hours must be set aside, in a classroom or similar environment.
If possible, it is strongly recommended to have more than one workshop to accommodate schedules.
Variations on this can be approved by the Chief Judge.
Workshop schedule(s) must be announced to all judges at least three weeks prior to the start of the competition.

\section{Age Groups and Categories}
\oldrule{6.2}
Age groups and categories may be different for different types of events.
The minimum allowable age groups and categories are listed for each event.
Convention hosts are free to add more age groups but additional categories can only be added when agreed upon by the Artistic Director, Chief Judge, and Event Host.
Categories may not be added or removed at a Unicon without approval by the IUF Board.
Age group is determined by the rider's age on the first day of the convention.
Junior Expert is open to all riders 0-14.
Expert is open to riders of any age, including 0-14.
Riders must state the category in which they are entering for each Freestyle event in which they participate.

\textbf{Example:} Riders who enter Individual Freestyle as Experts can enter Pairs in another category if they wish.
Riders are divided male/female in Standard Skill and Individual Freestyle, but not in Pairs or Group.

\subsection{Minimum Age Groups: Individual}
\oldrule{6.13.1}
 0-14, 15-UP, Expert.
The decision to enter as Expert or Jr. Expert is optional, but must be stated in advance.

\subsection{Minimum Age Groups: Pair}

\oldrule{6.14.1}
Age group (all ages), Expert.
Each rider may enter only once.
The age group of the older rider is the age group for the pair.
Expert is treated as the ``oldest'' age group, followed by Jr. Expert, and then all other age groups.
The decision to enter as Expert or Jr. Expert (if used) is optional, but must be stated in advance.

\subsection{Minimum Age Groups: Group}

\oldrule{6.15.3}
Small Group: 0-14, 15+
Large Group: none.

\section{Categories for Smaller Competitions}
\oldrule{6.4}
At competitions where the number of Artistic Freestyle competitors is low, the Event Host may choose to only offer categories and no age groups.
This decision would be made to encourage a competition that is fair and engaging for both spectators and competitors.

\section{Pre-Event Practice Time}
\oldrule{6.24}
In order to give fair practice time in the Freestyle competition venue to the high level competitors, thirty minutes for practice must be reserved immediately before each Jr. Expert and Expert competition.
During each thirty minute warm-up period, only the competitors for that event are allowed to be on the competition floor.

Each group that is competing also must be given time to rehearse on the competition floor.
The Artistic Director is responsible for publishing a rehearsal schedule at least two weeks before the competition day.
The amount of time allotted to each group is to be determined by the Artistic Director, however, each group must be given the same amount of rehearsal time and it cannot be less than fifteen minutes.

\section{Order Of Performance}
\oldrule{6.18}
Performance order for Jr. Expert and Expert in Pairs/Individual/Group Freestyle are defined by an open drawing without a computer.
The drawing/selection should be done publicly and transparently, at a time that is announced ahead of time so people can witness it.
The method to determine performance order for age groups is completely up to the Artistic Director.

\subsection{Media Types}
\oldrule{6.7.1}
The host is required to have the capability of playing recorded CDs.
Other media types may also be supported, at the host's discretion.
The Artistic Director is responsible for announcing what media types will be supported, and making sure the necessary equipment is provided.

\subsection{Music Volume}
\oldrule{6.7.3}
Volume level is controlled by the DJ, at instructions from the Chief Judge.
The base volume for Freestyle music should be loud enough to sound clear, and be heard by all.
For Standard Skill, volume level should not be loud enough to interfere with judge communication, but otherwise similar to the level for Artistic Freestyle.
Some competitors' music may start with especially loud or quiet sections, and the DJ should be advised of these so volume levels do not get compensated in the wrong direction.
Some competitors may request that their music be played at lower levels.
These requests can be made directly to the DJ.
Requests for higher volumes must be approved by the Chief Judge, who has the option of passing this responsibility to the DJ.

\subsection{Special Music Instructions}
\oldrule{6.7.4}
Some competitors may have special music instructions, such as stopping or starting the music at a visual cue, changing volume level during the performance, etc.
The DJ is not responsible for errors carrying out these instructions.
For best results, the competitor should supply a person to coach the DJ during the performance, so there are no mistakes.
If the DJ receives instructions that sound unusual, the Chief Judge should be consulted for approval.

\section{Limiting Competitors}

\subsection{Maximum Number of Competitors for Jr. Expert and Expert: Individual}
\oldrule{6.13.7}
\textbf{Non-Unicon:} Organizers of non-Unicon events can choose to limit the number of competitors using the guidelines below or have no limit.

\textbf{Unicon:} Each country can submit a maximum of three individuals in each category to compete at Unicon in the Individual Freestyle events (three in Jr. Expert Male, three in Jr. Expert Female, three in Expert Male, three in Expert Female).
If a country has placed 1st, 2nd, or 3rd in Individual Freestyle at the previous Unicon, they can submit one additional competitor for each placing in that category.
For example, if Country-A wins first place in Expert Male at the previous Unicon, they may submit up to four individuals for Expert Male at the current Unicon.
If Country-B wins second and third place in Jr. Expert Female at the previous Unicon, they may submit up to five individuals in Jr. Expert Female at the current Unicon.

\subsection{Method for Limiting the Competitors at Unicon: Individual}
\oldrule{6.13.8}
A country that wishes to submit more than their allocated number of individuals should select individuals by their own way.
Any type of competition using the IUF judging methods to determine their competitors is recommended.
If a country is unable to hold a competition, a country can choose individuals by their own rating method.
For example, if a country has placed 1st, 2nd or 3rd in Individual Freestyle at the previous Unicon, it can give these individuals a higher rating, because they brought additional number of individuals to a country.
If a country did not place in the top three, it can give only the highest placing individual a higher rating.
It is strongly recommended to complete the selection at least three months prior to the start of the Unicon.
If a country cannot select by then, the method and schedule of the selection must be communicated to the Chief Judge and Artistic Director at least three months prior to the start of the Unicon.

\subsection{Maximum Number of Competitors for Jr. Expert and Expert: Pair}
\oldrule{6.14.7}
\textbf{Non-Unicon:} Organizers of non-Unicon events can choose to limit the number of competitors using the guidelines below or have no limit.

\textbf{Unicon:} Each country can submit a maximum of three pairs in each category to compete at Unicon in the Pairs Freestyle events (three in Jr Expert Pairs, three in Expert Pairs).
If a country has placed 1st, 2nd, or 3rd in Pairs Freestyle at the previous Unicon, they can submit one additional competitor for each placing in that category.
For example, if Country-A wins first place in Expert Pairs at the previous Unicon, they may submit up to four Pairs for Expert Pairs at the current Unicon.
If Country-B wins second and third place in Jr Expert Pairs at the previous Unicon, they may submit up to five pairs in Jr Expert Pairs at the current Unicon.
If a pairs team is submitted consisting of members from two countries, that team must choose one of their two countries to represent.

\subsection{Method for Limiting the Competitors at Unicon: Pair}
\oldrule{6.14.8}
A country that wishes to submit more than their allocated number of pairs should select competitors by their own way.
Any type of competition using the IUF judging methods to determine their competitors is recommended.
If a country is unable to hold a competition, a country can choose pairs by their own rating method.
For example, if a country has placed 1st, 2nd, or 3rd in Pairs Freestyle at the previous Unicon, it can give these pairs a higher rating if BOTH partners from the previous Unicon still be pairs, because they brought additional number of pairs to a country.
If a country did not place in the top three, it can give only the highest placing pairs a higher rating.
It is strongly recommended to complete the selection at least three months prior to the start of the Unicon.
If a country cannot select by then, the method and schedule of the selection must be communicated to the Chief Judge and Artistic Director at least three months prior to the start of the Unicon.

\subsection{Maximum Number of Competitors for Jr. Expert and Expert: Group}
\oldrule{6.15.9}
\textbf{Non-Unicon:} Organizers of non-Unicon events can choose to limit the number of small/large groups using the guidelines below or have no limit.

\textbf{Unicon:} Each country can submit a maximum of three groups to compete at Unicon in each of the following categories: Expert Small Group, Jr. Expert Small Group, Expert Large Group, and Jr. Expert Large Group.

\subsection{Method for Limiting the Competitors at Unicon}
\oldrule{6.15.10}
A country that wishes to submit more than their allocated number of groups should select groups by their own way.
Any type of competition using the IUF judging methods to determine their groups is recommended.
If a country is unable to hold a competition, a country can choose groups by their own rating method.
It is strongly recommended to complete the selection at least three months prior to the start of the Unicon.
If a country cannot select by then, the method and schedule of the selection must be communicated to the Chief Judge and Artistic Director at least three months prior to the start of the Unicon.
