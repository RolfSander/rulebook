\chapter{Event Organizer Rules}

\section{Venue}

\subsection{Size Of Performing Areas}
\oldrule{5.16}
The minimum size for Freestyle event must be 28m x 15m.
Hosts shall give additional space to riders.
Hosts must publicize the dimensions of the available performing area as far in advance of the competition as possible, and organizers of international championships at least three months prior to the event.

\section{Officials}

\textbf{\begin{itemize}
\item Artistic Director
\item Chief Judge
\item Timer
\item Judge
\item Tabulator
\item Runner
\item Announcer
\item DJ
\item Rider Liaison
\item Stage Crew
\item Results Poster
\end{itemize}}

\section{Communication}

\subsection{Announcing Of Results}

\oldrule{5.7}
Final results will be continuously announced and/or posted for public view.
Results Sheets will be posted after each age category of an event.
The protest period begins at this point.

\section{World Champions \label{sec:freestyle_world-champions}}
\begin{framed}
We recommend that this be moved to Chapter 1 for all events.
\end{framed}

\oldrule{5.11}

Winners in the Expert category at of each event at Unicon are the \textbf{World Champions}.
In the individual events, separate titles are awarded for male and female.
Winners in the Jr. Expert category at Unicon are the \textbf{Junior World Champions}.

\section{Age Groups}

\oldrule{5.2}
\textbf{Note:} Age groups may be different for different types of event.
The minimum allowable age groups are listed for each event.
Convention hosts are free to add more age groups.
Age group is determined by the rider's age on the first day of the convention.
Junior Expert is open to all riders 0-14.
Expert is open to riders of any age, including 0-14.
Riders must state the age group in which they are entering for each artistic event in which they participate.

\textbf{Example:} Riders who enter Individual Freestyle as Experts can enter Pairs in their age group if they wish.
Riders are divided male/female in Standard Skill and Individual Freestyle, but not in Pairs or Group.

\subsection{Minimum Age Groups: Individual}
\oldrule{5.12.1}
 0-14, 15-UP, Expert.
The decision to enter as Expert or Jr. Expert is optional, but must be stated in advance.

\subsection{Minimum Age Groups: Pair}

\oldrule{5.13.1}
Age group (all ages), Expert.
Each rider may enter only once.
The age group of the older rider is the age group for the pair.
Expert is treated as the ``oldest'' age group, followed by Jr. Expert, and then all other age groups.
The decision to enter as Expert or Jr. Expert (if used) is optional, but must be stated in advance.

\subsection{Minimum Age Groups: Group}

\oldrule{5.14.3}
None.

\section{Practice}

\section{Order Of Performance}
\oldrule{5.17}
Performance order for Jr. Expert and Expert in Pairs/Individual/Group freestyle are defined by an open drawing without a computer.
The drawing/selection should be done publicly and transparently, at a time that is pre-announced, so people can witness it.
The method to determine performance order for age groups is completely up to the Artistic Director.

\section{Music \label{sec:freestyle_music}}
\oldrule{5.6}
In Freestyle events, music is included in the judging and competitors should use it.
In Standard Skill music is not judged.
But background music will be provided during all Standard Skill routines, or competitors may provide their own.
Competitors may also, at their request, have no music played.
It is recommended to have one or more backup copies of all music in case of loss or damage.
For recordable disks, competitors are also recommended to test their music on multiple players to make sure it will work at competition time.

\subsection{Media Types}
\oldrule{5.6.1}
The host is required to have the capability of playing recordable CDs.
Other media types may also be supported, at the host's discretion.
The Artistic Director is responsible for announcing what media types will be supported, and making sure the necessary equipment is provided.

\subsection{Music Preparation}
\oldrule{5.6.2}
Competitors must provide their music in a type that is supported, and has been announced by the Artistic Director.
All music must be clearly labeled with the competitor name(s), age group, event type (such as Pairs), and if needed, the track number.
Whenever possible, competition music should be the first track on the CD.
The DJ (music operator) is not responsible for any errors resulting from unsupported types or mislabeled tracks.

\subsection{Music Volume}
\oldrule{5.6.3}
Volume level is controlled by the DJ, at instructions from the Chief Judge.
The base volume for Freestyle music should be loud enough to sound clear, and be heard by all.
For Standard Skill, volume level should not be loud enough to interfere with judge communication, but otherwise similar to the level for Freestyle.
Some competitors' music may start with especially loud or quiet sections, and the DJ should be advised of these so volume levels do not get compensated in the wrong direction.
Some competitors may request that their music be played at lower levels.
These requests can be made directly to the DJ.
Requests for higher volumes must be approved by the Chief Judge, who has the option of passing this responsibility to the DJ.

\subsection{Special Music Instructions}
\oldrule{5.6.4}
Some competitors may have special music instructions, such as stopping or starting the music at a visual cue, changing volume level during the performance, etc.
The DJ is not responsible for errors carrying out these instructions.
For best results, the competitor should supply a person to coach the DJ during the performance, so there are no mistakes.
If the DJ receives instructions that sound unusual, the Chief Judge should be consulted for approval.

\section{Limiting Competitors}

\subsection{Maximum Number of Competitors for Jr. Expert and Expert: Individual}
\oldrule{5.12.7}
\textbf{Non-Unicon:} Organizers of non-Unicon events can choose to limit the number of competitors using the guidelines below or have no limit.

\textbf{Unicon:} Each country can submit a maximum of three individuals in each category to compete at Unicon in the Individual Freestyle events (three in Jr. Expert Male, three in Jr. Expert Female, three in Expert Male, three in Expert Female).
If a country has placed 1st, 2nd, or 3rd in Individual Freestyle at the previous Unicon, they can submit one additional competitor for each placing in that category.
For example, if Country-A wins first place in Expert Male at the previous Unicon, they may submit up to four individuals for Expert Male at the current Unicon.
If Country-B wins second and third place in Jr. Expert Female at the previous Unicon, they may submit up to five individuals in Jr. Expert Female at the current Unicon.

\subsection{Method for Limiting the Competitors at Unicon: Individual}
\oldrule{5.12.7}
A country that wishes to submit more than their allocated number of individuals should select individuals by their own way.
Any type of competition using the IUF judging methods to determine their competitors is recommended.
If a country is unable to hold a competition, a country can choose individuals by their own rating method.
For example, if a country has placed 1st, 2nd or 3rd in Individual Freestyle at the previous Unicon, it can give these individuals a higher rating, because they brought additional number of individuals to a country.
If a country did not place in the top three, it can give only the highest placing individual a higher rating.
It is strongly recommended to complete the selection at least three months prior to the start of the Unicon.
If a country cannot select by then, the method and schedule of the selection must be communicated to the Chief Judge and Artistic Director at least three months prior to the start of the Unicon.

\subsection{Maximum Number of Competitors for Jr. Expert and Expert: Pair}
\oldrule{3.13.7}
\textbf{Non-Unicon:} Organizers of non-Unicon events can choose to limit the number of competitors using the guidelines below or have no limit.

\textbf{Unicon:} Each country can submit a maximum of three pairs in each category to compete at Unicon in the Pairs Freestyle events (three in Jr Expert Pairs, three in Expert Pairs).
If a country has placed 1st, 2nd, or 3rd in Pairs Freestyle at the previous Unicon, they can submit one additional competitor for each placing in that category.
For example, if Country-A wins first place in Expert Pairs at the previous Unicon, they may submit up to four Pairs for Expert Pairs at the current Unicon.
If Country-B wins second and third place in Jr Expert Pairs at the previous Unicon, they may submit up to five pairs in Jr Expert Pairs at the current Unicon.
If a pairs team is submitted consisting of members from two countries, that team must choose one of their two countries to represent.

\subsection{Method for Limiting the Competitors at Unicon: Pair}
\oldrule{3.13.8}
A country that wishes to submit more than their allocated number of pairs should select competitors by their own way.
Any type of competition using the IUF judging methods to determine their competitors is recommended.
If a country is unable to hold a competition, a country can choose pairs by their own rating method.
For example, if a country has placed 1st, 2nd, or 3rd in Pairs Freestyle at the previous Unicon, it can give these pairs a higher rating if BOTH partners from the previous Unicon still be pairs, because they brought additional number of pairs to a country.
If a country did not place in the top three, it can give only the highest placing pairs a higher rating.
It is strongly recommended to complete the selection at least three months prior to the start of the Unicon.
If a country cannot select by then, the method and schedule of the selection must be communicated to the Chief Judge and Artistic Director at least three months prior to the start of the Unicon.

\subsection{Maximum Number of Competitors for Jr. Expert and Expert: Group}
\oldrule{3.14.9}
\textbf{Non-Unicon:} Organizers of non-Unicon events can choose to limit the number of small/big groups using the guidelines below or have no limit.

\textbf{Unicon:} Each country can submit a maximum of three groups to compete at Unicon in each of the following categories: Expert Small Group, Jr. Expert Small Group, Expert Big Group, and Jr. Expert Big Group.

\subsection{Method for Limiting the Competitors at Unicon: Group}
\oldrule{3.14.10}

A country that wishes to submit more than their allocated number of groups should select groups by their own way.
Any type of competition using the IUF judging methods to determine their groups is recommended.
If a country is unable to hold a competition, a country can choose groups by their own rating method.
For example, if a country has placed 1st, 2nd, or 3rd in Group Freestyle at the previous Unicon, it can give these groups a higher rating, because they brought additional number of groups to a country.
If a country did not place in the top three, it can give only the highest placing groups a higher rating.
Not all members from the previous Unicon are required to be members of a new group.
It is strongly recommended to complete the selection at least three months prior to the start of the Unicon.
If a country cannot select by then, the method and schedule of the selection must be communicated to the Chief Judge and Artistic Director at least three months prior to the start of the Unicon.

