\documentclass[a4paper,oneside,parskip=half,numbers=noenddot]{scrbook}
\textwidth=14.7cm
\textheight=22.1cm
%\hoffset=5.8754pt % enable if using letterpaper

% \usepackage{etoolbox}

\usepackage{framed}
\usepackage[utf8]{inputenc}
\usepackage[T1]{fontenc}
\usepackage{lmodern}
\usepackage{longtable}
\usepackage{multicol}
\usepackage[shortlabels]{enumitem}
\setlist{itemsep=-1mm, topsep=-1pt, partopsep=0pt}
% \usepackage[nenglish]{babel}


% turning off part numbering in the backmatter
% parts will still go to the TOC
\makeatletter
\g@addto@macro\backmatter{\setcounter{secnumdepth}{-2}}
\makeatother



% Numberings, table of Contents, bookmarks
\usepackage{hyperref} % clickable table of contents, index in pdf files (must be loaded before minitoc)
\usepackage{minitoc} % for table of contents per part
\usepackage{remreset} % for removal of counter resets
\usepackage[numbered]{bookmark} % customize PDF bookmarks
\renewcommand{\mtcgapbeforeheads}{0pt}
\renewcommand{\mtcgapafterheads}{0pt}


%TOC customization
\usepackage[tocindentauto]{tocstyle} % prettier TOC
\usetocstyle{allwithdot} %use 'KOMAlike" if you don't want dots
\settocstylefeature [-1] {entryvskip} {15pt}
\settocstylefeature [0] {entryvskip} {10pt}
\settocstylefeature [0] {entryhook} {\hspace{18pt} }
\mtcsettitle{parttoc}{Contents} %renames the part TOC to match main TOC
\usepackage[titles]{tocloft}
\renewcommand*\cftsecnumwidth{3em}


%%%headings
\usepackage[markcase=ignoreuppercase,autooneside]{scrlayer-scrpage} %better headings without uppercase TOC heading
\usepackage{titlesec}
\pagestyle{scrheadings}
\clearscrheadfoot %clear all header and footer styles to allow custimization
\cfoot[\pagemark]{\pagemark} %pagenumbers in footer
\makeatletter
\let\Oldpart\part
\newcommand{\parttitle}{}
\renewcommand{\part}[1]{\Oldpart{#1}\renewcommand{\parttitle}{#1}}

\renewcommand{\chaptermark}[1]{\markboth{#1}{}}


\chead[]{
  \if@mainmatter
    \ifnum\value{part}>0
      \ifnum\value{chapter}>0
        \thepart \ \parttitle -- \leftmark\fi
      \else
        \thepart \ \parttitle \fi 
      \fi
    \fi
  \fi}
\makeatother

\newcommand\singlechapter[1]{%if a part only has a single chapter in it use \singlechapter{} otherwise use the normal \chapter{}
\begin{mtchideinmaintoc}
\addstarredchapter{#1}
\chapter*{#1}
\markright{\thepart\ #1}
\end{mtchideinmaintoc}}


\setcounter{tocdepth}{-1} % only includes parts and chapters (not sections) in main TOC
\mtcsetdepth{parttoc}{1} % only includes chapters and sections (not sub or subsubsections) in part TOCs
\setcounter{secnumdepth}{3} % makes subsubsections numbered
\renewcommand{\thepart}{\arabic{part}} % arabic part numbering
\renewcommand{\thechapter}{\arabic{part}\Alph{chapter}} % make chapters numbered with part and then letter
%\renewcommand{\thesection}{\arabic{part}.\arabic{section}} % replace chapter number with part number in sections 
\makeatletter
\@addtoreset{chapter}{part} % reset chapter numbering for each part
\@addtoreset{section}{part} % reset section numbering for each part
%\@removefromreset{section}{chapter} % don't reset section numbering in new chapters
\makeatother
\hypersetup{
    colorlinks,
    citecolor=black,
    filecolor=black,
    linkcolor=black,
    urlcolor=black
}
% If you get the error: "Argument of \contentsline has an extra }"
% This is because hyperref and minitoc don't work perfectly together.
% Just continue to typeset ignoring all errors.
% Un-commenting the following line should supress errors.
% \batchmode
% Make sure to re-comment this line after typeseting to see future errors.
% Then typeset again and the error should be gone.
% This has to be done for each tex file (chapter) that contains a TOC.



\newcommand{\unit}[1]{\ensuremath{\, \mathrm{#1}}} % use \unit{km} in math mode for better units

\usepackage{graphicx}
\graphicspath{{img/}}
\usepackage{wrapfig}
\usepackage{pict2e}

% Documentation of gitinfo:
% http://ftp.fernuni-hagen.de/ftp-dir/pub/mirrors/www.ctan.org/macros/latex/contrib/gitinfo/gitinfo.pdf
%Simple instructions: copy "hooks" directory into ".git" directory
\usepackage{gitinfo}
