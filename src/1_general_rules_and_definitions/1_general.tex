\part{General Rules and Definitions}
\parttoc

\chapter{General}
This Rulebook is intended to govern all unicycle competition sanctioned by the International Unicycling Federation, and can be used as a guideline for other competitions.
The major sections are General, Racing, Artistic, Games, Trials and Skill Levels.
Various charts and forms that supplement these rules may be published separately.

\section{These Are Official IUF Rules}
All IUF Unicons (International Unicycling Conventions) must abide exclusively by these rules.
Further rules may be added to cover specific situations, but they may not override the IUF rules without prior approval by the IUF Board of Directors.
All additional rules must be published well in advance of international competition, and published together with registration forms.

\subsection{Updating This Rulebook}
This publication should be updated after every Unicon.
The IUF Rulebook Chair will head the committee, but may optionally name a sub-committee.
The Rulebook Committee will officially start meeting at the close of the Unicon, though the Chairperson can open it before, to take advantage of having so many persons physically together.
The Committee should finish their business and make their specific proposals 

\section{Convention Aspect of Unicon}
All competitions at a Unicon need to make every effort to have equal time for the convention side of Unicon by involving
as many competitors as possible and making the event spectator-friendly for other Unicon participants as well as non-
unicyclists. Any of the following are examples to achieve this goal:
\begin{itemize}
  \item{Workshops related to the event}
  \item{Fun competitions based on the event}
  \item{Instant results for the spectators}
  \item{Ways for other competitors to be introduced to the event}
  \item{Entertainment during breaks in the competition (such as half time entertainment)}
  \item{Schedule of the events posted in multiple places}
\end{itemize}

\section{Host's Option - Unicon}
Unicon should include at least one event from each of the following event groups. 
Hosts are free to add events, age groups or variations that do not appear here, as long as there is no conflict with the existing rules. 
When in doubt contact the IUF Rules Committee.
\begin{itemize}
  \item{Track Racing — the required races from section 2.16}
  \item{Other Racing — Road, specialty/novelty races; see section 2.20}
  \item{Team Games — Unicycle Hockey, Unicycle Basketball. See sections 6 and 7.}
  \item{Field events — Long Jump, High Jump, Gliding/Coasting. See section 2.19}
  \item{Non-competition events — workshops, fun games, sightseeing rides, MUni rides}
  \item{Artistic events—Freestyle, Standard Skill, Flatland, Street; see sections 3 and 4}
  \item{MUni—Cross Country, Orienteering, Uphill, Downhill, Trials; see sections 2.21 and 5}
\end{itemize}

\subsection{Safety Equipment}
Hosts are free to add additional rules for safety equipment, but are not allowed to remove any requirements listed in the Rulebook for any event.

\subsection{Combining Age Groups}
At an International convention six riders are needed to complete an age group. 
At smaller conventions with less than 50 riders, six in each age group are still highly recommended, however three riders are the minimum to complete an age group. 
Riders generally enter all events with their age group except for events similar to artistic competitions where there is Junior Expert and Expert categories. 
The convention host has the option of combining age groups which less than six riders (three riders for smaller conventions) if needed. 
This means that published age groups are not guaranteed. This can be done on a per-event basis. 
Age groups cannot be combined if they conflict with section 2.1.2 (Age Groups) that specifies minimum age groups for the required races. 
When combined, riders aged 18 and under would move up to the next older group. 
Riders over 18 would move down to the next younger group. If several age groups consecutively are collapsed, it might lead to riders of vastly different ages competing against each other. 
This problem should be taken into consideration.

\subsection{Awards}
The type, number, and quality of awards are the choice of the convention host. 
Because awards are paid for out of the convention budget, the host may determine the amount and level of those awards. 
Generally we have trophies for "top" events, medals for "sub-top" events, and ribbons or certificates for lower events or places. 
The IUF has most frequently awarded 1-3 place in most events, but this too is up to the convention host.

\subsection{Video recordings for protest}
Private videos are generally forbidden as a means of verification in case of a protest. 
The host can decide to make an official video of some competitions (e.g. at the 5-meter-line of the 50m one-foot race), which he has to announce before the competition to let the competitors know about their option to protest through this video.

\subsection{Sponsors}
The convention host has the option to seek and obtain private sector Sponsorship; e.g. The Unicycle.com Freestyle Awards, the Coca-Cola Hockey Cup, etc. 
This will allow opportunities for external funding to defray costs for host organizations and competitors. 
Sponsors are limited to organizations that would not bring the IUF into disrepute and are consistent with the aims and objectives of the International Unicycling Federation, Inc.

\section{Notification, Disclosure, and Communication}
Convention dates and other information must be announced and/or published at the earliest possible date. 
The best way to control the publication of convention information is with a convention Web site, with regular updates to provide all the latest information. 
For Unicon and other large events, registration forms should be made available no less than eight months before the convention start date. 
A list of all planned competition events, including all rules and information pertinent to quality training, should be published at the same time with newly available data to be added as soon as it is known. 
Wherever possible, hosts should provide maps, directions and other information to help make peoples' convention as enjoyable as possible.

\subsection{Providing Special Rules}
For international competitions, written rules are needed for any planned events not described in the IUF Rulebook, and for events where additional rules are required. 
These special rules could be variations on the optional events found in this Rulebook. 
Such rules should be published at the same time as registration forms, or earlier, and must be published at least one month before the start of the event. 
These rules can be published along with registration forms, and/or on the convention web site. 
Competitors need to know the specific rules so they can train for those specific events! 
Hosts also need to decide on rules early, so there is less to worry about near competition time. 
Rule changes may be a necessary reality, for reasons such as changes in venue, weather or available equipment. 
When this happens these changes must be posted to the convention web site immediately. 
Examples: Dismount rules or timing details for off- track races, obstacle information for Street Comp, planned age divisions or combination awards.

\subsection{Full Disclosure of Course Details}
Details of all non-track racing events, or other events with unique courses or details must be published as soon as they are known. 
This is to provide competitors with the information they need to train, and to help them prepare the appropriate unicycles. These are major needs for attendees from far away. 
Necessary details depend on the event, but include things like course length, elevation and elevation change, steepness, level of terrain difficulty, amount of turns, riding surfaces, course width, etc. Maps should be provided when possible. 
While sometimes courses cannot be planned until weeks or days before the convention, as soon as they are known the details must be posted to the convention web site and/or all places where convention information is posted. 
It is acceptable to publish tentative courses while waiting for permits to be approved, etc.

\subsection{Course Availability for Practice}
If the course is open for practice to all riders for at least 7 days leading up to the event, then there are no restrictions on who can compete. 
If the course is not open for practice until the day of the event, then anyone who has pre-ridden the course is not allowed to compete. 
Organizers must therefore ensure that course marking and set-up are done by non- competing staff/volunteers.

\subsection{Communication}
In international events especially, good communication makes the difference between a memorable event, and frustration for many. 
Hosts must cultivate good lines of communication to attendees, both before the convention starts and once people have arrived. 
Team mailboxes, contact persons, centralized phone numbers or an organized method must be used to keep people aware of schedule changes, venue changes, last-minute details, etc.
