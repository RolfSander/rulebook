
\chapter{Competitor Rules}

\section{Safety}
Riders must wear shoes, kneepads, gloves and a helmet (see definitions in chapter \ref{chap:general_definitions}).

\textbf{Personal music systems are disallowed.}

Water and food are the responsibility of the rider. Hosts may offer food and water stations at their discression.

\section{Unicycles}

\textbf{This section needs to be addressed in committee.  Probably, unlimited needs to be allowed, here.  The definitions in section 1d need to be updated, accordingly.}

Only standard unicycles may be used.
\subsection{Wheel Size and Crank Arm Limit}
\begin{itemize}
\item For a ``Standard 29$"$ Unicycle'' the outside diameter of the tire may not be larger than 768mm and there is no minimum crank arm limit.
No gearing may be used.
\item For a ``Standard 24$"$ Unicycle'' the outside diameter of the tire may not be larger than 618mm and crank arms may be no shorter than 125mm.
No gearing may be used.
\item For a ``Standard 20$"$ Unicycle'' the outside diameter of the tire may not be larger than 518mm and crank arms may be no shorter than 100mm.
No gearing may be used.
\end{itemize}

24$"$ and smaller wheels are not allowed for races longer than 20km without express permission of the racing director.

\section{Rider Identification}

\textbf{Something probably needs to be said about rider number or the wearing of a chip.  This seems to be missing from the exiting rules.}

\section{Event Flow}

\textbf{This the sections could be lifted up one level, but we arranged it this way, because ``event flow'' is an important element of all events.}

\subsection{Riders Must Be Ready}
Riders must be ready when called for their races.
Riders not at the start line when their race begins may lose their chance to participate.
The Starter will decide when to stop waiting, remembering to consider language barriers, and the fact that some riders may be slow because they are helping run the convention.

\subsection{Starting}
Riders start mounted, holding onto a starting post or other support.
The Starter will give a four-count start, for example, ``One, two, three, BANG!'' Alternatively, an electronic starter may be used.

Riders start with the fronts of their tires (forwardmost part of wheel) behind the nearest edge of the starting line.
Rolling starts are not permitted in any road race.
However, riders may start from behind the starting line if they wish, provided all other starting rules are followed.
Riders may lean before the gun fires, but their wheels may not move forward before the gun fires.
Rolling back is allowed, but not forward.
Riders may place starting posts in the location most comfortable for them, as long as it doesn't interfere with other riders.

A rider's starting time is taken as when their heat begins (when the gun goes off) regardless of when they actually cross the starting line.

\subsection{False Starts}
A false start occurs if a rider's wheel moves forward before the start signal, or if one or more riders are forced to dismount due to interference from another rider or other source.

False starts will be handled as determined by the host, and as described in section \ref{subsec:false_starts}.

\subsection{Passing}
When passing on a curve, the overtaking rider must pass on the outside, unless there is enough room to safely pass on the inside.
A slower rider must maintain a reasonably straight course, and not interfere with a faster rider.

\subsection{Dismounts}
Dismounting and remounting is allowed. 
If a rider is forced to dismount due to a fall by the rider immediately in front, it is considered part of the race, and both riders should remount and continue.

\subsection{Illegal Riding}
Illegal riding includes intentionally interfering in any way with another rider, deliberately crossing in front of another rider to prevent him or her from moving on, deliberately blocking another rider from passing, or distracting another rider with the intention of causing a dismount.
A rider interfered with by another rider may file a protest immediately at the end of the race.
Riders who intentionally interfere with other riders may receive from the Referee a warning, a loss of placement (given the next lower finishing place), disqualification from that race/event, or suspension from all races.

\subsection{Finishes}
Finish times are determined when the front of the tire first crosses the vertical plane of the nearest edge of the finish line. Riders are always timed by their wheels, not by outstretched bodies.

If riders do not cross the line in control, they are awarded a 5 second penalty to their time.
``Control'' is defined by the rearmost part of the wheel crossing completely over the vertical finish plane (as defined above) with the rider having both feet on the pedals.
(Note: a rider is not considered in control if the unicycle crosses the finish line independent of the rider.
The finish time is still measured by when the wheel crosses the vertical finish plane and the 5 second penalty is applied.)

In the case where a rider is finishing with a broken unicycle, the rider must bring at minimum the wheel to the finish line, and time is still taken when the wheel crosses the finish line.
The 5 second penalty is applied.

\subsection{Optional Race-End Cut-Off Time}

There may be a race cut-off time, as communicated by the host.
