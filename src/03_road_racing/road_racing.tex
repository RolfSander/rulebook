\chapter{Road Racing}

\section{Definition} %more needed here?
These are races held usually on roadways or bike paths, generally for longer distances than our events on the track.
All riders may race together and be separated by age group afterward.

\section{General Rules}
Water/food stations are recommended.
Personal music systems are not allowed for any races on public roads where there may be motorized traffic.
Riders can be divided by age and/or unicycle type, such as 24$"$ and 29$"$ track unicycles, Standard (any size wheel and cranks), and Unlimited (see definitions).
24$"$ and smaller wheels are not allowed for races longer than 20km without express permission of the racing director.
Traditional road race distances have been 10k, Full Marathon (42.195k) and 100k.
Any distances can be used, but if specific distances are advertised, the race course must be measured to be the correct distance.

\section{Dress}
Riders must wear shoes, kneepads, gloves and a helmet (definitions in section 1.23).
The Referee has final say on whether a rider’s safety equipment is sufficient.
Elbow pads are also good considerations for safe unicycle racing.
The Starter will remove from the starting line-up any riders not properly equipped to race, including riders with dangerously loose shoelaces.

\section{Age Groups}
The following age groups are the minimum required by the IUF to be offered at the time of registration for any Road Racing discipline: 0-13, 14-18, 19-29, 30-UP.
For any discipline for which there is a Standard 24$"$ wheelsize category, also an age group 0-10 (20$"$) must be offered.

Convention hosts are free to offer more age groups, and often do.
For example, a full range of offered age groups might look like 0-8 (20$"$), 9- 10 (20$"$), 0-12, 13-14, 15-16, 17-18, 19-29, 30-39, 40-49, 50-59, 60-UP.
All age groups must be offered as male and female age group.

\section{Traditional Race Distances}
Traditional road race distances have been: 
\begin{itemize}
\item 10k with Unlimited and Standard 24$"$ classes, and 
\item Full Marathon (42.195k) with Unlimited and possibly Standard 29$"$ classes.
\end{itemize}
However, any distances or wheelsize classes can be used for Road Races.

\subsection {Distance Measurement for Traditional Distances}
In the case where a traditional race distance is used (i.e. 10k or Marathon---42.195k), the course must be accurately measured along the shortest possible path.
The course must be guaranteed to be no shorter than the advertised distance.

The following procedure is acceptable for accuracy.
A more accurate method is of course allowed.
\begin{enumerate}
\item Set out a calibration course on straight, flat asphalt, with a minimum length of 100 meters, using a steel measuring tape of 5 meters or longer.
\item Ride the calibration course at least once with a bike or unicycle (minimum wheelsize 24$"$).
Ride normally, without too much wobble, and at normal speed.
Take care that mounting and dismounting don't cause the wheel to swerve, or be lifted from the surface.
Carefully count the number of wheel revolutions required to ride the calibration course.
Include partial wheel revolutions (e.g. through counting the number of spokes passed for the last partial revolution).
\item Calculate the wheel rollout (meters per revolution) from step 2.
\item If you are going to use a cycle computer: enter the wheel rollout value to the nearest millimeter in a reliable cycle computer with a wheel sensor (e.g. a magnet).
\item Fit the cycle computer, or a wheel revolution counter, to the same bike or unicycle used in Step 2.
\item Ride the actual race course, following the shortest possible path.
Take care to ride in the same way as in step 2.
\item Read the distance from the cycle computer, or calculate from wheel revolutions and wheel rollout.
\item Calculate the applicable safety margin by adding up (1) $0.4\%$ of the measured distance, and (2) the resolution of the cycle computer distance readout.
\textbf{Example:} if your cycle computer shows $10.15\unit{km}$, the safety margin is $0.4\% \cdot 10.15\unit{km} + 0.01\unit{km} = 0.0506\unit{km} = 50.6\unit{m}$.
\textbf{Note:} you can skip (2) if you use a wheel revolution counter that can resolve single wheel revolutions.
\item Add the safety margin to the actual course (i.e. shift the start and/or finish line), to guarantee that the course is at least the advertised distance.
\end{enumerate}
Note that Steps 2 through 7 must be done without breaks.
The same rider should ride the calibration course and the race course.
The tire pressure should not be altered in the mean time.

\subsection {Distance Measurement for Other Distances}
In the case where a non-traditional race distance is used (i.e. any distance other than 10k or 42.195k), the course must be measured with an accuracy of plus or minus 3\% or better.
\textbf{Example:} if a race is advertised as 100 km, the actual distance must be between 97 and 103 km.
A good consumer-type GPS unit is acceptible, provided the track shows continuous reception of suffucient satellites (no `stray' data points, or missing points).
Also acceptable is the Distance Measurement Tool of Google Maps.
A car odometer, on the other hand, might easily be off by more than 3\%, and is therefore not acceptable unless you know how to correct it.
Obviously, using a more accurate measurement is allowed, such as the method described for `traditional distances'’'.

\section{Unicycles for Racing}
Only standard unicycles may be used.
Riders may use different unicycles for different racing events, as long as all comply with the rules for events in which they are entered.
\subsection{Wheel Size and Crank Arm Limit}
\begin{itemize}
\item For a ``Standard 29$"$ Unicycle'' the outside diameter of the tire may not be larger than 768mm and there is no minimum crank arm limit.
No gearing may be used.
\item For a ``Standard 24$"$ Unicycle'' the outside diameter of the tire may not be larger than 618mm and crank arms may be no shorter than 125mm.
No gearing may be used.
\item For a ``Standard 20$"$ Unicycle'' the outside diameter of the tire may not be larger than 518mm and crank arms may be no shorter than 100mm.
No gearing may be used.
\end{itemize}

\section{Heat Assignments}
Line-up order and heats must be assigned prior to the race.
There are three allowable formats for designating the starting configuration of a Road Race: individual start (section \ref{subsec:road_heat-assignment_individual-start}), heat start (section \ref{subsec:road_heat-assignment_heat-start}), or mass start (section \ref{subsec:road_heat-assignment_mass-start}).

To determine which start configuration to use, read the following rules from top to bottom.
Once you have an outcome, \textem{disregard} the remaining rules.
\begin{itemize}
\item If this is an ``Individual Time Trial'' format race, use individual start.
\item If the course is too narrow to allow for racers to safely and fairly start in heats, use individual start.
\item If you cannot safely start five or more riders across, use individual start.
\item If the starting field consists of 30 riders or less, use a mass start.
\item If the course does not allow for ten riders to ride abreast for at least 500m before the course narrows, use heats of 12 or more riders.
\item If the starting field consists of more than 50 riders, use heats of 20 or more riders.
\item In all other cases, use a mass start.
\end{itemize}
Standard racers should always start separately from Unlimited racers, also in the case of a mass start or heat starts.
Unlimited racers should start first, unless there is no risk that Unlimited riders have to pass standard riders (e.g. they race on different days).

In the sections below, ``fastest rider'' means ``fastest rider by seed time.'' Seed time is defined as an estimated finish time, preferably based on past performance in similar event(s).
If no seed time is submitted by the rider or their coach, the organisation can assign a seed time.

\subsection{Starting Order}
The goal in determining the starting order is to sort racers fairly by speed while still making sure that genders race amongst themselves.
Unless otherwise noted below, the fastest riders start first, and also within a start group (heat or mass start), riders should be positioned in the line-up by speed with the fastest in front.
Starting order can be determined by seed time, or from the results of a previous Road Race in that competition.
For example, if the Marathon follows the 10k, the results of the 10k can be used to determine the starting order for the Marathon.
In the case that a racer does not have a seed time, and is signed up for a particular event (i.e. the Marathon) and did not participate in the previous race (i.e. the 10k), the Racing Clerk has the right to assign a starting position where they see fit.

\subsection{Individual Start \label{subsec:road_heat-assignment_individual-start}}
Each rider is individually started at a fixed time interval, such as every 20 or 30 seconds.
Riders are sorted by speed with the fastest rider going first.
(Except in the case of an Individual Time Trial, where the race can start with either the fastest or slowest rider.)

\subsection{Heat Start \label{subsec:road_heat-assignment_heat-start}}
Heats should consist of at least 12 riders, either male or female (no mixed heats).
Heats are sorted by speed with the fastest heat going first.
Heats should be started every one to five minutes.

\textbf{The following example describes how this can be done:}
The first three heats might contain the fastest men, then a heat of the fastest women who are of proportionate speed with the third heat of men.
This format makes sure that the top women start together while still giving them the opportunity to race and pace off of men of similar speed.

\subsection{Mass Start \label{subsec:road_heat-assignment_mass-start}}
A mass start is a start in which all racers of a certain class (such as Standard or Unlimited) start together.
Genders start at the same time.

\section{Special Marathon Events}
Exceptions from the default rules may be allowed for a marathon race that is embedded in a big city marathon (i.e. Düsseldorf Marathon).
This allows the unicycling organizer to follow some requirements of the main marathon organizer in order for the unicycling marathon to fit within the larger event.

The following exceptions to the rules may be made:
\begin{itemize}
\item Mass start / Group start (Mass start could be forced by the main host for schedule requirements) 
\item Start groups do not have to be per gender and/or wheel size
\item Netto times (time from when the rider's wheel crosses the start line) can be used for placements while the Brutto time (time from when the race is started) counts for Records.
\end{itemize}

\section{Starting}
Riders start mounted, holding onto a starting post or other support.
The Starter will give a four-count start.
For example: ``One, two, three, BANG!'' There should be about 3/4 second between each number in the count.
This allows riders to predict the timing of the gun, for a fair start.
Starters should practice this before the races begin.
Timing of the count is very important for an accurate start.
This count can be in the local language, or a language agreed upon before competition starts.

As an alternative a Startbeep apparatus can be used.
In that case we have a six-count start.
For example: ``beep - beep - beep - beep - beep - buup!'' The inter-beep timing is one second.
The first 5 beeps have all the same frequency.
The final tone (buup) has a higher frequency, so that the racer can easily distinguish this tone from the rest.

Riders start with the fronts of their tires (forward most part of wheel) behind the edge of the starting line that is farthest from the finish line.
Rolling starts are not permitted in any race.
However, riders may start from behind the starting line if they wish, provided all other starting rules are followed.
Riders may lean before the gun fires, but their wheels may not move forward before the gun fires.
Rolling back is allowed, but nothing forward.
Riders may place starting posts in the location most comfortable for them, as long as it doesn't interfere with other riders.

A rider’s starting time is taken as when their heat begins (when the gun goes off) regardless of when they actually cross the starting line.
\subsection{Riders Must Be Ready}
Riders must be ready when called for their races.
Riders not at the start line when their race begins may lose their chance to participate.
The Starter will decide when to stop waiting, remembering to consider language barriers, and the fact that some riders may be slow because they are helping run the convention.

\section{False Starts}
A false start occurs if a rider's wheel moves forward before the start signal, or if one or more riders are forced to dismount due to interference from another rider or other source.

There are several options on how to deal with false starts:
\begin{itemize}
\item \textbf{One False Start Allowed Per Rider:}
In case of a false start, the heat is restarted.
Any rider(s) who caused their personal first false start may start again.
Any rider(s) causing their personal second false start are disqualified.
\item \textbf{One False Start Allowed Per Heat:} 
In case of a false start, the heat is restarted.
For the first false start of a particular heat, all riders may start again.
Thereafter, any rider(s) causing a false start are disqualified.
\item \textbf{Time Penalty:}
In case of a false start, the heat is not restarted.
If a false start occurs by one or multiple riders, these riders receive a time penalty (e.g. 10 seconds).
\end{itemize}
If a heat has to be restarted, the Starter will immediately recall the riders, e.g. by firing a gun or blowing a whistle or any other clear and pre-defined signal.

If the race is started using individual starts or heat starts (see \ref{road:individual_start} and \ref{road:heat_start}) a time penalty is the recommended option.
In the case of a mass start (\ref{road:mass_start}), any option is viable.
The host must announce the false start method at least two months before the event.

\section{Riding Behavior}

\subsection{Passing}
An overtaking rider must pass on the outside, unless there is enough room to safely pass on the inside.
Riders passing on the inside are responsible for any fouls that may take place as a result.
No physical contact between riders is allowed.
The slower rider must maintain a reasonably straight course, and not interfere with the faster rider.

\subsection{Dismounts}
A dismount is any time a rider’s foot or other body part touches the ground and the unicycle must be remounted.
If a rider is forced to dismount due to a fall by the rider immediately in front, it is considered part of the race and both riders must remount and continue.
The Referee can override this rule if intentional interference is observed.

\subsection{Illegal Riding}
This includes intentionally interfering in any way with another rider, deliberately crossing in front of another rider to prevent him or her from moving on, deliberately blocking another rider from passing, or distracting another rider with the intention of causing a dismount.
A rider who is forced to dismount due to interference by another rider may file a protest immediately at the end of the race.
Riders who intentionally interfere with other riders may receive from the Referee a warning, a loss of placement (given the next lower finishing place), disqualification from that race/event, or suspension from all races.

\section{Finishes}
Finish times are determined when the front of the tire first crosses the vertical plane of the nearer edge of the finish line.

Riders are always timed by their wheels, not by outstretched bodies.
If riders do not cross the line in control, they are awarded a 5 second penalty to their time.
``Control'' is defined by the rearmost part of the wheel crossing completely over the vertical finish plane (as defined above) with the rider having both feet on the pedals.
(Note: a rider is not considered in control if the unicycle crosses the finish line independent of the rider.
The finish time is still measured by when the wheel crosses the vertical finish plane and the 5 second penalty is applied.)

In the case where a rider is finishing with a broken unicycle, the rider must bring at minimum the wheel to the finish line, and time is still taken when the wheel crosses the finish line.

If finish times for a race are timed using microchips or other non-photographic electronic equipment, finish order must be verified by photo timing equipment if the finishers are within 0.1 seconds of each other.
Also, in the case where a world record is suspected of being set, the time must be verified with photo timing equipment.

\section{Optional Race-End Cut-Off Time}
It may be necessary to have a maximum time limit for long races, to keep events on schedule.
When this is planned in advance, it must be advertised as early as possible, so attending riders will know of the limit.
Additionally, at the discretion of the Racing Director, a race cut-off time may be set on the day of or during an event.
The purpose of this is to allow things to move on if all but a few slow racers are still on the course.
These cut-offs need not be announced in advance.
At the cut-off time, any racers who have not finished will be listed as incomplete (no time recorded, or same cut-off time recorded for all).
Optionally, if there is no more than one person on the course per age category and awards are at stake, they can be given the following place in the finishing order.
But if each participating age category has had finishers for all available awards (no awards at stake), there is no need to wait.



