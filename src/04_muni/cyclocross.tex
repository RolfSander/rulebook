\chapter{Cyclocross \label{chap:cyclocross} \setcounter{chapter}{1}}

All rules in the general muni chapter apply unless changed below.

\section{Overview}
Cyclocross racing is one of the fastest growing segments of bicycle racing in the United States.
The course typically includes grassy fields, pavement, and possibly dirt trails.
The courses are multi-lap and very spectator friendly.
The typical bike of choice is a drop bar bike with skinny knobby tires.
Cyclocross differs from the usual bike races in that there are mandatory dismounts sections, called run-ups, where racers must carry their bikes over barriers, up stair-sets, or other obstacles that are too big to ride over.
The riders run through these sections and remount on the other side.
The speed in which a rider can transition from riding to running and back to riding is very important.

\section{Course Definition}
It will be a multi-lap event featuring a bit of cross country trail, grassy fields and natural and man-made obstacles where dismounting will be necessary.
A course should have no fewer than two and no more than six obstacle or barrier sections where riders normally dismount and run with the unicycle.
The starting and finishing stretches shall be free of obstacles within 10 meters.

It is suggested that the length of the course (used by both classes) not be much shorter than 1km in length and no longer than 2.5km in length. 
Organizers should keep in mind that most of the course should be visible from several vantage points.

\section{Classes}
It is advised that Cyclocross be run as two separate races, (Unlimited and Standard) as the nature of a multi-lap event on a short course will lead to passing and lapping.

For Cyclocross, Unlimited is defined as any standard unicycle (see definition in chapter \ref{chap:general_definitions}) with a rim with a bead seat diameter (BSD) of 622mm (700c) or larger.
Unlimited also includes a unicycle with a rim smaller than a 622mm BSD {\em only} if it has a functioning gearing system which will yield a virtual wheel size greater than a 622mm BSD.
The Standard category only includes standard unicycles without gearing and with a rim less than 622mm BSD.
There are no restrictions on crank length.
For example, the following wheel sizes generally fit the above definitions: \\
Unlimited: 28/29/36, geared 24/26/28/29/36 inch \\
Standard: 24/26 inch

\section{Starting}
There will be a Le Mans style start.
Unicycles will be lined up in a designated area away from the riders near the lap/finish line.
Riders will line up behind an additional line and then be required to run to retrieve their unicycle when the race starts.
They will then need to mount their unicycle to ride.
Riders must be mounted within 10 meters after crossing the lap/finish line.

\section{Timing}
It is suggested that the Unlimited race be close to 45 minutes in length and the Standard race be close to 30 minutes in length.
Using the time from the top rider's first two laps, the referee will determine how many laps could be completed in the desired time limit (i.e. 45 minutes).
From this point on, the number of remaining laps (for the leaders) will be displayed and this will be used to determine when finish of the race occurs. A bell will be rung with one lap to go.

\section{Dismounts and Dismounted Riders}
Upon dismounting there are no restrictions about passing riders.
Dismounted riders may run with their unicycle.
Courtesy is expected to avoid accidents, but the running unicyclist does not have to yield to riding unicyclist.

\section{Lapping}
In the case of a rider being lapped, the passing rider has the right-of-way.
The approaching rider needs to alert the slower rider of their intentions to pass.
Special care at international events should be taken due to language differences.
Lapped riders in the race will all finish on the same lap as the leader and will be placed according to the number of laps they are down and then their position at the finish.

\section{Obstacles}
Riders cannot cut the course around the obstacles.
They may ride through the obstacle section if possible or dismount and run with their unicycle.
By definition, the majority of riders should not be able to ride or hop the obstacle section.
Riding or hopping through the obstacle section should not damage or break the obstacle.

\section{Majority Riding Rule}
Unicyclists must attempt to ride at least 50\% of the course on each lap.
This is to avoid someone running the entire race carrying or pushing a unicycle without riding it. 
A racer in violation will be warned by a racing official.
Failure to heed the warning will result in a disqualification.

\section{Finishing}
Riders must ride completely across the finish line in control, as described in section \ref{sec:track-field_finishes}.
