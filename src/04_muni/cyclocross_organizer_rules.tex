\chapter{Event Organizer Rules}

\section{Venue}

It will be a multi-lap event featuring a bit of cross country trail, grassy fields and natural and man-made obstacles where dismounting will be necessary.
A course should have no fewer than two and no more than six obstacle or barrier sections where riders normally dismount and run with the unicycle.
The starting and finishing stretches shall be free of obstacles within 10 meters.

It is suggested that the length of the course (used by both classes) not be much shorter than 1km in length and no longer than 2.5km in length. 
Organizers should keep in mind that most of the course should be visible from several vantage points.

\section{Officials}

\textbf{Does the IUF require particular officials for these events?}

\section{Communication}

\textbf{Some thoughts on what might need to be communicated:
\begin{itemize}
\item age groups
\item results
\end{itemize}
}

\section{Age Groups}

\textbf{This was copied from the muni chapter.}

Age groups must be offered as male and female age group.
There must not be any age group specific restrictions on equipment.
The following age groups are the maximum allowable for muni competitions:

\begin{tabular}{ l l}
Under 15 & Youth \\
15-16 & Juniors \\
17-18 & Rookies \\
19-29 & Elites \\
30-49 & Masters \\
50+ & Veterans \\
\end{tabular}

\section{Practice}

\textbf{Does the course need to be set up early enough to allow riders to practice, or is practice specifically disallowed?}

\section{Race Configuration}

It is advised that Cyclocross be run as two separate races, (Unlimited and Standard) as the nature of a multi-lap event on a short course will lead to passing and lapping.

It is suggested that the Unlimited race be close to 45 minutes in length and the Standard race be close to 30 minutes in length.
Using the time from the top rider's first two laps, the referee will determine how many laps could be completed in the desired time limit (i.e. 45 minutes).
From this point on, the number of remaining laps (for the leaders) will be displayed and this will be used to determine when finish of the race occurs. A bell will be rung with one lap to go.

\section{Starting Configuration}

\section{Starting Order}

\section{Starter}

\textbf{Are there any starter requirements?}

\section{False Starts}

\textbf{Are there any false start rules?}

\section{Finishes}

\textbf{Are there any finishing?}

\section{Disputes}

\textbf{Some words about how disputes are handled should go here.}
