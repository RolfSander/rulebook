\singlechapter{Jumps \label{chap:Jumps}}

Unicycling versions of the High Jump and Long Jump.

\section{Unicycles}
Standard unicycles must be used (see definition in chapter \ref{chap:general_definitions}).
No restriction on wheel or crank size.
Metal pedals are allowed for their strength and better grip.
This may make it impossible to hold this event on a sensitive track surface.

\section{Broken Unicycle}
If the unicycle breaks during an attempt, a new attempt must be given to the rider.

\section{Safety Gear}
Riders must wear shoes, kneepads, gloves and a helmet (definitions in chapter \ref{chap:general_definitions}).

\section{High Jump}
The rider and unicycle jump over a bar, without knocking it down, and ride away without a dismount.
There are three parts to a successful jump: 
\begin{enumerate}
\item Riders must mount before the start line, to show they are on the unicycle and in control.
The attempt starts when the rider crosses the start line.
The rider may break off from a jumping attempt before leaving the ground, but must then start again from behind the start line.
That attempt then doesn't count.
\item Riders must jump over the bar without knocking the bar off the apparatus.
The bar can be hit as long as it does not fall.
If the bar falls before the rider crosses the finish line, it counts as an unsuccessful attempt.
\item After landing, the rider must stay in control of the unicycle until he cross the finish line without dismounting, touching a hand to the ground or any other object, or knocking down the bar or any of the high jump apparatus.
Riders get two attempts at each height.
The rider starts at a low height and after each successful attempt, the height increases at set intervals until the rider fails to be successful on both attempts.
When the rider fails both attempts, the maximum height that was completed is recorded.
\end{enumerate}

\subsection{Setup}
Around the High Jump apparatus a circle with a radius of 3 meters must be marked.
This circle is start and finish line.
The rider can cross it wherever he wants.
Riders must ride or hop across the finish line for the attempt to count.
Successfully crossing the finish line is judged the same as in racing (see section \ref{sec:track-field_finishes}).
The bar must be held loosely in the jumping apparatus so it can fall or break away if the rider does not complete the desired height.
Magnetic systems are not allowed.
The bar shall have a minimum diameter of 2cm.

\section{Long Jump}
The rider jumps as far as possible from a jump marker, to a landing without a dismount.
The rider must then continue riding across a finish line to show control.
Riders must clear 3 markers (jump marker, landing marker and finish line) to make the jump count.
Riders may jump with the wheel going forward or sideways.
After landing, the rider must stay in control of the unicycle for the remainder of the distance from the jump marker to the finish line without dismounting, or touching a hand to the ground or any other object.
If the tire touches the jump marker before takeoff or the landing marker, it counts as a foul.
Riders may break off in a run as long as he is between start line and jump marker but if they cross or touch the jump marker, the attempt counts, including fouls.
Riders get two attempts for each length.
The farthest non-fouling, successful jump is recorded.

To avoid endless competitions, the length to jump will always increase by 5cm for each round.
Once there are only 5 riders left, it's up to the riders to decide in which steps they continue.
For each age group the minimum length should be adjusted to a useful level such as 150cm for 15+ and 70cm for 0-15.
The host can adjust this depending on the level at his competition.

\subsection{Setup}
The riding area consists of a start line, a jump marker, a landing marker and a finish line beyond the jump marker.

The finishing line should be at least 4 meters from the landing marker but no more than 8 meters away.
We suggest that judges set up the finishing line 8 meters from the jump marker and move it further away if need during longer jumps.
Riders must ride or hop across the finish line for the attempt to count.
Successfully crossing the finish line is judged the same as in racing (see section \ref{sec:track-field_finishes}).
The start line must be a minimum of 25 meters in front of the jump marker to allow the riders to accelerate.
There must be an area behind the finishing line which is a minimum of 7 meters long and 2 meters wide as safety zone.
Riders may use all or part of the 25 meters between start line and jump marker.
Riders are also allowed to start from beside to be able to do accelerated side jumps.
Markers for takeoff and landing (jump marker and landing marker) must consist of a material which not can be deformed to have the same conditions for all riders.
The markers must be at least 1.20 meter in width (across the runway), no more than 10 mm in height (above the runway), and no less than 5 centimeters in depth (front to back).
A Long Jump competition needs a minimum area of 40x2.5 meters.

\subsection{Judging}
The rider must clear the jump marker and the landing marker without touching them; he also has to clear the finish line to make it a valid jump.
Jump distance is measured between the outer edges of the jump and landing marker.
There has to be at least one judge (better two) to look at the markers.
For national championships and Unicons, two judges are always needed; one to observe each marker.
