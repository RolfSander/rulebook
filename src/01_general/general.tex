\chapter{General}
This Rulebook is intended to govern all unicycle competition sanctioned by the International Unicycling Federation, and can be used as a guideline for other competitions.
The major sections are General, Racing, Artistic, Games, Trials and Skill Levels.
Various charts and forms that supplement these rules may be published separately. 

\section{These Are Official IUF Rules}
All IUF Unicons (International Unicycling Conventions) must abide exclusively by these rules.
Further rules may be added to cover specific situations, but they may not override the IUF rules without prior approval by the IUF Board of Directors.
All additional rules must be published well in advance of international competition, and published together with registration forms.

National or local unicycling organizations may have their own rules.
Though they may use IUF rules as a basis for their own rules, in national or local competitions, those rules can no longer be called IUF rules.
To get proper results for Unicon qualification it is needed to follow the IUF Rulebook as described above.

Any national organization that wishes to get its modified rules approved by the IUF for a national competition must submit a proposal to the IUF Executive Board at least 90 days before the start of the given event.
If approved, the national competition can still be recognized as an official IUF event.

To host an ``official IUF event'' means that the results of this event are comparable with results from other official IUF events and can count for possible qualification restrictions.
Rules, which are approved for use at a national or local competition by the IUF Board to be used, must not be referred to as IUF rules to prevent confusion for the riders.

\subsection{Updating This Rulebook}
This publication should be updated after every Unicon.
The IUF Rulebook Chair will head the committee, but may optionally name a sub-committee.
The Rulebook Committee will officially start meeting at the close of the Unicon, though the Chairperson can open it before, to take advantage of having so many persons physically together.
The Committee should finish their business and make their specific proposals within three (3) months of the close of the Unicon.
If they need more time, they may ask the IUF President for a time extension.
This is meant to be the only time that changes to the Rulebook are made, although exceptions are possible in extraordinary cases.
The IUF President is responsible for making sure that the Rulebook committee stays focused and on schedule.

Anyone may submit a potential change to IUF Rulebook at any time.
These will not be official proposals, but suggestions for potential topics during the next Rulebook session.
A forum will also be provided to discuss potential changes throughout the year.
The Rulebook Committee voting time frame and official members of the Rulebook Committee, however, will still be determined by the IUF Rulebook Committee Chair and the IUF Executive Board.

\section{Convention Aspect of Unicon}
All competitions at a Unicon need to make every effort to have equal time for the convention side of Unicon by involving as many competitors as possible and making the event spectator-friendly for other Unicon participants as well as non-unicyclists.
Any of the following are examples to achieve this goal:
\begin{itemize}
  \item Workshops related to the event
  \item Fun competitions based on the event
  \item Instant results for the spectators
  \item Ways for other competitors to be introduced to the event
  \item Entertainment during breaks in the competition (such as half time entertainment)
  \item Schedule of the events posted in multiple places
\end{itemize}

\section{Host's Option - Unicon}
Unicon should include at least one event from each of the following event groups. 
Hosts are free to add events, age groups or variations that do not appear here, as long as there is no conflict with the existing rules. 
When in doubt contact the IUF Rules Committee.
\begin{itemize}
  \item Track Racing: the required races from section \ref{sec:track_field_minimum-racing-events}.
  \item Other Racing: Road, specialty/novelty races; see section \ref{sec:track_field_alternate-optional-fun-events}.
  \item Team Games: Unicycle Hockey, Unicycle Basketball. See chapters \ref{chap:hockey} and \ref{chap:basketball}.
  \item Field events: Long Jump, High Jump, Gliding/Coasting. See section \ref{sec:track_field_jumping-events}.
  \item Non-competition events: workshops, fun games, sightseeing rides, muni rides.
  \item Artistic events: Freestyle, Standard Skill, Flatland, Street; see chapters \ref{chap:freestyle} and \ref{chap:flat-street}.
  \item Muni: Cross Country, Uphill, Downhill, Trials; see section \ref{sec:track_field_muni} and chapter \ref{chap:trials}.
\end{itemize}

\subsection{Safety Equipment}
You will find the detailed rules about safety equipment in each chapter as point x.3 (2.3, 3.3, 4.3 etc.).
Hosts may only deviate from these rules for safety equipment if this is inevitable. The status of ``inevitable'' has to be documented and must be approved by the IUF executive board. 
Any deviation from the IUF safety equipment requirements must be approved and announced at least two months before the event.

\subsection{Combining Age Groups \label{subsec:general_host's-option-unicon_combining-age-groups}}
In a competition with more than 50 riders, six riders are needed to complete an age group.
In competitions with less than 50 riders, six in each age group are still highly recommended, however three riders are the minimum to complete an age group.
Riders generally enter all events with their age group except for events similar to artistic competitions where there is Junior Expert and Expert categories.

The convention host must combine age groups with less than six riders (three riders for smaller conventions) if needed.
This means that published age groups are not guaranteed.
This can be done on a per-event (= per-discipline) basis.

When combining, combine the smallest age group (that is, the age group with the smallest number of participants) with its smallest neighboring age group (either up or down).
If more than one age group is the smallest, choose the age group with the smallest neighbor for combining.
Continue this process until all resulting age groups (combined and/or original) have at least the minimum required/recommended number of participants.
Male age groups are never combined with female age groups.

\subsection{Awards}
The type, number, and quality of awards are the choice of the convention host. 
Because awards are paid for out of the convention budget, the host may determine the amount and level of those awards. 
Generally we have trophies for``top'' events, medals for ``sub-top'' events, and ribbons or certificates for lower events or places. 
The IUF has most frequently awarded 1-3 place in most events, but this too is up to the convention host.

\subsection{Video recordings for protest}
Private videos are generally forbidden as a means of verification in case of a protest. 
The host can decide to make an official video of some competitions (e.g. at the 5-meter-line of the 50m one-foot race), which he has to announce before the competition to let the competitors know about their option to protest through this video.

\subsection{Sponsors}
The convention host has the option to seek and obtain private sector Sponsorship; e.g. The Unicycle.com Freestyle Awards, the Coca-Cola Hockey Cup, etc. 
This will allow opportunities for external funding to defray costs for host organizations and competitors. 
Sponsors are limited to organizations that would not bring the IUF into disrepute and are consistent with the aims and objectives of the International Unicycling Federation, Inc.

\section{Notification, Disclosure, and Communication}
Convention dates and other information must be announced and/or published at the earliest possible date. 
The best way to control the publication of convention information is with a convention web site, with regular updates to provide all the latest information. 
For Unicon and other large events, registration forms should be made available no less than eight months before the convention start date. 
A list of all planned competition events, including all rules and information pertinent to quality training, should be published at the same time with newly available data to be added as soon as it is known. 
Wherever possible, hosts should provide maps, directions and other information to help make peoples' convention as enjoyable as possible.

\subsection{Providing Special Rules}
For international competitions, written rules are needed for any planned events not described in the IUF Rulebook, and for events where additional rules are required. 
These special rules could be variations on the optional events found in this Rulebook. 
Such rules should be published at the same time as registration forms, or earlier, and must be published at least one month before the start of the event. 
These rules can be published along with registration forms, and/or on the convention web site. 
Competitors need to know the specific rules so they can train for those specific events! 
Hosts also need to decide on rules early, so there is less to worry about near competition time. 
Rule changes may be a necessary reality, for reasons such as changes in venue, weather or available equipment. 
When this happens these changes must be posted to the convention web site immediately. 
Examples: Dismount rules or timing details for off- track races, obstacle information for Street Comp, planned age divisions or combination awards.

\subsection{Full Disclosure of Course Details}
Details of all non-track racing events, or other events with unique courses or details must be published as soon as they are known. 
This is to provide competitors with the information they need to train, and to help them prepare the appropriate unicycles.
These are major needs for attendees from far away. 
Necessary details depend on the event, but include things like course length, elevation and elevation change, steepness, level of terrain difficulty, amount of turns, riding surfaces, course width, etc.
Maps should be provided when possible. 
While sometimes courses cannot be planned until weeks or days before the convention, as soon as they are known the details must be posted to the convention web site and/or all places where convention information is posted. 
It is acceptable to publish tentative courses while waiting for permits to be approved, etc.

\subsection{Course Availability for Practice}
If the course is open for practice to all riders for at least 7 days leading up to the event, then there are no restrictions on who can compete. 
If the course is not open for practice until the day of the event, then anyone who has pre-ridden the course is not allowed to compete. 
Organizers must therefore ensure that course marking and set-up are done by non- competing staff/volunteers.

\subsection{Communication}
In international events especially, good communication makes the difference between a memorable event, and frustration for many. 
Hosts must cultivate good lines of communication to attendees, both before the convention starts and once people have arrived. 
Team mailboxes, contact persons, centralized phone numbers or an organized method must be used to keep people aware of schedule changes, venue changes, last-minute details, etc.

\subsection{Disclaimers, Cancellations}
The host reserves the right to make changes, if necessary, to ensure the success of a convention or competition. 
Sometimes these changes must be made at the last minute, such as in switching outdoor events for indoor in the event of rain. 
Sometimes activities must be cancelled due to events beyond the host's control, such as weather or power outages. 
When changes or cancellations are made, notification must be posted, communicated and/or distributed as early as possible.

\section{Publishing Rules}
If competition events or games not found in the IUF Rulebook are planned, written rules must be provided. 
These rules, if not pre-existing, should be published at the time of announcement of those events. 
This generally means at or before the posting of registration forms. 
For competitors to properly train, and be on an equal footing with local riders, all must be aware of the rules to be used.

\section{Names And Terminology}
Event hosts must learn and use the proper names and terminology for our sport and competition events. 
They should take care not to continue the misuse of outdated or incorrect names and terminology. 
The correct ones must be used in all announcements, advertising, publicizing, internal and external documents, and especially in any official documents, such as those within, and printed out by, convention software. 
For example, the specific artistic event names are Individual Freestyle, Pairs Freestyle, Group Freestyle, Flatland, Street Comp, and Standard Skill. 
Note that the word Artistic is not part of any of the individual event names.

While we call our event ``Unicon'' (Unicycling Convention), remember this word is unfamiliar to the general public. 
Remember to spell out the full name of your event so people know what it's about. 
If it doesn't say unicycle or unicycling, the general public may not know what your event is about.

\section{World Champions}
The Male and Female winners of each individual event at Unicon are the World Champions for that event. 
There is no age limit to winning the overall title.

Age group winners can use the title `Age-Group Winner', and the term `World Champions' generally refers to winners of Overall, Finals or Expert class. 
See also section \ref{sec:freestyle_world-champions}.


\section{World Records, IUF Records}
Competition results cannot break existing world records if they were not conducted and recorded under the conditions required by the IUF.
The host should ensure that the conditions while the competitions are according to the IUF rulebook and the IUF World Record standards.

\section{Ownership of Convention Data}
Each Unicon or other large unicycling convention is a piece of history.
At the conclusion of a Unicon or other international event, or within one month thereafter, the convention host must supply the IUF with a list of competition and other results.
This list will include all data collected to determine placement and winners at all levels and in all events held at the convention.
This data is considered public, and is not the sole property of the host.
Copies of attendee registration details, judging sheets, protest forms, and related paperwork are not necessarily public, but are the shared property of the host and the International Unicycling Federation, and must be made available upon request.
If the host wishes to discard any of this paperwork or data, it must be turned over to the IUF, not thrown away.
If requested, the host and convention officials must also provide further information, not necessarily in writing, about decisions made, methods used, and other details covered in the process of planning and running the convention.
This information can be invaluable to future hosts, and must not be hidden or lost.

\section{Publishing Results}
Results of national and international championships must be published including details such as time, distance, total score and score per judge.
For each event, the names and represented nationality of competitors as well as the names and nationality of all officials shall be published.
In the artistic events, countries and/or names of the entire judging-panel must be published.
However score per judge may be replaced by J1, J2, J3, etc.
if desired by any of judges.

An official protest/correction form must be available to riders at all times.
All protests against any results must be submitted in writing on the proper form within two hours after the results are posted, unless there is a shorter time specified for certain events (for example: track racing).
The form must be filled in completely.
This time may be extended for riders who have to be in other races/events during that time period.
Every effort will be made for all protests to be handled within 30 minutes from the time they are received.
Mistakes in paperwork and interference from other riders or other sources are all grounds for protests.
Protests handed in after awards have been delivered will not be considered if the results have been posted for at least three hours before the awards.
If awards are delivered before results are posted, it is recommended to announce the schedule of posting and the deadline for protests at the awarding ceremonies.
All Chief Judge or Referee decisions are final, and cannot be protested.

\section{Registration Forms}
Because of the various options available to riders in different events, riders may enter different events in different age groups.
A properly structured registration form is essential for making these choices clear to the participants.
For example, a rider may enter Pairs as an Expert with an older rider, but may wish to compete in Individual Freestyle in his or her own age group.
Before publishing, a Unicon registration form should be examined and approved by members of the IUF Rules Committee or Board of Directors.
No rider may enter any event until his or her registration form has been completed, including payment and completion of waivers and/or signatures.
No minor may compete until a parent or legal guardian has signed his or her release.

\section{Program Book}
At Unicons, all registrants shall be provided with a package of pre-printed information containing a full schedule of all events, maps and directions to all event locations, and as much rule and background information as possible.
This information shall be provided when registrants first check in at Unicon.
Unicon organizers should consider placing as much of this information as is practical in an official Program Book.
This can make excellent reading for family members and spectators, and gets them more involved in our sport.
It's also a great place to sell ads as a source for convention revenue.
At other unicycling events, it is recommended that pre-printed information be provided to all participants.

%Racing Facilities
\section{Track}
A track must be made available for conducting the track races.
The track must be marked in meters, and should be prepared in advance with start and finish lines for the various racing events that are unique to unicycle racing (such as 50, 30, 10 and 5 meter lines).
In addition to the track, a smooth area of sufficient size must be set aside to run the IUF Slalom (and Slow Races, if held).
A public address system must be provided to announce upcoming events and race winners.
Bullhorns are usually not adequate for the track environment.

\section{Weather}

\subsection{Track Racing}
If the track is outdoors, plans must be made to deal with inclement weather.
Using an indoor track can eliminate this problem.
The track must be available for enough days to allow for inclement weather 
\subsection{Trials, Street Comp and Flatland competitions}
In the Trials and Street Comp, the organizers should postpone the events and exchange all the affected parts of the course for dry ones (replacing pallets for example).
These competitions should be canceled if considered dangerous for the riders.
If postponed or moved to an indoor location the organizers must try to keep the allowances the same as outdoors competitions (Metal pedals allowed for example).
If originally on the competition schedule, these canceled competitions should be rescheduled during the convention duration.
The event host should try to place events that may be influenced by weather conditions in the first days of the event, giving a larger period of time to reschedule it.

%Indoor Events Facilities
\section{IUF Public Meeting}
The host will provide time in the convention schedule for the IUF Public Meeting.
At this meeting, the IUF will elect officers or other volunteers, and otherwise do business and encourage the opinions and assistance of all interested convention attendees.

The meeting time should be as close to the end of the convention as possible, excepting on the final day, as people may have to leave before that time.
At minimum, the meeting should be during the second half of the convention.

A minimum of two hours should be allocated, during which no other official convention events, other than open gym or other informal activities, should take place.

A meeting room must be provided that has adequate space/seating, lighting and acoustical properties to communicate and conduct the meeting.
A lecture hall or theater are optimal locations, and a sound and/or projection system would be very helpful.

Other IUF meetings may be held during the convention, both public and private, but the strict requirements apply only to the big public meeting.

\section{Artistic Riding Areas}
Traditionally a gymnasium is used.
Artistic competitions are also possible in an auditorium, if the stage is large enough.
If this is done, a gym must also be available for practice, and possibly for group competition.
Gymnasiums used for competition should have enough room to set up two Individual and Pairs performing areas side by side.
There must also be enough room for judges and spectators.
Seating must be provided for spectators, and a practice area must be provided for riders.
Ideally, this practice area would be in a separate gym.
The primary practice area cannot be outdoors, as wet or extreme weather would prevent riders from warming up and exchanging skills.
If necessary, the practice area can be behind a curtain in the competition gym, or behind the spectator seating.
Neither of these solutions is as desirable, due to the distraction that is unavoidably caused by riders using these areas.

The gym or riding surface must be marked with the boundaries of all riding areas for individual and pairs events.
In some facilities black tires, metal pedals, untaped wooden hockey sticks, etc.
might not be allowed.
The host must make sure the participants are informed of this in advance.
All performing and practice areas must be in well-lit places that are protected from the weather, or have fallback locations in case the weather is bad.

It is very important that a good quality public address system be available for announcements, and to play competition music.
At least two music-playing devices must be provided (one as a backup or test machine).
These should be compatible with all the media types specified for the various events to be held there.

\section{Open Practice Area}
At least one area with a smooth safe riding surface, sheltered from the weather, should be made available for all or part of the day on most or all days of the convention.
These areas are to be used for non-competition events such as workshops, skills exchange and free practice.

\section{Materials \& Equipment}
The Host must supply all necessary materials and equipment to run the competitions, such as a timing system, starting posts, cones for the IUF Slalom, etc.
Don't forget the more obvious things, such as paper and writing materials, judging tables, printers, basketballs, hockey sticks, etc.

\section{Training Officials}
As the rules state, competitions cannot be started until all key track and artistic officials have been trained and understand their tasks.
For Racing, the Referee is in charge of making sure this happens.
For Artistic events, the Chief Judge is in charge.
The host must make sure there are plenty of copies of the rulebook for officials to study on the spot.
Testing can consist of a simple verbal quiz, or anything the Referee or Chief judge deem appropriate.
For certain artistic events, a minimum level of judging experience is required.
See section \ref{sec:freestyle_judging-panel}.

%Responsibilities of Individual Participants
\section{Nations Represented}
For events where the number of participants is limited by country, there may be some question of what country a rider, pair or group may represent.
Riders must represent the country in which they hold citizenship, or in which they are a legal resident.
For example, if a rider is attending school in a different country, and is in that country legally, the rider can represent that country, \textit{or} the rider's home country.

If necessary, citizenship or residence may be established with a passport, driver's license, or legal ID for the country the rider wishes to represent.
Riders on extended vacation, exchange students, and other temporary residents of other countries are not eligible to represent those countries, except in multi-rider events (see below).

For Pairs Freestyle or other two-person events, the pair can represent any country that either rider is eligible to represent.

For Group Freestyle, sports teams or other multi-rider events, the group must represent the country that the greatest number of the group's riders is eligible to represent.
If there is a tie in this number, the group can represent either of the tied countries.

\section{Racing}
Riders must use unicycles that conform to the definitions and dimensions for racing unicycles.
Riders must have kneepads, gloves and shoes that meet the definitions below, and helmets for certain events.

\section{Artistic}
Any performance music must be provided on CD, or only those other media types supported by the event host.
See also section \ref{sec:freestyle_music}.

\section{Personal Responsibility}
A parent, guardian or other designated person, must supervise all minors.
All attendees should remember that they are guests of the convention hosts, and \textit{ambassadors} of our sport to all new riders, visitors from far away, and to people in the hosting town.
Remember that the Host is \textit{renting} the convention facilities, and attendees are expected to treat them well.
Each rider is responsible for the actions of his or her family and non-riding teammates.
Riders may lose placement in races, risk disqualification from events, or be ejected from the convention if they do not work to minimize disruptions from these people.

\subsection{Option to Remove People From Events}
The host is allowed to remove an individual or a group if they are acting aggressively or abusively against others.
These individuals/groups should be given a first warning, followed by removal from the specific event by the host or the Chief Judge/Referee who is in charge for the competition where the problem appears.
The person(s) should only be removed from that competition to have a chance to calm down.
If the aggressive or abusive behavior continues, it is also possible to remove the individual or group from the rest of the convention.

\section{Knowing The Rules}
Lack of understanding of rules will be at the disadvantage of riders, not officials or the IUF.
The IUF is also not responsible for any errors that may occur in the translation of rules and information into languages other than those in which they were originally written.

\section{Your Privilege}
Entry in the competition is your privilege, not your right.
You are a guest at the Host's event.
You may be in an unfamiliar country, with different customs that are considered the norm.
The Host and convention officials determine whether certain events, age groups, or policies will be used.
As an attendee, you are obligated to obey all rules and decisions of convention officials and hosts.

\section{Definitions \label{sec:general_definitions}}
\textbf{Age:} Rider's age for all age categories is determined by their age on the first day of the convention.

\textbf{Expert:} The top category in events that don't have a system to determine Finalists.
When no other limitations are present, riders can choose to compete in this category against the other top riders.
Limitations on this may be if top riders are chosen at previous competitions, such as national events, or if there is a limit on the number of competitors per country.
The category is called Expert, and riders entered in it can be called Experts.
The distinction of Experts over Finalists is that they are not chosen based on competition results at the current convention.

\textbf{Figure:} (noun) 1. A unicycle feat or skill, such as walking the wheel or riding backward, used to describe skills in the Standard Skill event.
2. A riding pattern, such as a circle or figure 8.

\textbf{Finalist, Finals:} A Finalist is a person, and ``the Finals'' is the last category or group in any event that has multiple rounds.
For example in Track racing, the top riders from the age groups compete against each other in the Finals of most events.
{Footwear For Racing:} Shoes with full uppers are required.
This means the shoe must cover the entire top of the foot.
Sandals or thongs are not acceptable.
Shoelaces must not dangle where they can catch in crank arms.

\textbf{Gloves:} (For racing) Any glove with thick material covering the palms (Leather is acceptable, thin nylon is not).
Gloves may be fingerless, such as bicycling gloves, provided the palm of the hand is completely covered.
Wrist guards, such as those used with in-line skates, are an acceptable alternative to gloves.

\textbf{Helmet:} Helmets are mandatory for races on Unlimited unicycles, and any other unicycles larger than 29$"$.
Helmets are also mandatory for Backward races, anything downhill, all Muni, Trials and Street events, High Jump, Long Jump, and possible other non-traditional races.
They are recommended for all races not mentioned here.
Helmets must be of bicycle quality (or stronger), and should meet the prevalent safety standards for bicycle (or unicycle) helmets, such as ASTM, SNELL, CPSC, or whatever prevails in the host country.
Helmets for sports other than cycling or skating are not permitted, unless the Referee makes exceptions.

\textbf{IUF:} International Unicycling Federation.
The IUF sponsors and oversees international competitions such as Unicon, creates rules for international competition, and promotes and provides information on unicycling in general.

\textbf{Junior Expert:} Same as Expert, but open only to riders age 0-14.
Riders in this age range may optionally enter Expert instead, to compete in the highest/hardest category.
{Kneepads:} (For racing) Any commercially made, thick version is acceptable, such as those used for basketball and volleyball, or any with hard plastic caps.
Kneepads must cover the entire knee and stay on during racing.
Long pants, ace bandages, patches on knees, and Band-Aids are not acceptable.

\textbf{Muni:} Mountain unicycling, or mountain unicycle.
The previous term for this was UMX.
{Non-unicycling Skills:} (for Freestyle judging) The riding of any vehicle with two or more wheels on the ground, and any skills not performed on a unicycle.
Any skill with more than one support point on the riding surface, such as standing on the unicycle with it lying on the floor, or hopping while standing on the frame (seat on floor); two contact points with the riding surface (wheel and seat), both carrying part of the rider's weight.
The term also refers to skills such as dance, mime, comedy, juggling, playing music or riding vehicles that do not meet the definitions of unicycles.

\textbf{Prop:} Almost anything other than the unicycle(s) being ridden by competitor(s) in a Freestyle performance.
A unicycle being used for a non-unicycling skill (such as a handstand on it while it's lying down) is a prop at that moment.
A hat that is dropped and picked up from the floor is a prop.
A pogo stick or a tricycle (unless ridden on one wheel) is a prop.
{Standard Unicycle:} Has only one wheel.
Is driven by crank arms directly attached to the wheel's axle/hub, with no gearing or additional drive system.
Pedals and cranks rotate to power the wheel.
Is balanced and controlled by the rider only, with no additional support devices Brakes and extended handles/handlebars are permitted.
For some events, such as track racing, standard unicycles have restrictions on wheel size and/or crank arm length.
Other events may specify other restrictions.
When not noted otherwise, there are no size limitations.

\textbf{Unlimited unicycle:} Multiple wheels are permitted, but only one may touch the ground and nothing else.
Is human powered only.
Gearing and/or a transmission are allowed.

\textbf{Ultimate wheel:} A special unicycle consisting of only a wheel and pedals, with no frame or seat.

\textbf{UMX:} Unicycle Motocross.
This term has for the most part been replaced by muni.

\textbf{Unicycling skill:} (noun, for Freestyle judging) Also known as ‘figure.' Any skills (feats of balance) performed on a vehicle with one support point in contact with the riding surface, this being a wheel, the movement of which is controlled by the rider, thus maintaining balance.
All mounts are also ‘unicycling skills.' 

\textbf{Unintentional dismount:} In most cases, any part of a rider unintentionally touching the ground.
For example, a fingertip on the floor while spinning is not unintentional.
A pedal and foot touching the ground in a sharp turn is not a dismount as long as the foot stays on the pedal while the pedal is on the ground.
Dismounts during many races disqualify the rider.

\textbf{Unicon:} Unicycling Convention.
This word usually refers to the IUF World Unicycling Championships conventions.

\textbf{Wheel walking:} Propelling the unicycle by pushing the top of the tire with the feet.
Feet touch wheel only, not pedals or crank arms.
A non-pushing foot may rest on the fork.


\section{Convention Officials}
These people make the competition events work.
All of the tasks detailed below must be covered for the events to work.
Names must be assigned for all the jobs listed below, to create a hierarchy of authority for the convention.
All officials are expected to work objectively and impartially.

\subsection{General Officials}
\textbf{Convention Host:} This is a single person, or a collective group, that has made the commitment to host a unicycle convention using IUF rules and guidelines.
By agreeing to host an IUF convention, they also agree to follow those rules and guidelines wherever possible.
If known problems arise in the arrangement of facilities, schedules and events, the Host and the IUF will work together to resolve the problems.
For the most part, the Convention Host is the ultimate authority for what happens, and does not happen, at the convention.
The exception is any IUF requirements for convention facilities or contents, and rules for IUF competition events.

\textbf{IUF Board Of Directors:} The IUF Board represents the interests of the IUF on convention requirements, both in the area of competition rules and the necessary spaces and facilities for them, and for any other requirements that go along with putting on an IUF convention.
If problems arise in meeting the IUF requirements, the IUF Board and Convention Host work together to find solutions or compromises.
The bulk of this should happen during the early planning stages for a convention, when facilities and schedules are being assembled.

\textbf{IUF Convention Liaison:} The Liaison is an optional person who can represent the IUF Board when communicating with convention hosts.
The Liaison essentially has the same powers as the IUF Board, but must report to the IUF Board and take direction from it.

\textbf{Top Competition Officials:} The Race Director, Referee, Artistic Director and Chief Judge are the positions of authority for racing and artistic events, respectively.
They are not autonomous, and must answer to the Convention Host.
It is highly recommended that none of these jobs be combined, and that there be at least one separate person for each.

\subsection{Racing Officials}
\textbf{Race Director:} The Race Director is in charge of seeing that all equipment, forms, people, sound systems, and other requirements are taken care of before the convention starts.
Ideally, the Race Director is a member of the host organization, or has convenient access to the convention's locations.
The Race Director is responsible for the logistics, equipment, and scheduling for all racing events unless otherwise noted.

\textbf{Referee:} The Head racing official.
Makes all final decisions regarding race competitions.
Handles protests.
Makes sure racing areas and officials are trained and ready.
Works within the system set up by the Race Director for running the events.
The Referee should be someone very experienced in all aspects of unicycle racing, and must above all be objective and favor neither local, nor outside riders.
There can be separate Referees for different venues, or different categories of racing.

\textbf{Clerk:} Sets up riders in lanes before races.
Checks riders for correct unicycles and safety equipment.

\textbf{Starter:} Starts races; explains race rules; calls riders back in the event of false starts.
Also checks riders for correct unicycles and safety equipment.

\textbf{Picker:} Assists Timers by observing riders' finishing order.
Watches for finish line dismounts.

\textbf{Timer:} Takes the time of riders at the finish line.
Also watches for finish line dismounts.
Two timers may be used for 1st place, the average time being official.

\textbf{Recorder:} Writes down place and time of each rider after each event.
Riders must not leave the finish area until the Recorder has gotten their numbers and information.

\textbf{Runner:} A general helper who brings racing forms from start to finish line, to tabulators, and to announcer.

\textbf{Tabulator:} Processes all race results; totals points; prepares awards for presentation.

\textbf{Results Poster:} Puts up Race Recording and Results Sheets for all to see, and marks the time.

\textbf{Announcer:} Operates public address system, announces race results, and calls riders for upcoming races.

\textbf{IUF Slalom and Slow Race Operators:} Run and administrate these two events in an area separate from the track, doing most of the above jobs for each.

\subsection{Artistic Officials}
\textbf{Artistic Director:} The head organizer and administrator of artistic events.
The Artistic Director's job starts well before the convention, arranging equipment for the gyms (or performing areas) and recruiting the other artistic officials.
With the Convention Host, the Artistic Director determines the operating systems, paperwork and methods to be used to run the events.
With the Chief Judge, the Artistic Director is in charge of keeping events running on schedule, and answers all questions not pertaining to rules and judging.
The Artistic Director is the highest authority on everything to do with the artistic events, except for decisions on rules and results.

\textbf{Chief Judge:} Like the Referee, the Chief Judge should be a thoroughly experienced person who must above all be objective and favor neither local, nor outside riders.
The Chief Judge must be thoroughly familiar with all of the artistic officials' jobs and all aspects of artistic rules.
The Chief Judge oversees everything, deals with protests, and answers all rules and judging questions.
The Chief Judge is responsible for seeing that all artistic officials are trained and ready, and that the artistic riding areas are correctly measured and marked on the floor.
The Chief Judge is also responsible for the accuracy of all judging point tabulations and calculations.

\textbf{Timer:} Keeps the time for all performances, and makes acoustic signals at key points in performances.

\textbf{Judge:} Rates the performances.
The various artistic categories require different judging qualities, and may use different judges.
All judges must be completely impartial, and must understand the rules and judging criteria.

\textbf{Tabulator:} Processes all judging sheets and fills out final results sheets and other forms.

\textbf{Runner:} A general helper who transfers forms and other information from one place to another.
\textbf{Announcer:} Operates sound system with DJ, and announces all upcoming riders and results of competitions.
May also provide color commentary between performances.

\textbf{DJ:} Operates sound system, plays all performance music, and keeps track of riders' music media and any special instructions.

\textbf{Rider Liaison:} Checks in riders before they compete.
Determines performing order.
Makes sure riders' music is properly marked and is otherwise prepared.

\textbf{Stage Crew:} Helps riders set up, tear down, and clean up after themselves.

\textbf{Results Poster:} Puts up artistic results sheets for all to see, and marks the time.
