\chapter{Flatland Street \label{chap:flat-street}}

\section{Difference Between These Events}
In Flatland, riders perform in a flat area with no obstacles or props.
There is no judging of music and costume, and the emphasis is on originality and creativity.
Street is about using the environment, such as ramps, rails, stairs and platforms, to do tricks in the style of skateboarding and.
Riders are judged on the skill and creativity of moves and combinations they do.

\textbf{Note:} These rules apply to both Flatland and Street unless otherwise noted.

\section{Deadline For Signing Up}
These events have a deadline for entry, which must be specified in the registration form.
If not specified in the registration form, the deadline is one month before the official convention start date.
A maximum of ten Individuals for each event will be allowed to be added after this time to account for difficulties in travel planning or other valid reasons that are communicated about in advance.
These will be added in the order of their request to the Chief Judge and Convention Director via email or fax.
Participants who attempt to sign up less than 36 hours prior to the beginning of the specified competition will not be allowed to enter.

\section{Size Of Performing Areas}
Hosts must publicize the dimensions of the available performing area as far in advance of the competition as possible, and organizers of international championships at least three months prior to the event.

\subsection{Street Comp Performing Area \label{subsec:flat-street_size-of-performing-area_street-comp}}
The street course is to be composed of three ``zones''.
Each zone should have multiple obstacles, but each obstacle should encourage a specific type of skill.
The list below is an example of three typical things that can be used for the zones; however designers of the street comp area should not limit themselves to the exact list.

\textbf{Zone 1:}
A ramp with a skate park rail in the middle, and a ledge on either side.
This zone will encourage technical grinds, without giving an advantage to a right of left footed grinder.

\textbf{Zone 2:}
Two different manny pads (a smooth platform of at least 3m x .5m and between 7cm and 15cm in height), one with two revs of length, and one with just one rev of length.
This will encourage the ability to perform technical flip tricks and other street moves while having to set up quickly for the move down.

\textbf{Zone 3:}
A set of 5 stairs and a set of 7 stairs with a handrail in the middle of each (that are of a similar size to one that you would find in a city, not extremely steep).
This section would encourage the ability to perform bigger moves of all types.

\subsection{Street Comp: Problems With Required Obstacles}
The required obstacles must be built strong enough to endure many hours of heavy use.
They need to survive the competition without changing their shape or stability.
If one of the required obstacles is broken or made unusable during the competition, it must be repaired if one or more competitors say they need to use the damaged part.
If no competitors have a problem with the damage, no repair is necessary except for safety reasons, such as in the event of sharp exposed parts. 

\subsection{Flatland Competition Performing Area}
A 11 x 14 meter area is required.
The audience may be as close to the boundaries as possible.
It can be done indoor and outdoor depending on the host's possibilities and weather conditions.
For indoor competitions the host should think about the free use of unicycles and protect the ground. 

\subsection{Riders Must Be Ready}
Riders who are not ready at their scheduled performance time may or may not be allowed to perform after the last competitor in their age group.
The Chief Judge will remember to consider language barriers, and that riders may be engaged in convention work to slow them down.
A rider may not perform before a different set of judges than those that judged the rest of their age group.

\section{Interruption Of Judging}
An interruption of judging can result from material damage, injury or sudden illness of a competitor, or interference with a competitor by a person or object.
If this happens, the Chief Judge determines the amount of time left and whether any damage may be the fault of the competitor.
Re-admittance into competition must happen within the regulatory competition time.
If a routine is continued and the competitor was not at fault for the interruption, all devaluations coming forth from the interruption will be withdrawn.

\section{Music}
In Flatland, competitors may optionally bring their own music but is not judged.
For Street Comp, background music will be played.

\subsection{Media Types}
The host is required to have the capability of playing recordable CDs.
Other media types may also be supported, at the host's discretion.
The Artistic Director is responsible for announcing what media types will be supported, and making sure the necessary equipment is provided.

\subsection{Music Preparation}
Competitors who bring music must provide it in a form that is supported, and has been announced by the Artistic Director.
All music must be clearly labeled with the competitor name, age group and the track number.
Whenever possible, competition music should be the first track on the CD.
The DJ (music operator) is not responsible for any errors resulting from unsupported types or mislabeled tracks.

\subsection{Music Volume}
Volume level is controlled by the DJ, at instructions from the Chief Judge.
The base volume should be loud enough to sound clear, and be heard by all.
Some music may start with especially loud or quiet sections, and the DJ should be advised of these so volume levels do not get compensated in the wrong direction.
Some competitors may request that their music be played at lower levels.
These requests can be made directly to the DJ.
Requests for higher volumes must be approved by the Chief Judge, who has the option of passing this responsibility to the DJ.

\section{Announcing Of Results}
Final results will be continuously announced and/or posted for public view.
Results Sheets will be posted after each age category of an event.
The protest period begins at this point.

\section{Protests}
Must be filed in writing, within 15 minutes from the posting of event results.
Protest against judges' scores is not permissible.
Protest is only possible against calculation mistakes or other mistakes not connected to the scoring.
The Chief Judge must resolve all protests within 30 minutes from receipt of the written form.

\section{Judging Panel}
For Flatland, there must always be an odd number of judges.
For Street, there are three judges for the preliminary rounds, and five judges for the finals.

\subsection{Selecting Judges}
A person should not judge an event if he or she is:
\begin{enumerate}
\item A parent, child or sibling of a rider competing in the event.
\item An individual or team coach, manager, trainer, colleague who is member of the same club specified in the registration form, colleague's family etc. of a rider competing in the event.
\item More than one judge from the same family judging the same event at the same time.
If the judging pool is too limited by the above criteria, restrictions can be eliminated starting from the bottom of the list and working upward as necessary only until enough judges are available.
If there are some candidates who have the same level of restrictions and judging score, their agreement about publishing the results need to be considered.
The eliminations must be agreed upon by the Chief Judge and Artistic Director, or next-highest ranking artistic official if the Chief Judge and Artistic Director are the same person.
\end{enumerate}

\subsection{Standard Skill Vs. Freestyle Vs. Flatland or Street Comp Judging}
With entirely different sets of rules, qualified judges for Standard Skill are not necessarily qualified to judge Freestyle, the Street Comp, Flatland, and vice versa.
Judges' qualifications must list the types of events they are qualified to judge.

\subsection{Judging Panel May Not Change}
The individual members of the judging panel must remain the same for entire age groups; i.e. one judge may not be replaced by another except between age groups.
In the event of a medical or other emergency, this rule can be waived by the Chief Judge.

\subsection{Rating Judge Performance}
Judges are rated by comparing their scores to those of other judges at previous competitions.
Characteristics of Judging Weaknesses:
\begin{itemize}
\item \textbf{Excessive Ties:}
A judge should be able to differentiate between competitors.
Though tying is most definitely acceptable, excessive use of tying defeats the purpose of judging.
\item \textbf{Group Bias:}
If a judge places members of a certain group or nation significantly different from the other judges.
This includes a judge placing members significantly higher or significantly lower (a judge may be harsher on his or her own group members) than the other judges.
\item\textbf{Inconsistent Placing:}
If a judge places a large number of riders significantly different from the average of the other judges.
\end{itemize}

\subsection{Judges Workshop}
The hosts of the convention must provide for a judge's workshop at least 24 hours prior to the start of the first competition.
A minimum of 3 hours must be set aside, in a classroom or similar environment.
If possible, it is strongly recommended to have more than one workshop to accommodate schedules.
Variations on this can be approved by the Chief Judge.
Workshop schedule(s) must be announced to all judges at least three weeks prior to the start of the competition.

Judges should have read the rules prior to the start of the workshop.
The workshop will include a practice judging session.
Each judge will be required to sign a statement indicating they have read the rules, attended the workshop, agree to follow the rules, and will accept being removed from the list of available judges if their judging accuracy scores show Judging Weaknesses.

\section{World Champions}
Winners in each event are the \textbf{World Champions}. Separate titles are awarded for male and female, unless only one competition group is offered.

\section{Flatland Overview}

\subsection{Age Group:}
Junior (0-14) and Senior class (15-UP), male/female separated (3 riders are the minimum requirement for each category).
If there are less than 3 riders for one of the categories, those riders will compete in the older age groups.
If there are less than three females or less than three males overall, the male and female categories are merged.

\subsection{Time Limit:}
Preliminary round: Two minutes.
Competitors are allowed to finish a line that was started before the limit elapsed, and as long the line is continued without interruption.
If more than 20 competitors are present, the chief judge may choose to reduce the time to 1:30 if time restrictions are present.

\subsection{Definitions:}
A Flatland skill is any unicycle skill performed on a flat surface.
Flatland encourages riders to demonstrate a high level of technical difficulty and variety, as well as combinations and transitions between skills.
Skills typically known from freestyle should be judged with equal weight.

\subsection{Battle Finals:}
Each battle will last two minutes total.
No rider may ride for over 15 seconds at one time without allowing the other rider to perform.
A timekeeper is responsible for keeping track of time for both riders.
If a rider exceeds the 15-second limit, a beep will indicate they must dismount.
Riders do not need to ride for the whole 15 seconds; they should generally perform two to three tricks then allow the other rider to go.
The championship battle will last three minutes.

\subsection{Unicycles:}
Standard unicycles only (see definition), though any number can be used.

\subsection{Music, Costume and Props:}
Riders may provide their own music, but it is not judged.
Costume is not judged.
Props and obstacles are not allowed.

\subsection{Competition Format:}
Riders perform a two-minute preliminary run and the top riders continue on to tournament-style Battle finals.

\subsection{Battle-style Overview:}
In a flatland battle, riders compete head-to-head in groups of two, taking turns performing lines of tricks.
The winner of each battle is determined immediately following the battle by the judges.
The winner continues to the next battle and the loser is eliminated.

\subsection{Number Of Competitors Entering Battles:}
The final battles will consist of the top 4 or 8 riders, based on their scores in the preliminary round.
If the judges consider there to be less than 8 top-level riders, the 4 with the highest scores from the Preliminary Round will advance to the Finals.
If there are 4 or fewer riders competing, there will be no battles and the results from the preliminary runs will be final results.

\subsection{Final Tricks:}
Preliminaries: With 15 seconds left, the announcer will announce ``last trick.''
At that time, the rider will attempt their final trick.
If they fail, the rider will be given one more attempt.
The rider is not obligated to try the same trick in every attempt.
A final trick cannot last longer than 15 seconds to complete.

\subsection{Final Trick In Battles:}
Once the 2-minute battle is completed, the announcer will announce that it is time for the final tricks.
Each rider has three attempts to land their final trick.

\section{Flatland Judging and Scoring \label{sec:flat-street_flatland-judging-scoring}}

\subsection{Number of Judges}
There must always be an odd number of judges to prevent ties. 

\subsection{Performing Area}
A 11 x 14 meter area is required.
The audience may be as close to the boundaries as possible.
It can be done indoor and outdoor depending on the host's possibilities and weather conditions.
For indoor competitions the host should think about the free use of unicycles and cover the ground.

\subsection{Preliminaries}
Difficulty, consistency, variety, and last trick contribute to the total score.
Scoring: A total of 40 points is possible.
Higher numbers are better scores.
The judges will add up all scores for each competitor and rank then accordingly.
Rankings from individual judges are averaged to determine overall ranking.
The points are allocated as following: 

\subsubsection{Variety/Combos:}
(Score of 1-15 is given:)
High scores are awarded to competitors who perform a wide range of tricks and combos.
Lots of repeated tricks or similar tricks will receive low scores.
Combos are combinations; the linking of two or more tricks together without returning to a neutral riding position.
Creative combos will receive higher scores than tricks done at a time.

\subsubsection{Consistency/Flow:}
(Score of 1-10 is given:)
Fewer falls relative to number of landed skills results in higher score.
Higher points are rewarded to skills completed smoothly with minimal corrective hops or drastic movements to regain balance.

\subsubsection{Difficulty:}
(Score of 1-10 is given:)
Technical, difficult tricks result in high scores only if they are completed successfully.
If a rider completes part of a flat line then falls, they are awarded points for everything they did up until the fall.
A longer line of difficult tricks deserves a higher score than a short one, or one that is broken by a dismount.

\subsubsection{Last Trick:}
(Score of 0-5 is given:)
The last trick is supposed to judge how strong the rider is (physically and mentally) in the end.
Partial points may be given for a trick that is almost landed.
The best attempt counts, other failed attempts do not subtract from the score.

\subsection{Battle \label{subsec:flat-street_flatland-judging-scoring_battle}}
Battles are judged using the same criteria as the preliminary round.
Judges must determine a winner individually, then the chief judge holds a vote to decide on the winner of that battle.
Judges are not required to write down scores for each category during battles.
If a rider repeatedly rides longer than their allowed time, distracting the audience and other rider, the judges may choose to eliminate that rider.

\subsubsection{Battle Assignments:}
In the first round of battles, the riders who competed best in the preliminary round will compete against the lower scoring competitors.
For example, 1 will battle 8, 2 will battle 7, etc.
The rest will follow the following chart. (http://uniflatland.com/flatbattles.png)

\subsubsection{Finals/Semi-Finals:}
The two competitors who make it to the last battle compete for 1st and 2nd place in the Finals.
The two competitors who lose in the second round of battles will continue to the Semi-Finals where they will battle for 3rd and 4th place.

\subsection{Flatland Scoring}
In the preliminary round, raw scores from the judges are added to determine the placing of the riders.
The highest and lowest scores are removed.
If there are two riders with equal points in places 1–8, the rider with most points in ``last trick'' get an additional fraction of a point to break the tie.
The additional fraction of a point cannot result in that rider receiving a higher score than any previously higher-scoring rider.
If the riders' ``last trick'' scores are equal, they must show a last line and the judges must vote for the best, like later in battles.
Once place 1–8 is figured out, the battles can be configured like described in section \ref{subsec:flat-street_flatland-judging-scoring_battle}.

For battles, judges must decide on a single rider to vote on, they cannot tie the riders.
If a judge feels both riders performed equally based on their judging criteria, they must look at the ``last trick.''
The rider with the best score for ``last trick'' will be the winner.
 
\section{Street Comp Overview}

\subsection{Minimum Age Groups:}
None.

\subsection{Preliminaries:}
Riders will be put into groups of three or four (preferably 4, but in some cases, there may need to be up to 3 groups of 3 depending on the number of competitors).
Each group will be given a starting time, and they will proceed to their starting Zone.
They will be given 5 minutes in each zone to perform as many tricks as possible.
The riders are assigned an order and they may only attempt a trick when it is their turn.
The order should be presented in writing as well as announced before the competition.
Riders may choose to skip their turn in the event of an injury or any other reason.
The group will then move on to the next zone (so it will take each group 25 minutes to finish, with 5 minutes after for discussion, and it will take 10n+20 minutes to finish prelims, where n is the number of groups).

\subsection{Judging:}
Each of the three judges will rank the riders (see section \ref{sec:flat-street_flatland-judging-scoring}) based on the number of tricks done, and the difficulty of the tricks.
Consistency should not be considered, because it is inevitable that a consistent rider will land the most tricks.

\subsection{Finals:}
The top 5 or 6 riders will be chosen to participate in the finals, which should be a few hours later, or the next day.
Finals should preferably not be before noon, because we want a lot of spectators, and we want to riders to have a chance to warm up and be ready to be at their best.
In the finals, the same 3 zones will be used, and all riders will go at the same time for 12 to 15 minutes (open for discussion) in each zone.
The riders are assigned an order and they may only attempt a trick when it is their turn.
The order should be presented in writing as well as announced before the competition.
Riders may choose to skip their turn in the event of an injury or any other reason.
There will be 5 judges in the finals, and these can be made up from some of the judges of prelims, or even riders that may not have performed their best in prelims, and did not make it into the finals.

\subsection{Zones:}
See section \ref{subsec:flat-street_size-of-performing-area_street-comp} for detailed description.

\subsection{Unicycles:}
Any type and any number.
Trials unicycles with metal pedals and marking tires are allowed, so this competition is generally intended for outdoors.

\subsection{Dress:}
Riders must wear the same footwear as required for unicycle racing (see Section 2.3), plus shin pads and helmets.
Riders found not to wearing the minimum required safety gear will be first called to order as a reminder and disqualified if continue to not wear the proper safety gear.

\subsection{Music:}
Music is not judged.
Background music may be played.

\subsection{Costume and Props}:Clothing has no influence on the score.
Riders are encouraged to dress in the uniform of their national teams or clubs, or in clothing that represents their teams, groups or countries.
No props allowed, other than what is included in the performing area.

\subsection{Judging Method}:Riders scored in four equal categories: Height/Distance, Technical Difficulty, Originality/Variety, and Consistency/Flow.

\section{Street Comp Judging}
There will be three judges for each zone for the preliminaries, and five judges for each zone for the final.

The judges will have 5 minutes after each session to discuss the tricks.
They should use their preferred system for taking notes during competition (for example: they may take notes of tricks that were landed in that zone or assign point values).
It is recommended to ask riders about specific tricks (other riders should be present to justify the response), but neither the riders nor judges may discuss relative difficulty as it could influence the score.
After the judges have seen ALL of the riders at a single zone, they will rank the riders from best to worst.
Each judge is responsible for one set of rankings at their zone; they must judge all riders against each other even if they are in different groups.
Each place is awarded points as follows:

\textbf{1st:} 10 points\\
\textbf{2nd:} 7 points\\
\textbf{3rd:} 5 points\\
\textbf{4th:} 3 points\\
\textbf{5th:} 2 points\\
\textbf{6th:} 1 point\\ 
\textbf{7th and Beyond:} 0 points

The ranking should be influenced by the number of tricks done, and the difficulty of the tricks.
Consistency should not be considered, because it is inevitable that a consistent rider will land the most tricks.
However, note that the number of tricks
should also not always be the deciding factor on who wins.
Some one who performs 18 easy tricks should not be scored higher than someone who performs 3 outstanding tricks.
Once the judges assign places for every zone, the points will be added and the final results can be calculated (either to decide who makes it to the finals in the case of prelims, or who wins the competition in the case of finals).